% Comment for anonymization
% Add preprint option for the extended version
% Also look for "SHORT" comments
\def\fullversion{}

\documentclass[nocopyrightspace,preprint]{sigplanconf}


% Structure:
%
% 1) Intro
% 2) Overview and Examples
% 3) Cost Semantics for Clight
% 4) Compositional Resource-Bound Analysis
% 5) Automatic Inference via LP Solving
% 6) Auxiliary Variables and User Interaction
% 7) Quantitative Hoare Logic and Soundness Proof
% 8) Experimental Evaluation
% 9) Related Work
% 10) Conclusion



\newcommand{\iffull}[2]{\ifx\fullversion\undefined{#2}\else{#1}\fi}
\newcommand{\ifshort}[2]{\ifx\fullversion\undefined{#1}\else{#2}\fi}

\newcommand{\itemskip}[0]{\ifshort{\vspace{-3pt}}{}}
\newcommand{\itemskipIn}[0]{\ifshort{\vspace{-1pt}}{}}
\newcommand{\sectskip}[0]{\ifshort{\vspace{-3pt}}{}}
\newcommand{\paraskip}[0]{\ifshort{\vspace{-2pt}}{}}
\newcommand{\lstskip}[0]{\ifshort{\vspace{-2pt}}{}}
\newcommand{\aftersectskip}[0]{\ifshort{\vspace{-1pt}}{}}
\newcommand{\subsectskip}[0]{\ifshort{\vspace{-1pt}}{}}

%SHORT
% \usepackage[normal,belowskip=-11pt,aboveskip=1pt]{caption} %Remove extra space around captiosn
% \DeclareCaptionFormat{myformat}{#1#2#3\vspace{-4pt}\hrulefill}
% \captionsetup[figure]{format=myformat}
% \captionsetup[table]{format=myformat}
% \renewcommand{\captionlabelfont}{\bf}
%SHORT

\usepackage[numbers]{natbib}

\usepackage[utf8]{inputenc} %for utf8 input
\usepackage{amssymb} %for shift symbol
\usepackage{amsmath}
\usepackage{listings} %for code
\usepackage{mathpartir} %for typing rules
\usepackage{microtype} %better micro typing
\usepackage{stmaryrd} %for llbracket
\usepackage{mathabx} % for boxes
\usepackage{graphicx} %to include png images
\usepackage{xcolor} %for colors
\usepackage{url}
\usepackage{enumitem}
\usepackage{array} %for stupid tables

%----------------------------------------------------

\usepackage{prettyref}
\newcommand{\pref}[1]{\prettyref{#1}}
\newcommand{\Pref}[1]{\prettyref{#1} \vpageref[]{#1}}
\newcommand{\ppref}[1]{\vpageref[]{#1}}
\newrefformat{fig}{Figure~\ref{#1}}
\newrefformat{app}{Appendix~\ref{#1}}
\newrefformat{tab}{Table~\ref{#1}}
\newrefformat{cha}{Challenge~\ref{#1}}
\newrefformat{compiler}{Point~\ref{#1} of \pref{thm:compiler}}

%----------------------------------------------------

\usepackage{amsthm}
\newtheorem{lemma}{Lemma}
\newtheorem{corollary}{Corollary}
\newtheorem{theorem}{Theorem}
\newtheorem{proposition}{Proposition}

%----------------------------------------------------

\lstset{
   numbers=none, %use numbers=left
   numberstyle=\tiny,
   stepnumber=1,
   numbersep=5pt,
   basicstyle=\tt\small,
   escapechar=\#,
   mathescape,
%   language=CaML,
   columns=flexible,
%   basewidth=0.455em,
   xleftmargin=\leftmargini,
%   xrightmargin=\leftmargini,
%   morekeywords={mod,div,matchD},
   deletekeywords={as},
   literate={<-}{$\;\leftarrow\;\;\,$}{3} {->}{$\;\rightarrow\;\;\,$}{3},
   firstnumber=auto % use name=NAME to identify split listings
%
% Placement: belowskip,aboveskip,lineskip,boxpos=l|c|r
%
% wonna figure-style listings? Use: caption={Useless code,label=useless
%
% keywordstyle=\color{black}\bfseries\underbar
% morekeywords={one,two,three,four}
}
% \lstinline!! 
% use basicstyle=\small ?

%How many languages are there in CompCert?
\newcommand{\howmanylanguages}[0]{11}

%How many passes are there in CompCert?
\newcommand{\howmanypasses}[0]{20}

\newcommand{\lines}[1]{\textsl{{\scriptsize #1}}}

\newcommand{\valid}[4]{\ensuremath{\mathit{valid}(#1,#2,#3,#4)}}
\newcommand{\safePre}[4]{\ensuremath{\mathit{safe}(#1,#2,#3,#4)}}
\newcommand{\safePost}[3]{\ensuremath{\mathit{safeK}(#1,#2,#3)}}

\newcommand{\refines}[0]{\ensuremath{\prec}}

\newcommand{\qrefines}[0]{\ensuremath{\mathop{\prec_Q}}}

\newcommand{\pruned}[1]{\ensuremath{\overline{ #1} }}

\newcommand{\sem}[1]{\ensuremath{\llbracket #1 \rrbracket}}

\newcommand{\progs}[0]{\ensuremath{\mathcal{P}}}

\newcommand{\id}[0]{\ensuremath{\mathit{x}}}

\newcommand{\xevent}[0]{\ensuremath{\nu}}
\newcommand{\cevent}[0]{\ensuremath{\mu}}
\newcommand{\event}[0]{\ensuremath{\iota}}

\newcommand{\evalue}[0]{\ensuremath{v}}

\newcommand{\intval}[1]{\ensuremath{\mathsf{int}(#1)}}

\newcommand{\floatval}[1]{\ensuremath{\mathsf{float}(#1)}}

\newcommand{\call}[1]{\ensuremath{\mathsf{call}(#1)}}
\newcommand{\return}[1]{\ensuremath{\mathsf{ret}(#1)}}
\newcommand{\malloc}[1]{\ensuremath{\mathsf{malloc}(#1)}}
\newcommand{\free}[1]{\ensuremath{\mathsf{free}(#1)}}

\newcommand{\trace}[0]{\ensuremath{t}}

\newcommand{\Trace}[0]{\ensuremath{T}}

\newcommand{\conv}[2]{\ensuremath{\mathsf{conv}(#1,#2)}}

\newcommand{\divt}[1]{\ensuremath{\mathsf{div}(#1)}}

\newcommand{\fail}[1]{\ensuremath{\mathsf{fail}(#1)}}

\newcommand{\behav}[0]{\ensuremath{B}}

\newcommand{\getchar}[0]{\ensuremath{\mathsf{getchar}}}

\newcommand{\Z}[0]{\ensuremath{\mathbb{Z}}}
\newcommand{\N}[0]{\ensuremath{\mathbb{N}}}
\newcommand{\Qplus}[0]{\ensuremath{\mathbb{Q^+}}}
\newcommand{\Q}[0]{\ensuremath{\mathbb{Q}}}

\newcommand{\weight}[2]{\ensuremath{W_{#2}(#1)}}
\newcommand{\tval}[2]{\ensuremath{V_{#2}(#1)}}

\newcommand{\prefs}[1]{\ensuremath{\mathit{prefs}(#1)}}

\newcommand{\type}[1]{\ensuremath{\mathsf{#1}}}

\newcommand{\code}[1]{\ensuremath{\mathsf{#1}}}

\newcommand{\env}[0]{\ensuremath{\mathit{\theta}}}

\newcommand{\mem}[0]{\ensuremath{\mathit{H}}}

\newcommand{\Mem}[0]{\ensuremath{\mathit{Mem}}}

\newcommand{\Var}[0]{\ensuremath{\mathit{Var}}}

\newcommand{\Val}[0]{\ensuremath{\mathit{Val}}}

\newcommand{\Aux}[0]{\ensuremath{\mathit{Aux}}}

\newcommand{\Input}[0]{\ensuremath{\mathit{StdIn}}}

\newcommand{\Assn}[0]{\ensuremath{\mathit{Assn}}}
\newcommand{\State}[0]{\ensuremath{\mathit{State}}}
\newcommand{\state}[0]{\ensuremath{\sigma}}

\newcommand{\Prop}[0]{\ensuremath{\mathit{Prop}}}

\newcommand{\ret}[0]{\ensuremath{\mathit{ret}}}

\newcommand{\htriple}[3]{\ensuremath{\{#1\}\, #2\, \{#3\}}}

\newcommand{\smallstep}[1]{\ensuremath{\to_{#1}}}

\newcommand{\conform}[2]{\ensuremath{\mathit{agree}(#1,#2)}}

\newcommand{\dom}[1]{\ensuremath{ \text{dom}(#1)}}
\newcommand{\img}[1]{\ensuremath{ \text{img}(#1)}}

\newcommand{\pimp}[2]{\ensuremath{#1 \Rightarrow #2}}

\newcommand{\Assert}[0]{\ensuremath{\mathit{Assert}}}

\newcommand{\VAssert}[0]{\ensuremath{\mathit{VAssert}}}

\newcommand{\eenv}[0]{\ensuremath{\Sigma}}

\newcommand{\genv}[0]{\ensuremath{\Delta}}

\newcommand{\fenv}[0]{\ensuremath{\Sigma}}

\newcommand{\Fundef}[0]{\ensuremath{\mathit{Fundef}}}

\newcommand{\funs}[0]{\ensuremath{\mathit{FID}}}

\newcommand{\vars}[0]{\ensuremath{\mathit{VID}}}

\newcommand{\statements}[0]{\ensuremath{\mathcal{S}}}

\newcommand{\traces}[0]{\ensuremath{\mathcal{T}}}

\newcommand{\behavs}[0]{\ensuremath{\mathcal{B}}}

\newcommand{\events}[0]{\ensuremath{\mathcal{E}}}

\newcommand{\initState}[2]{\ensuremath{\mathit{initSt}(#1,#2)}}

\newcommand{\retState}[1]{\ensuremath{\mathit{retState}(#1)}}

\newcommand{\Potential}[0]{\ensuremath{\mathit{Pot}}}

\newcommand{\evalE}[2]{\ensuremath{\llbracket #1 \rrbracket_{#2}}}

\newcommand{\Kseq}[2]{\ensuremath{\mathsf{Kseq}\, #1 \, #2}}

\newcommand{\Kloop}[2]{\ensuremath{\mathsf{Kloop}\, #1 \, #2}}

\newcommand{\Kstop}[0]{\ensuremath{\mathsf{Kstop}}}

\newcommand{\Kcall}[3]{\ensuremath{\mathsf{Kcall}\, #1 \, #2 \, #3}}

\newcommand{\cont}[0]{\ensuremath{K}}

\newcommand{\cost}[0]{\ensuremath{c}}

\newcommand{\used}[0]{\ensuremath{\sqbullet}}

\newcommand{\init}[0]{\ensuremath{\square}}

\newcommand{\FV}[1]{\ensuremath{\text{FV}(#1)}}

\newcommand{\loc}[0]{\ensuremath{\text{Loc}}}

\newcommand{\inter}[2]{\ensuremath{[#1,#2]}}

\newcommand{\ind}[0]{\ensuremath{\mathcal{I}}}

\newcommand{\states}[0]{\ensuremath{\mathcal{H}}}

\newcommand{\shift}[0]{{\lhd}}

%-------------------------------------------------------
%Type rules (mathpatir)

\newcommand{\Rule}[4][]{\ensuremath{\inferrule*[right={\!(#2)},#1]{#3}{#4}}}
\newcommand{\RuleToplabel}[4][]{\ensuremath{\inferrule[(#2)]{#3}{#4}}}
\newcommand{\RuleNolabel}[3][]{\ensuremath{\inferrule*[#1]{#2}{#3}}}

%-------------------------------------------------------


%%% Local Variables: 
%%% mode: latex
%%% TeX-master: "main"
%%% End: 

\begin{document}

%SHORT
% \lstset{basicstyle=\tt\scriptsize}
% \abovedisplayskip=3pt
% \belowdisplayskip=3pt
% \abovedisplayshortskip=3pt
% \belowdisplayshortskip=3pt
%SHORT

\conferenceinfo{POPL} {January XX-XX, 2014, CITY, STATE, India.}
\copyrightyear{2014}
\copyrightdata{XXX-X-XXXX-XXXX-X/XX/XX}

%----------------------------------------------------

\newcommand{\jan}[1]{{\color{orange}[[{#1}]]}}


\titlebanner{\iffull{Extended Version}{}}%{Draft -- Do not distribute}        % These are ignored unless
\preprintfooter{\iffull{Extended Version\hspace{-.8cm}}{Draft\hspace{.5cm}}}%{Draft}   % 'preprint' option specified.

\title{Compositional Certified Resource Bounds\ifshort{\vspace{-.15cm}}{}}
% Certified Resource Bounds for C
% Compositional Certified Resource Bounds for C
% Compositional Quantitative Resource Analysis of C Programs
% Compositional Resource-Bound Certification for C Programs

\authorinfo
{Quentin Carbonneaux \and Jan Hoffmann \and Zhong Shao}
{Yale University \vspace{-1cm}}


\maketitle


\begin{abstract}
This paper presents a new approach to quantitative
analysis for C programs.  The technique described uses
ideas from amortized analysis and abstract interpretation
to derive tight linear worst-case bounds that come with a
proof certificate.  Contrary to most available tools for
C, ours has a compositional design that provides the
programmer with bounds expressed in terms of meaningful
variables.  The compositionality allows to handle uniformly
nested sequenced loops and recursive sequenced function
calls.  The automatic analysis is implemented
as a tool that leverages state-of-the-art SMT solvers
LP solvers to automatically infer resource bounds on
a wide range of C programs.  It is the first automatic
amortized analysis that can derive bounds that depend
on negative integers and differences between numbers.
To prove its soundness and give a concrete shape to proof
certificates, we developed a quantitative Hoare logic
that we proved sound using the Coq proof assistant.  This
logic allows manual proofs of arbitrarily complex resource
bounds on C programs and interaction with the automation.
To prove the scalability and precision of this new tool,
we run it on multiple code snippets from the literature,
reproduce results from previous work, and analyze a
consequent part of a modern benchmark suite.

%   The goal of quantitative resource analysis is to provide developers
%   with quantitative information about the runtime behavior of software
%   at development time.  Recent years have seen tremendous progress in
%   automatically deriving worst-case resource bounds, yet many
%   challenges in specifying, interactively deriving, formally
%   certifying, and composing resource bounds remain unsolved.
%
%   This paper describes a novel quantitative resource analysis
%   framework that tackles these challenges for C programs.  The
%   analysis framework consists of three parts: First, a parametric cost
%   semantics formalizes the resource consumption of CompCert Clight
%   programs.  Second, a quantitative Hoare logic enables users to
%   interactively develop resource bounds in the Coq Proof Assistant.
%   The logic is proved sound with respect to the cost semantics, and
%   shallow embedding enables a derived bound to be any function that is
%   definable in Coq.  Third, an automatic amortized resource analysis
%   for Clight computes derivations in the quantitative Hoare logic.
%   It is the first automatic amortized analysis that can derive bounds
%   that depend on negative integers or differences between numbers,
%   which is crucial to handle typical systems code.  Both, the
%   quantitative logic and the automatic amortized analysis are
%   naturally compositional and can be combined to semi-automatically
%   derive global resource bounds.
%   %
%   An experimental evaluation demonstrates the practicality of the
%   analysis framework.  The expressivity of the logic is shown by
%   manually deriving customized bounds that are tailored to specific
%   algorithms.  A comparison of the automatic amortized analysis with
%   other automatic tools on 30 challenging loop and recursion patterns
%   from the literature and open-source software shows that the bounds
%   derived by the automatic amortized analysis are often more precise.
\end{abstract}

% \category{D.2.4}{Software Engineering}
% {Software/Program Verification}
% \category{F.3.1}{Logics and Meanings of Programs}
% {Specifying and Verifying and Reasoning about Programs}

% \terms Verification, Reliability


% \keywords Formal Verification, Compiler Construction, Program Logics,
% Stack-Space Bounds, Quantitative Verification


\sectskip
\section{Introduction}
\label{sec:intro}
\aftersectskip

In software engineering and software verification,
we often would like to have static information
about the quantitative behavior of programs.
For example, stack and heap-space bounds
are important to ensure the reliability of
safety-critical systems~\cite{veristack14,Regehr05}.
Static energy usage information is critical
for autonomous systems and has applications in
cloud computing~\cite{CohenZSL12,CarrollH10}.
Worst-case time bounds can help to create
constant-time implementations that prevent
side-channel attacks~\cite{KasperS09,BartheBCLP14}.
Loop and recursion-depth bounds are used to
ensure the accuracy of programs that are executed
on unreliable hardware~\cite{CarbinMR13} and
complexity bounds are needed to verify cryptographic
protocols~\cite{BartheGB09}.  In general, quantitative
performance information at design time can provide useful
feedback for developers.




% NEW NEW NEW NEW NEW NEW NEW NEW NEW NEW NEW NEW NEW

Available techniques for deriving worst-case resource bounds
split in two categories.  Techniques
in the first category derive impressive bounds for
imperative programs, but they are not fully compositional.
This is problematic if one needs to derive global whole-program bounds.
In the second category, techniques derive tight whole-program
bounds on functional programs.
They are highly compositional, scale for large
programs, and work directly on the syntax.  However, despite all
these theoretical and practical advantages, it has been
a long-time open problem to make them work in the
presence of negative integers, mutation, and non-linear
control flow. As a result, they are not suitable for
typical imperative code.
%
\jan{I changed the order here a bit. Good style: First sentence is a
  summary of the following paragraph.}

Notable tools in the first category include
SPEED~\cite{GulwaniMC09}, KoAT~\cite{BrockschmidtEFFG14},
PUBS~\cite{AlbertAGPZ12}, Rank~\cite{AliasDFG10},
and LOOPUS~\cite{SinnZV14}.
They lack compositionality in at least three ways.
First, they all base their analysis on some form of
\emph{ranking-function}.
Since this
is often a local analysis, the loop bounds are
arithmetic expressions that depend on the value of
variables  just before the loop.
This makes it hard to give a
resource bound on sequenced loops or function calls in
terms of the input parameters of a procedure.
Second, while all popular imperative programming languages
provide a function or procedure abstraction, available
tools are not able to abstract their resource behavior
and have to inline the procedure body to perform their
analysis.  Third, when a tool fails to find a resource
bound for a program, the program has to be rewritten.
There is no possibility for sound user interaction with the
tool. For example, there is no concept of manual proofs for
resource bounds and no framework that could support composition
of manually derived bounds with automatically inferred bounds.
One reason for this lack of interaction is that available tools
work on an abstract version of the source program. This
abstracted program has more behaviors than the
original one and is usually expressed in a language that
does not feature memory operations.
%
\jan{@Quentin: What do you mean by ``more behaviors''?}

Tools in the second category are based on type systems for functional
programs and on the potential method of amortized
analysis~\cite{Jost03,HoffmannAH12}.  The first challenge of
imperative programs is already the absence of a fine-grained type system to hook
the analysis on.  Another issue is that the techniques associate
\emph{potential} directly to program variables.  In
imperative programs, this would fail for loops as simple as
\lstinline{for(i=x;i<y;i++)} where $y - i$ decreases, but not $|i|$.
Mutation is yet another challenge.  While functional programs relate
multiple fixed inputs (the typing context) with one output (the
expression typed), imperative programs can change the value of
variables in scope, so the potential of these variables has to be
updated after mutations or, unsound bounds can be derived.
%
\jan{@Quentin: There is a mismatch between the previous two paragraphs.
  In the first one we talk about things that category one techniques
  can't do. In the second one we talk about ``challenges'' for the
  category two. We instead should just say what the state-of-the-art
  can't do. We also have to mention Hofmann/Jost ESOP'06 and Atkey ESOP'10.}

This paper presents a new framework for automatically
deriving resource bounds on C programs.  This new
approach is an attempt to unify the two previous
categories: it solves the compositionality issues of the
imperative world by adapting amortized-analysis based
techniques from the functional world.

Our automated analysis is able to infer linear
resource bounds on C programs with mutually-recursive functions and integer loops.
To our knowledge this is the first technique based on
amortized analysis that is able to derive bounds that depend on negative numbers
and differences of variables.  It is also the first
resource analysis technique for C that deals naturally with
recursive functions, resources that can become available during execution (e.g., when freeing
memory), and sequenced loops.  Compared to more classical
apporaches based on ranking functions, our tool inherits
the benefits of amortized reasoning.  Using only one
simple mechanism, it handles:

\begin{itemize}
\item interactions between \emph{sequential loops or
  function calls} through size changes of variables,
\itemskipIn
\item \emph{nested loops} that influence each other
  with the same set of modified variables,
\itemskipIn
\item and \emph{amortized bounds} as found, for example, in
  the Knuth-Morris-Pratt algorithm for string search.
\end{itemize}
\jan{We can say that our technique is the only one that reduces the problem to
  LP solving. I used that as a selling point in the DARPA-STAC proposal.}

The main innovations that make amortized analysis work
on imperative languages is to base the analysis on a
Hoare-like logic and track multivariate quantities instead
of program variables directly.  This leads to precise bounds
expressed as functions of sizes $|[x, y]| = \max(0, y-x)$ of
intervals.

\jan{Here we need a description of the implementation and the
  experiments.  Message: You can immediately see the great benefits of
  the unification of the two categories in practice. It scales to
  larger programs than existing tools (for global bounds) and is as
  good on local bounds (for tricky loop and recursion patterns).}

% NEW NEW NEW NEW NEW NEW NEW NEW NEW NEW NEW NEW NEW


Give a description of the sections here.                    XXX

\jan{We also mention before that existing imperative tools don't allow
  for user interaction. We should pick this up again and say that we
  can do it. But then we have to mention the PLDI paper. My suggestion:
  \begin{itemize}
  \item mention PLDI'14 in the paragraph about the second category and
    say it has great flexibility for manual bound derivation but
    almost no support for automation
  \item just add another bullet to say that we have the same
    flexibility for manual bound derivation as in PLDI'14 and that
    manual and automatic bounds can be composed
  \end{itemize}
}


% In software engineering and software verification, we often would like
% to have static information about the quantitative behavior of
% programs.  For example, stack and heap-space bounds are important to
% ensure the reliability of safety-critical systems~\cite{veristack14,Regehr05}.
% % Clock-cycle bounds are needed to guarantee the safety of real-time
% % systems [XXX wilhelm].
% Static energy usage information is critical for autonomous systems and
% has applications in cloud computing~\cite{CohenZSL12,CarrollH10}.
% Worst-case time bounds can help to create constant-time
% implementations that prevent side-channel
% attacks~\cite{KasperS09,BartheBCLP14}.  Loop and recursion-depth
% bounds are used to ensure the accuracy of programs that are executed
% on unreliable hardware~\cite{CarbinMR13} and complexity bounds are
% needed to verify cryptographic protocols~\cite{BartheGB09}.
% % XXX Mention differential privacy here?
% %
% In general, quantitative performance information at design time can
% provide useful feedback for developers.
%
% Static analysis of quantitative properties of imperative programs is
% an active research area and recent years have seen many innovations.
% Notable tools that have been developed include
% SPEED~\cite{GulwaniMC09}, KoAT~\cite{BrockschmidtEFFG14},
% PUBS~\cite{AlbertAGPZ12}, Rank~\cite{AliasDFG10}, and LOOPUS~\cite{SinnZV14}.
% %
% While the these tools can derive impressive results for realistic
% software, there still exist shortcomings that hamper the application
% of existing techniques in practice:
% \itemskip
% \begin{itemize}
% \item Analysis tools are black boxes that\iffull{either}{} deliver a result or fail
%   without enabling \emph{user interaction} to manually or
%   semi-automatically derive bounds for challenging parts of the
%   program.
% \itemskipIn
% \item The computed bounds are often \emph{non-compositional} local
%   bounds (for a single (nested) loop) that are difficult to combine to
%   global whole program bounds.
% \itemskipIn
% \item Existing techniques often rely on complex external tools such as
%   abstract interpretation--based invariant
%   generation~\cite{GulwaniMC09} or translation of the program into a
%   term-rewriting system~\cite{BrockschmidtEFFG14, SinnZV14} without
%   providing \emph{verifiable certificates} for the correctness of the
%   derived bound.
% \end{itemize}
% \itemskip
% %
% While there has been much progress in static quantitative analysis,
% Knuth correctly points out in a recent interview~\cite{KnuthInter}
% that the state-of-the-art in formal quantitative methods still falls
% short in comparison with semantic techniques.
% %
% \itemskip
% \begin{quote}
%   % Consider, for example, a computer program that sorts a list of
%   % numbers into order.
%   $[\cdots]$ Thanks to the work of Floyd, Hoare, and others,
%   we have formal definitions of semantics, and tools by which we can
%   verify that sorting is indeed always achieved. My job is to go
%   beyond correctness, to an analysis of such things as the program's
%   running time $[\cdots]$. I'm 100\% sure that my recurrence correctly
%   describes the program's performance, and all of my colleagues agree
%   with me that the recurrence is "obviously" valid. Yet I have no
%   formal tools by which I can prove that my recurrence is right. I
%   don't really understand my reasoning processes at all!
%   \vspace{-4ex}
%   \begin{flushright}
%     -- Donald E. Knuth, 2014
%   \end{flushright}
% \end{quote}
% \itemskip
% %
% In this work, we develop a resource analysis framework for C programs
% that is based on a solid semantic foundation.  The choice of C is
% primarily motivated by \iffull{our}{the} ongoing work on the formally verified
% hypervisor kernel CertiKOS~\cite{GuVFSC11}.  CertiKOS is mainly
% developed in C and is supposed to provide verified guaranties on
% timing and memory usage.  Moreover, C is a natural choice because it
% is still the most widely used language for system development; in
% particular in safety-critical embedded and real-time systems where
% resource bound analyses are often required by regulatory
% authorities~\cite{May2013}.  Our \emph{contributions} are as follows:
% %
% \itemskip
% \begin{enumerate}
% \item We define an operational cost semantics for CompCert Clight that
%   defines the resource consumption of terminating and diverging
%   executions.  The cost is parametric in a user-definable cost metric
%   and can be negative if resources are released during an execution.
%
% \item We develop a quantitative Hoare logic for interactively deriving
%   resource bounds for Clight programs.  The logic is implemented and
%   proved sound in the Coq Proof Assistant with respect to the cost
%   semantics of Clight.
% \itemskipIn
% \item We describe an automatic static analysis that computes bounds
%   together with derivations in the quantitative Hoare logic.  The
%   analysis is inspired by type-based amortized resource analysis for
%   functional programs~\cite{Jost03,HoffmannAH11}.  It computes
%   derivations by generating simple linear constraints that can be
%   solved by an off-the-shelf LP solver, and does not require any
%   fixpoint computations to obtain loop invariants.
% \itemskipIn
% \item We show with a publicly available prototype implementation, and
%   an experimental evaluation, that our novel amortized resource
%   analysis works precisely and efficiently for challenging loop and
%   recursion patterns, and that the derived constant factors in the
%   bounds are close to or identical with the optimal ones.
% % \item We prove the completeness of the program logic
% %   with respect to the operational cost semantics.
% \end{enumerate}
% \itemskip
% %
% To the best of our knowledge, this article presents the first resource
% analysis framework for C that makes it possible to combine non-trivial
% automatically derived bounds with interactively derived bounds in a
% proof system that produces verifiable certificates for the bounds.
% Our approach complements existing work since it provides a semantic
% foundation for the computation of bounds while still supporting
% automation that sometimes goes beyond the capabilities of existing
% techniques.  Furthermore, the automatic amortized analysis and the
% quantitative logic are naturally compositional and describe the
% resource behavior of code fragments without referring to the source
% code.  As a result, we directly derive global bounds that are
% functions of the input parameters of the program.  Another unique
% feature of our system is, that it can handle resources like memory
% that may become available during execution.
%
% Our starting point is a recent work~\cite{veristack14} \iffull{in
%   which we have}{that has} formally verified bounds on the stack usage
% of C programs.  A key part of this stack-bound verification is based
% on a quantitative Hoare logic for deriving abstract \emph{stack
%   bounds} for CompCert Clight programs\ifshort{.}{ that depend on the
%   depth of (recursive) function calls.}  % For
% % non-recursive programs it is possible to compute derivations for such
% % stack bounds in the quantitative Hoare logic automatically.
% % By compiling these bounds together with the C program, we have derived
% % verified stack bounds for x86 assembly code.
% %
% Our \emph{first contribution} is a generalization of the quantitative
% Hoare logic for CompCert Clight that enables us to derive resource
% bounds that are parametric in the resource of interest.  % The
% % quantitative logic is implemented in Coq and proved sound with respect
% % to a small-step operational cost semantics for Clight.
% % that counts a
% % constant resource consumption for each atomic execution step.  The
% % resource consumption at each step is parametric in a user-definable cost
% % metric and can be negative to express the release of resources (as
% % needed, e.g., to reason about memory usage).  Furthermore, a specific
% % \emph{tick} function can be inserted in the source code to further
% % customize the cost model.
% In the implementation of the logic, we use
% a shallow embedding in Coq which makes the logic very flexible.  There
% is practically no limitation on the format of the derived resource
% bounds since they can be any function that is definable in Coq.  The
% derived bounds can also be parametric in a set of cost metrics or
% specifically apply to one fixed cost metric.
%
% The precision and expressivity of the quantitative Hoare logic provide
% an ample foundation for reasoning about resource consumption of Clight
% programs.  % The logic is naturally compositional and the derived
% % \emph{quantitative Hoare triples} can be used to specify the resource
% % consumption of library functions or to describe resource contracts
% % between different functions.
% However, reasoning about quantitative
% properties can often be more tedious than reasoning about intentional
% properties.  Consequently, automation is inevitable to derive resource
% bounds for large code bases like CertiKOS.  Such an automation was
% easily achievable in the case of constant stack bounds for a code base
% that does not contain recursive functions~\cite{veristack14}.  In
% general however, resource bounds have to be parametric in the
% arguments of a function and depend on the number of loop iterations
% and (recursive) function calls that are performed by the function.
%
% For functional programs, there exist resource analysis systems---based
% on type-based amortized resource analysis~\cite{Jost03,
%   HoffmannAH12}---that can automatically derive complex polynomial
% bounds with tight constant factors.  This type-based amortized
% resource analysis has been an inspiration in the design of the
% quantitative Hoare logic and is therefore a natural candidate to
% automate the reasoning.  Type-based amortized resource analysis works
% well for functional programs with pattern matching but it has been a
% long-time open problem to apply it to C-like programs with control
% flow that depends on integer arithmetic, negative numbers, and non-linear
% control flow.
% % example, it is a
% % long-time open problem how to use type-based amortized analysis to
% % derive bounds for functions whose resource consumption depends on
% % (possibly negative) integers and non-sequential control flow as
% % introduced by \code{break} and \code{return} statements.
% % Unfortunately, these are the functions \iffull{with non-constant resource
% % consumption}{} that are most common in CertiKOS and other system
% % software.
%
% Our \emph{main contribution} is an automatic amortized resource
% analysis for CompCert Clight that computes derivations in the
% quantitative Hoare logic.  It is the first automatic amortized
% resource analysis that can derive bounds that depend on negative
% integers. It also handles programs with mutually recursive functions
% as well as \code{break} and \code{return} statements.  The
% potential-based approach of amortized analysis provides a single
% mechanism for analyzing programs that need special treatment in other
% techniques; including
% \itemskip
% \begin{itemize}
% \item interaction between \emph{sequential loops or function calls} through
%   size changes of variables,
% \itemskipIn
% \item \emph{nested loops} that influence each others number of iterations,
% \itemskipIn
% \item and \emph{tricky iteration patterns} as found for instance in
%   the Knuth-Morris-Pratt algorithm for string search.
% \end{itemize}
% \itemskip
% %
% The main innovation that makes the analysis practical for C is the use
% of interval sizes in potential functions instead of the sizes of
% variables.  This leads to precise bounds of programs whose resource
% consumption can be described as a function of sizes $|[x,y]| =
% \max(0,y-x)$ of intervals of integer variables.  Such bounds arise for
% instance from standard \emph{for loops} \code{for (i = x; i+K \leq y;
%   i=i+K)} where the step-size $K>0$ is a constant.
% %
%
% Despite the apparent simplicity of the new analysis system, it is able
% to reproduce results from the literature (e.g,
% SPEED~\cite{GulwaniMC09}) that have been obtained using sophisticated
% abstract interpretation--based methods such as disjunctive invariant
% generation~\cite{GulwaniMC09} and symbolic backward
% execution~\cite{GulwaniZ10}.  In contrast with abstract
% interpretation--based methods, our technique does not require any
% fixpoint computations to obtain loop invariants.  The mechanism we
% designed is able to leverage local assertions such as $x < y$ that we
% collect along the branching points of the program to obtain global
% resource invariants.  We achieve this by generating a linear
% constraint system that reflects the resource cost and size changes of
% integer variables in the program; a solution of the linear program
% immediately yields a resource-usage bound for the C
% program.  % We can naturally, handle
% % advanced control flow such as break, return, iteration over negative
% % integer variables, and mutually-recursive functions.
%
% Following the development steps of automatic amortized analysis for
% functional programs~\cite{Jost03,HoffmannH10}, we deliberately
% restrict ourselves to linear bounds in the automatic analysis (there
% are no restrictions for the manually-derived bounds in the logic).
% More specifically, bounds have the form $\sum_{a,b} q_{(a,b)} |[a,b]|$
% where $a$ and $b$ are integer variables or constants.  The reason for
% our focus on linear bounds is mainly clarity of presentation.  We
% already experimented with an extension to multivariate resource
% polynomials~\cite{HoffmannAH11,HoffmannS13} but this would make the inference
% rules considerably more involved and should better be described
% separately.  However, we developed the linear inference system so that
% the extension to polynomial bounds shall work smoothly.
%
% We have evaluated the automatic analysis with system code and examples
% from the literature.  \iffull{\pref{app:cat}}{The extended version
%   of this article~\cite{anon_extended}}{} contains more than $30$
% challenging loop and recursion patterns that we collected from open
% source software and the literature.  Our analysis can find
% asymptotically tight bounds for all but 1 of these patterns, and in
% most cases the derived constant factors are tight.  To compare our
% automatic analyzer with existing techniques, we tested our examples
% with tools such as KoAT~\cite{BrockschmidtEFFG14},
% Rank~\cite{AliasDFG10}, and LOOPUS~\cite{SinnZV14}.  Our experiments
% show that the bounds that we derive are often more precise than those
% derived by existing tools.  Only LOOPUS~\cite{SinnZV14}, which also
% uses amortization techniques, is able to achieve a similar precision.
% %
% Several micro benchmarks demonstrate the practicality and
% expressiveness of the quantitative Hoare logic.  For example, we
% derive a logarithmic bound for a binary search function, a bound that
% depends on the contents of an array to describe the exact cost of a
% function that finds a maximal element in an array, and a linear
% bound that amortizes the cost of $k$ successive increments to a binary
% counter.
%
% % To show the practicality of our approach we developed a case study
% % with different real-time schedulers implemented in C.  To perform a
% % schedulability analysis that takes into account the overhead of the
% % scheduler we derived time bounds for context switching and
% % initialization of the scheduler.  We then computed time bounds for a
% % set of given user task and took account the previously computed
% % scheduler overhead to determine whether the scheduling requirements of
% % all user tasks can be met.
%
% % With the intention, of making the material easily accessible we first
% % informally describe the new automatic amortized analysis
% % (\pref{sec:inform}).  We then formalize the intuition with the
% % operational cost semantics an the inference rules for the amortized
% % analysis (\pref{sec:aa}).  In \pref{sec:logic} we describe the
% % quantitative Hoare logic and its soundness proof that is formalized in
% % Coq.  In \pref{sec:inter} we prove the soundness of the automatic
% % amortized analysis by showing that derivation using the inference
% % rules can be seen as derivations in the quantitative logic.  We also
% % show how to integrate hand-derived bounds with automatically-derived
% % bounds.  \pref{sec:exper} contains the results of the experimental
% % evaluation and \pref{sec:related} describes related research.

\sectskip
\section{Overview and Examples}
\aftersectskip

In this section, we informally introduce the automatic amortized
analysis (AAA) for Clight programs and demonstrate the technique with
example bounds that we derive step-by-step following our analysis
system.

\jan{Watch out with the headings. Either upper case or lower case, but
  don't mix.}

\subsection{Bounding Resource Usage with Potential Functions}

The idea that underlies the design of our framework is amortized
analysis~\cite{Tarjan-amort}.  Assume that a program $S$ executes
on a starting state $\state$ and consumes $n$ resource units of
some user-defined quantity.  We denote that by writing $(S, \state)
\Downarrow_n \state'$ where $\state'$ is the program state after
the execution.  The basic idea of amortized analysis is to define a
\emph{potential function} $\Phi$ that maps program states to non-negative
numbers and to show that $\Phi(\state) \geq n$ if $\state$ is a program
state such that $(S, \state) \Downarrow_n \state'$.  This gives that
$\Phi(\state)$ is a valid resource bound.

To obtain a compositional reasoning we also have to take into account the
state resulting from a program's execution.  We thus use two potential
functions, one that applies before the execution, and one that applies
after.  The two functions must respect the relation $\Phi(\state)
\ge n + \Phi'(\state')$ for all states $\state$ and $\state'$ such
that $(S, \state) \Downarrow_n \state'$.  Intuitively, $\Phi(\state)$
must provide enough \emph{potential} for both, paying for the resource
cost of the computation and paying for the potential $\Phi'(\state')$ on
the resulting state $\state'$. That way, if $(\state, S_1) \Downarrow_n
\state'$ and $(\state', S_2) \Downarrow_m \state''$, we get $\Phi(\state)
\ge n + \Phi'(\state')$ and $\Phi'(\state') \ge m + \Phi''(\state'')$.
This can be composed as $\Phi(\state) \ge (n + m) + \Phi''(\state'')$.
Note that the initial potential function $\Phi$ provides an upper bound
on the resource consumption of the whole program.  What we have observed
is that, if we define $\htriple{\Phi} {S}{\Phi'}$ to mean
$$
\forall \state\, n\, \state'.\, (\state, S) \Downarrow_n \state' \implies \Phi(\state) \ge
n + \Phi'(\state') \; ,
$$
then we get the following familiar looking rule.
$$
\RuleNolabel
{\htriple{\Phi}{S_1}{\Phi'} \\ \htriple{\Phi'}{S_2}{\Phi''}}
{\htriple{\Phi} {S_1; S_2}  {\Phi''}}.
$$
%
Similarly, other language constructs lead to rules for the potential
functions that look very similar to Hoare logic or effect system
rules.  These rules enable interactive reasoning about resource usage
in a flexible and compositional way, which, as a side effect, produces
a certificate for the derived resource bound.

The considerations above already show a departure from classical
techniques for imperative programs.  Indeed, reasoning with two
potential functions is more compositional than using
ranking-functions because it forces to consider the sequencing of program
operations.  In the previous rule, $\Phi$ gives a bound for
$S_1; S_2$ through the intermediate potential $\Phi'$, even though
it was derived on $S_1$ only.

% Potential functions can access the whole program state, so this
% framework can readily be used to derive bounds that depend on the
% memory contents.

\subsection{Automatic Amortized Analysis}

The reasoning presented above is very general and flexible.  But in
practice, to make the reasoning tractable on large code bases, we would
like to automatically infer the potential functions.  The first step
towards this goal is to reduce our search space by fixing the shape
of potential functions $\Phi$.  We settle on a form that allows
\emph{linear programming} (LP) to help inference:
%
\jan{I wouldn't say that LP is the reason we picked this format. We
  rather picked it because it covers the most common cases.}
%
$$
\Phi(\state) = q_0 + \sum_{x,y \in \dom(\state)}
  q_{(x,y)} \cdot |[\state(x),\state(y)]|
$$
where $q_{(x,y)} \in \Qplusz$ and $|[a,b]| = \max(b-a,0)$.  Here, we assume
that the program state is a simple map from program variables to integers
and we treat constants as program variables (that cannot be changed
by the program text).  For instance, $q_{(0,x)}|[0,x]|$ always appear in
the sum.  We then develop an inference rule for each syntactic construct
that derives a triple $\htriple{\Phi}P{\Phi'}$ sound in the sense of
the previous section.

This idea is best explained by example.  If we use the tick metric that
assigns cost $n$ to the function call \code{tick(n)} and cost $0$ to all
other operations then the cost of the following example can be bounded
by $|[x,y]|$.\footnote{Note that there are no restrictions on the signs
of the input variables in the examples.}
\begin{lstlisting}[basicstyle=\tt\small]
while (x<y) { x=x+1; tick(1); } #\hfill \textnormal{(Example 1)}#
\end{lstlisting}
To derive this bound in our automatic amortized analysis, we start with
the initial potential $\Phi_0 = |[x,y]|$ (short for $\Phi_0(\state)
= |[\state(x),\state(y)]|$) which we also use as the loop invariant.
For the loop body we have (like in Hoare logic) to derive a triple
like $\htriple{\Phi_0}{\code{x=x+1;tick(1)}}{\Phi_0}$.  We can only do
so if we utilize the fact that $x<y$ at the beginning of the loop body.
The reasoning then works as follows. We start with the potential $|[x,y]|$
and the fact that $|[x,y]| > 0$ before the assignment.  If we denote the
updated version of $x$ after the assignment by $x'$ then the relation
$|[x,y]| = |[x',y]| + 1$ between the potential before and after the
assignment \code{x=x+1} holds.  This means that we have the potential
$|[x,y]| + 1$ before the statement \code{tick(1)}.  Since \code{tick(1)}
consumes one resource unit, we end up with potential $|[x,y]|$ after
the loop body and have established the loop invariant again.

\jan{I would add another subsection here to explain how we automate
  the analysis with LP solving.  Message: Efficient reduction to LP
  solving and LP solvers are great and super fast. Maybe mention
  'network problems'.}

\subsection{Compositionality in Practice}

With two concrete examples from open-source projects we demonstrate
that the compositionality of our method is indeed needed in the
analysis of common code.

\jan{I don't like to talk about ``the reader''. Often you can leave it
  out. It is fine to address the reader with ``you''. See BUGS in
  Writing.}

The following example is typical in cryptographic primitives
implementations.  These primitives work by block, so when data of
arbitrary length needs to be processed, it is munched in blocks and
the leftovers are stored in a buffer for future use when more data
is available.  This looping pattern is particularly well handled by
our tool because the LP solver is leveraged to generate a tight bound.
The optimization problem it is given encodes that constant bounds are
preferable to linear bounds.  Thus, if $N >= 8$, the bound given by
AAA is $\frac{N}{8} |[0,l]|$, but if $N<8$ it becomes $7\frac{8-N}{8}
+ \frac{N}{8}|[0,l]|$, because the cost for the second loop is better
counted as constant.
%
\jan{N is unbound above. The explanation must be much slower
  here. Explain what the code does and how the ticks account for the
  cost. Then: What is the resource bound?. Then: What do we derive?
  Btw: Why do you use tick(1) and not tick(M)?}
%
\begin{lstlisting}
for (; l >= 8; l -= 8)
    // process one block
    tick(N);
for (; l > 0; l--)
    // save leftovers
    tick(1);
\end{lstlisting}
%
It is worth taking some time to explain the constant contribution
to the resource bound when $N<8$.  Remark first that the cost of
the second loop is $|[0,l]|$, and after the first one, we still have
$\frac{N}{8}|[0,l]|$ potential available from the invariant.  So we have
to raise the potential of $|[0,l]|$ from $\frac{N}{8}$ to 1, that is,
we must pay $\frac{8-N}{8}|[0,l]|$.  But since we got out of the first
loop, we know that $l<8$, so it is sound to only pay $7\frac{8-N}{8}$
potential units instead.  This explains the somewhat intriguing value
of the constant term in the bound.  All this reasoning is done within
our tool automatically and simply relies on general rules presented
in further sections.

This level of precision and compositionality is only achieved by
our work, no other available tool will attain the above tight bounds
on this common kind of looping code.  Thanks to the modularity of
amortized analysis the intricate interactions between the bounds of
the two loops can be modeled as one linear program, and because LP
solvers can optimize the quality of the final bound we can keep
it close to the actual resource consumption.

A clean theoretical framework for compositional bounds sometimes
even allow to get better asymptotic bounds.  During our experiments
we found an example of this situation in some code from the cBench
benchmark suite.
%
\begin{lstlisting}
for (;;) {
    do { l++; tick(1); } while (l < h && nondet());
    do { h--; tick(1); } while (h > l && nondet());
    if (h <= l) break;
    tick(1); /* swap two elements */ }
\end{lstlisting}
%
The above snippet is the inner loop of a Quicksort implementation, more
precisely, it is the partitioning part of the algorithm.  This partition
loop has linear complexity, and feeding into our analysis gives the
correct and tight worst-case bound $2+3|[l,h]|$.  However, all available
tools for C will derive a quadratic bound instead.  With the amortized
analysis approach in mind, it is clear that the three loops get their
potential from the same interval $[l,h]$.  But on the other hand, current
techniques will try to find one linear ranking-function for each loop
and combine them multiplicatively or additively, resulting in a loose
quadratic bound.
%
\jan{Did you test the last example with other tools? What are the
  bounds they get? State the bounds here.}

\sectskip
\section{Overview and Examples}
\label{sec:inform}
\aftersectskip

In this section, we informally introduce the quantitative program
logic and the automatic amortized analysis for Clight programs.
% The purpose of this section is to build some intuition that should
% make the following sections easy to read.

\subsection{Quantitative Hoare Logic}

The idea that drives the design of our framework is amortized
analysis~\cite{Tarjan-amort}.  Assume that a program $S$ executes on a
starting state $\state$ and consumes $n$ resource units of some
user-defined quantity.  We denote that by writing $(S, \state)
\Downarrow_n \state'$ where $\state'$ is the program state after the
execution.  The basic idea of amortized analysis is to define a
\emph{potential function} $\Phi$ that maps program states to non-negative
numbers and to show that $\Phi(\state) \geq n$ if $\state$ is a
program state such that $(S, \state) \Downarrow_n \state'$ to prove
a bound on the resource usage.

To obtain a compositional reasnoning we also have to take into account
the state resulting from a program's execution.  We thus use two
potential functions, one that applies before the execution, and one
that applies after.  The two functions must respect the relation
$\Phi(\state) \ge n + \Phi'(\state')$ for all states $\state$ and
$\state'$ such that $(S, \state) \Downarrow_n \state'$.  Intuitively,
$\Phi(\state)$ must provide enough \emph{potential} for both, paying
for the resource cost of the computation and paying for the potential
$\Phi'(\state')$ on the resulting state $\state'$. That way, if
$(\state, S_1) \Downarrow_n \state'$ and $(\state', S_2) \Downarrow_m
\state''$, we get $\Phi(\state) \ge n + \Phi'(\state')$ and
$\Phi'(\state') \ge m + \Phi''(\state'')$.  This can be composed as
$\Phi(\state) \ge (n + m) + \Phi''(\state'')$.  Note that the initial
potential function $\Phi$ provides an upper bound on the resource
consumption of the whole program.  What we have observed is that, if
we define $\htriple{\Phi} {S}{\Phi'}$ to mean
$$
\forall \state\, n\, \state'.\, (\state, S) \Downarrow_n \state' \implies \Phi(\state) \ge
n + \Phi'(\state') \; ,
$$
then we get the following familiar looking rule.
$$
\RuleNolabel
{\htriple{\Phi}{S_1}{\Phi'} \\ \htriple{\Phi'}{S_2}{\Phi''}}
{\htriple{\Phi} {S_1; S_2}  {\Phi''}}
$$
%
Similarly, other language constructs lead to rules for the potential
functions that look very similar to Hoare logic or effect system
rules.  These rules enable interactive reasoning about resource usage
in a flexible and compositional way, which, as a side effect, produces
a certificate for the derived resource bound.

% We have implemented and proved sound such a quantitative Hoare for
% CompCert Clight in the Coq Proof Assistant.  Because we use a shallow
% embedding in Coq, every function that can be expressed in Coq can in
% principle be used as a resource bound.

It is also possible to incorporate boolean conditions into a
bound to express that the bound is only valid for a certain class of
inputs.  To this end, we allow the potential function $\Phi$ in the
pre- and postconditions to take non-negative numbers or $\infty$
(infinity) as values.  Infinity, plays the same role as
$\mathit{false}$ in Hoare logic.  Boolean assertions can be embedded
into potential functions by mapping $\mathit{false}$ to $\infty$ and
$\mathit{true}$ to $0$.
%
\begin{figure}
\center
\begin{minipage}[b]{0.8\linewidth}
\begin{lstlisting}
${\color{blue}%
  \{ \#_1(a) + 2k \}}$
while (k > 0) {
  x=0;
  ${\color{blue}%
    \{ \#_1(a) + 2k \}}$
  while (x < N && a[x] == 1) {
    ${\color{blue}%
      \{ (a[x] = 1) + \#_1(a) + 2k \}}$
    a[x]=0;
    ${\color{blue}%
      \{ \#_1(a) + 2k + 1 \}}$
    tick(1);
    ${\color{blue}%
      \{ \#_1(a) + 2k \}}$
    x++; }
  ${\color{blue}%
    \{ (x \ge N \lor a[x] = 0) + \#_1(a) + 2k \}}$
  if (x < N) {
    ${\color{blue}%
      \{ (a[x] = 0) + \#_1(a) + 1 + 2(k-1) + 1 \}}$
    a[x]=1;
    ${\color{blue}%
      \{ \#_1(a) + 2(k-1) + 1 \}}$
    tick(1); }
  ${\color{blue}%
    \{ \#_1(a) + 2(k-1) \}}$
  k--;
  ${\color{blue}%
    \{ \#_1(a) + 2k \}}$
} ${\color{blue} \{ \#_1(a) \}}$

\end{lstlisting}
\end{minipage}

\caption{Example derivation using amortized reasoning.  We write
  $\#_1(a)$ for $\# \{ i \mid 0 \le i < N \land a[i] = 1 \}$ and use the
  tick metric that assigns cost $n$ to the statement \code{tick(n)} and
  0 to all other operations.
  }
\label{fig:xmplinc}
\end{figure}
%
\iffull{
}{}
We can then use the logic to prove a concrete bound for a given resource
metric, or a bound that is parametric in a set of resource metrics.
For example, we can derive the following quantitative
triple for a binary-search function \code{bsearch} that holds for all
stack metrics $M$.
%
\begin{center}
${ \{ (Z = \log_2(h-\ell)) + Z{\cdot}M_\code{bsearch} \} }$
\code{bsearch(x,l,h)}
${ \{ Z{\cdot}M_\code{bsearch} \} }$
\end{center}
%
A stack metric assigns cost $0$ to all evaluation steps except
function calls.  Before a call $f(x)$, we consume $M_f > 0$ resources
and after the call we return $M_f$ resources that can be used in
subsequent function calls.  So $M_f$ corresponds to the stack-frame
size of the function $f$.  In the pre- and postcondition, the logical
variable $Z$ is used to relate the size of the input with the
potential that is left after the function call.  The logical part $Z =
\log_2(h-\ell)$ of the precondition is $0$ if $Z =
\log_2(\state_0(h)-\state_0(\ell))$ in the initial state $\state_0$ and
$\infty$ otherwise.

For simplicity, assume for the remainder of this section that we are
interested in bounding the number of \emph{ticks} in a program.  That
is, we use the \emph{tick metric} that assigns cost $n$ to the function call
\code{tick(n)} and cost $0$ to all other operations.  The argument $n$
is an integer and a negative number means that $-n$ resources are
returned.

We say that the analysis is amortized because the use of a potential
function $\Phi$ enables us to amortize the cost of operations like in
typical textbook examples that use the potential method of Tarjan's
amortized analysis.  As an illustrative example, \pref{fig:xmplinc}
shows a program that repeatedly increments a binary counter
implemented with a boolean-valued ($0$ or $1$) array.  The derived
bound in the precondition states that the number of ticks
executed by the program is bounded by $\#_1(a) + 2k$.  Here, $k$ is
the number of increments to the counter that are performed by the
program and $ \#_1(a) = \# \{ i \mid 0 \le i < N \land a[i] = 1 \}$ is
the number of $1$'s in the array $a$.  We have inserted the statement
\code{tick(1)} into the program whenever we update the array $a$.  A
naive analysis of the algorithm would lead to a quadratic bound since
there can be a linear number of updates in the inner loop.

To derive the bound for the program in~\pref{fig:xmplinc} in the
quantitative Hoare logic, we follow the textbook
reasoning~\cite{Algorithms}.  When we assign \code{a[x]=0} in the
inner loop, we know that $a[x]=1$ before the assignment.  As a result,
we have $\#_1(a) = \#_1(a') + 1$ if $a'$ is the array $a$ after the
assignment.  So we obtain one resource unit in addition to the
invariant (i.e., $\Phi = \#_1(a) + 2k + 1$) that we can use to pay for the
cost of the statement \code{tick(1)}.  To pay for the tick statement
in the conditional, we use the fact that $k>0$ to write $2k$ as $1 +
2(k-1) +1$.  Dual to the first assignment, we have $\#_1(a) +1 =
\#_1(a')$ if $a$ is the array before and $a'$ is the array after the
assignment \code{a[x]=1}.  After paying one potential unit for the
statement $\code{tick(1)}$, we are left with the potential $\#_1(a) +
2(k - 1)$.  But since we increment $k$ at the end of the outer loop,
we can establish again our loop invariant $\#_1(a) + 2k$.

See \pref{sec:logic} for more a formal definition of the logic and
additional example derivations.

% For example we have the following judgement (provided that initially
% $\state_0(\code x) \ge \state_0(\code y) \ge 0$).
% $$
% \htriple
% {2{\cdot}x + 10}
% {\code{ x = x -  y}}
% {2{\cdot}x + 2{\cdot}y + 10}
% $$
% This must be understood as a motion of potential from the
% variable~$\code x$ to the variable~$\code y$. If some resource is
% consumed $\code y$ times afterwards, we can then use $\code y$'s
% potential to pay for it. However, the initial potential function
% $\Phi$ only refers to the variable~$\code x$: the potential function
% allows us to pass on the cost of the subsequent resource consumption
% to~$\code x$.  This flexibility combined with intelligent decisions on
% potential motions allows us to design an automatic analysis to
% discover linear invariants for surprisingly convoluted programs.

\begin{figure}[t]
  \centering
\begin{lstlisting}[mathescape]
${\color{blue}%
  \cdot; \, 0 + \frac{T}{K} {\cdot} |[x,y]| + 0 {\cdot} |[y,x]| \vdash }$
while (x+K<=y) {
  ${\color{blue}%
  x + K \le y; \; 0 + \frac{T}{K} {\cdot} |[x,y]| + 0 {\cdot} |[y,x]| \vdash }$
  x=x+K;
  ${\color{blue}%
  \dashv x \leq y; \, T + \frac{T}{K} {\cdot} |[x,y]| + 0 {\cdot} |[y,x]| \vdash }$
  tick(T);
  ${\color{blue}%
  \dashv x \leq y; \, 0 + \frac{T}{K} {\cdot} |[x,y]| + 0 {\cdot} |[y,x]| }$
}
${\color{blue}%
 \dashv x \geq y; 0 + \frac{T}{K} {\cdot} |[x,y]| + 0 {\cdot} |[y,x]|}$
\end{lstlisting}
  \caption{Automatic derivation of a tight bound on the number of
    ticks for a standard \emph{for loop}.  The parameters $K>0$ and
    $T>0$ are not program variables but denote concrete constants in
    the program.  The reasoning works uniformly for all such $K$ and
    $T$.}
  \label{fig:ex1}
\end{figure}


%
\newlength{\progwidth}

\begin{figure*}[t!]
 \setlength{\progwidth}{.22\linewidth}
  \centering
\hspace{-0.4cm}
  \begin{minipage}[b]{.18\linewidth}
    \begin{center}
   \begin{lstlisting}[]
while (n>x) {
  ${\color{blue}%
  n{>}x; \, |[x, n]| {+} |[y, m]| \vdash }$
  if (m>y)
    ${\color{blue}%
    m{>}y; \, |[x, n]| {+} |[y, m]| \vdash }$
    y=y+1;
    ${\color{blue}%
    \dashv \cdot ; \, 1 {+} |[x, n]| {+} |[y, m]| }$
  else
    ${\color{blue}%
    n{>}x; \, |[x, n]| {+} |[y, m]| \vdash }$
    x=x+1;
    ${\color{blue}%
    \dashv \cdot; \, 1 {+} |[x, n]| {+} |[y, m]| }$
  ${\color{blue}%
  \dashv \cdot; \, 1 {+} |[x, n]| {+} |[y, m]| \vdash }$
  tick(1);
  ${\color{blue}%
  \dashv \cdot; \, |[x, n]| {+} |[y, m]| }$
}
   \end{lstlisting}

$|[x, n]| + |[y, m]|$
\\[.4\baselineskip]
      {\bf speed\_1}
    \end{center}
  \end{minipage}
%
\hfill
%
  \begin{minipage}[b]{\progwidth}
    \begin{center}
   \begin{lstlisting}

while (x<n) {
  ${\color{blue}%
  x{<}n; \, |[x, n]| {+} |[z, n]| \vdash }$
  if (z>x)
    ${\color{blue}%
    x{<}n; \, |[x, n]| {+} |[z, n]| \vdash }$
    x=x+1;
    ${\color{blue}%
    \dashv \cdot; \, 1 {+} |[x, n]| {+} |[z, n]| }$
  else
    ${\color{blue}%
    z{\leq}x, x{<}n; \, |[x, n]| {+} |[z, n]| \vdash }$
    z=z+1;
    ${\color{blue}%
    \dashv \cdot; \, 1 {+} |[x, n]| {+} |[z, n]| }$
  ${\color{blue}%
  \dashv \cdot; \, 1 {+} |[x, n]| {+} |[z, n]| \vdash }$
  tick(1);
  ${\color{blue}%
  \dashv \cdot; \, |[x, n]| {+} |[z, n]| }$
  }
   \end{lstlisting}

$|[x, n]| + |[z, n]|$
\\[.4\baselineskip]
      {\bf speed\_2}
    \end{center}
  \end{minipage}
%
\hfill
%
  \begin{minipage}[b]{\progwidth}
    \begin{center}
   \begin{lstlisting}
while (z-y>0) {
  ${\color{blue}%
  y{<}z; \, 3.1|[y,z]| {+} 0.1|[0,y]| \vdash }$
  y=y+1;
  ${\color{blue}%
  \dashv \cdot; \, 3 {+} 3.1|[y,z]| {+} 0.1|[0,y]| \vdash }$
  tick(3);
  ${\color{blue}%
  \dashv \cdot; \, 3.1|[y,z]| {+} 0.1|[0,y]| }$
}
${\color{blue}%
\dashv \cdot; \, 3.1|[y,z]| {+} 0.1|[0,y]| \vdash }$
while (y>9) {
  ${\color{blue}%
  y{>}9; \, 3.1|[y,z]| {+} 0.1|[0,y]| \vdash }$
  y=y-10;
  ${\color{blue}%
  \dashv \cdot; \, 1 {+} 3.1|[y,z]| {+} 0.1|[0,y]| \vdash }$
  tick(1);
  ${\color{blue}%
  \dashv \cdot; \, 3.1|[y,z]| {+} 0.1|[0,y]| }$
}
   \end{lstlisting}

$3.1|[y,z]| + 0.1|[0,y]|$
\\[.4\baselineskip]
      {\bf t08a}
    \end{center}
  \end{minipage}
%
\hfill
%
%
%
  \begin{minipage}[b]{\progwidth}
    \begin{center}
   \begin{lstlisting}
while (n<0) {
  ${\color{blue}%
  n{<}0; \,  P(n,y) \vdash }$
  n=n+1;
  ${\color{blue}%
  \dashv \cdot; \,  59 {+} P(n,y) \vdash }$
  y=y+1000;
  ${\color{blue}%
  \dashv \cdot; \, 9 {+} P(n,y) \vdash }$
  while (y>=100 && *){
    ${\color{blue}%
    y{>}99; \, 9 {+} P(n,y) \vdash }$
    y=y-100;
    ${\color{blue}%
    \dashv \cdot; \, 14 {+} P(n,y) \vdash }$
    tick(5);
    ${\color{blue}%
    \dashv \cdot; \, 9 {+} P(n,y) }$
  }
  ${\color{blue}%
  \dashv \cdot; \, 9 {+} P(n,y) \vdash }$
  tick(9);
  ${\color{blue}%
  \dashv \cdot; \, P(n,y) }$
}
   \end{lstlisting}
$59|[n,0]| {+} 0.05|[0,y]|$
\\[.4\baselineskip]
      {\bf t27}
    \end{center}
  \end{minipage}
\vspace{1ex}
\caption{Derivations of bounds on the number of ticks for challenging
  examples.  Examples \emph{speed\_1} and \emph{speed\_2}
  (from~\cite{GulwaniMC09}) use \emph{tricky iteration patterns},
  \emph{t08a} contains \emph{sequential loops} so that the iterations
  of the second loop depend on the first, and \emph{t27} contains
  interacting \emph{nested loops}. In the potential functions, we only
  mention the non-zero terms and in the logical context $\Gamma$ we
  only mention assertions that we actually use in the reasoning. In
  Example \emph{t27}, we use the abbreviation $P(n,y) := 59|[n,0]| {+}
  0.05|[0,y]|$.}
  \label{fig:ex_list}
\end{figure*}
%
%

\subsection{Automatic Amortized Analysis}

A program logic provides a principled foundation for statically
analyzing programs.  However, program logics need to be supported by
automatic methods to be useful in practice.  % That is why we developed a
% new automatic amortized analysis for deriving quantitative Hoare
% triples for common C programs.

To enable automation, we fix the shape of the potential functions
$\Phi$ so that it becomes possible to use \emph{linear programming}
(LP) to compute a derivation in the logic.  If we assume for now that
a program state $\state$ simply maps variables to integers then we
require
$$
\Phi(\state) = q_0 + \sum_{x,y \in \dom{\state}} q_{(x,y)} {\cdot} |[\state(x),\state(y)]|
$$
where $q_{(x,y)} \in \Qplusz$ and $|[a,b]| = \max(b-a,0)$.  Note that
$q_0$ is the constant potential and that we treat constants as program
variables.  For instance, we always have $q_{(0,x)}|[0,x]|$ as a
component of $\Phi(\state)$ if $x \in \dom{\state}$.
%
We then develop an inference rule for each syntactic construct that
derives a sound triple $\htriple{\Phi}P{\Phi'}$ in the quantitative
logic but enables inference using an LP solver.

This idea is best explained by example.  If we again use the
tick metric that assigns cost $n$ to the function call
\code{tick(n)}\iffull{ and cost $0$ to all other operations}{} then
the cost of the following example can be bounded by $|[x,y]|$.\footnote{Note
that there are no restrictions on the signs of the input variables in
the examples.}
\begin{lstlisting}[basicstyle=\tt\small]
while (x<y) { x=x+1; tick(1); } #\hfill \textnormal{(Example 1)}#
\end{lstlisting}
To derive this bound in our automatic amortized analysis, we start
with the initial potential $\Phi_0 = |[x,y]|$ (short for
$\Phi_0(\state) = |[\state(x),\state(y)]|$) which we also use as the
loop invariant.  For the loop body we then (like in Hoare logic) have
to derive a triple like
$\htriple{\Phi_0}{\code{x=x+1;tick(1)}}{\Phi_0}$.  We can
only do so if we utilize the fact that $x<y$ at the beginning of the
loop body.  % The reasoning then works as follows. We start with the
% potential $|[x,y]|$ and the fact that $|[x,y]| > 0$ before the
% assignment.
If we denote the updated version of $x$ after the
assignment by $x'$ then the relation $|[x,y]| = |[x',y]| + 1$ between
the potential before and after the assignment \code{x=x+1} holds.
This means that we have the potential $|[x,y]| + 1$ before the
statement \code{tick(1)}.  Since \code{tick(1)} consumes one resource unit,
we end up with potential $|[x,y]|$ after the loop body and have
established the loop invariant again.

Note that our notion of a potential function is a generalization of the
concept of a ranking function.  A potential function can be used like
a ranking function if we use the tick metric and add the statement
\code{tick(1)} to every back edge of the program (loops and function
calls).  However, a potential function is more flexible.  For example,
we can use a potential function to prove that Example~2 does not
consume any resources in the tick metric.
\begin{lstlisting}[basicstyle=\tt\small]
  while (x<y) { tick(-1); x=x+1; tick(1); } #\hfill \textnormal{(Example 2)}#
\end{lstlisting}
Similarly we can prove that Example~3 can be bounded by $10|[x,y]|$.
In both cases, we reason exactly like in the first version of the
while loop to prove the bound.  Of course, such loops with different
tick annotations can be seamlessly combined in a larger program.
\begin{lstlisting}[basicstyle=\tt\small]
  while (x<y) { x=x+1; tick(10); } #\hfill \textnormal{(Example 3)}#
\end{lstlisting}
%
More formally, we develop (in \pref{sec:AAA}) a judgement
$$
\Gamma; Q \vdash S \dashv Q'; \Gamma'
$$
such that $\Gamma$ contains logical assertions like $x<y$ that we
collect along the conditional branches of the program and $Q$ is a
family of coefficents $Q = (q_{(a,b)})_{a,b \in \text{scope}}$ that is
indexed by the variables currently in scope.  The logical
context $\Gamma$ is simply a conjunction of inequalities between
linear combinations of variables.  It is used, for example, to determine
at variable assignments if we can extract constant potential to pay for future
cost.  It is important to note that the reasoning about assertions in
$\Gamma$ is very basic.  We do not perform any fixpoint computations
and only derive trivial loop invariants about variables that are
unchanged in the loop body.

\pref{fig:ex1} shows a derivation of the bound
$\frac{T}{K}{\cdot}|[x,y]|$ on the number of ticks for a generalized
version of Example~1 in which we increment $x$ by a constant $K>0$ and
consume $T>0$ resources in each iteration.  The reasoning is similar
to the one of Example~1 except that we obtain the potential
$K{\cdot}\frac{T}{K}$ after the assignment.  Note that the logical
assertions in $\Gamma$ are only used in the rule for the assignment
\code{x=x+K}.  To the best of our knowledge, no other implemented tool
for C is currently capable of deriving a tight bound on the cost of
such a loop.  For $T=1$ (many systems focus on the number of loop
iterations without a cost model) and $K=10$,
KoAT~\cite{BrockschmidtEFFG14} computes the bound $|x| + |y| + 10$,
Rank~\cite{AliasDFG10} computes the bound $y-x-7$, and
LOOPUS~\cite{SinnZV14} computes the bound $y-x-9$.  Only
PUBS~\cite{AlbertAGPZ12} computes the tight bound $0.1(y-x)$ if we
translate the program into a term-rewriting system by hand.

To automate the reasoning, we first introduce an unknown rational
variable for each factor in the potential functions.  We then use our
inference rules (see \pref{sec:AAA}) to emit linear constraints on
these variables that enforce that variable assignments that respect
the constraints correspond to sound potential annotations.  For
instance if $K = 1$ in \pref{fig:ex1} then we would have the
annotation $q_0 + q_{(x,y)} {\cdot} |[x,y]| + q_{(y,x)} {\cdot}
|[y,x]|$ before the assignment and $p_0 + p_{(x,y)} {\cdot} |[x,y]| +
p_{(y,x)} {\cdot} |[y,x]|$ after the assignment, where $q_i$ and $p_i$
are unknown and must satisfy constraints like $p_{(x,y)} = q_{(x,y)}$,
$p_{(y,x)} = q_{(y,x)}$, and $p_0 = q_0 + q_{(x,y)} - q_{(y,x)}$.

\iffull{It might be surprising that we}{We} only track simple linear relations in
the logical context $\Gamma$.  While it would of course be possible to keep
track of more sophisticated assertions, this simple form is
sufficient for the examples we considered and we can efficiently
decide queries such as $\Gamma \implies x<y \; ?$ that we make in the
assignment rule.  Nevertheless, this logical part of our analysis
system is orthogonal to the quantitative part and can be easily
extended if necessary.

As mentioned earlier, the automatic analysis can handle challenging example
programs without special tricks or techniques.  Examples
\emph{speed\_1} and \emph{speed\_2}, that are taken from previous
work~\cite{GulwaniMC09},  demonstrate that our method can handle
\emph{tricky iteration patterns}.  The SPEED tool~\cite{GulwaniMC09}
derives the same bounds as our analysis but requires heuristics for
its counter instrumentation.  These loops can also be handled with inference of
\emph{disjunctive invariants}, but in the abstract interpretation
community, these invariants are known to be notoriously difficult to
generate.
%
In example \emph{speed\_1} we have one loop that first increments
variable $y$ up to $m$ and then increments variable $x$ up to $n$.  We
derive the tight bound $|[x, n]| + |[y, m]|$.
%
Example \emph{speed\_2} is even trickier, and we found it hard to
find a bound manually.  However, using potential transfer reasoning as
in amortized analysis, it is easy to prove the tight bound
$|[x, n]| + |[z, n]|$.

Example \emph{t08a} shows the ability of the analysis to discover
interaction between \emph{sequenced loops} through size change of
variables.  We accurately track the size change of $y$ in the first
loop by transferring the potential $0.1$ from $|[y,z]|$ to $|[0,y]|$.
Furthermore, \emph{t08a} shows again that we do not handle the
constants $1$ or $0$ in any special way.  In all examples we could
replace $0$ and $1$ with other constants like in the second loop and
still derive a tight bound.  \iffull{The only information, that the analyzer
needs is $y \geq c$ before assigning $y = y - c$.}{}
%
Example \emph{t27} shows how amortization can be used to handle
\emph{interacting nested loops}.  In the outer loop we increment the
variable $n$ until $n = 0$.  In each of the $|[n,0]|$ iterations, we
increment the variable $y$ by $1000$.  Then we non-deterministically
(expressed by \code{*}) execute an inner loop that decrements $y$ by
$100$ until $y<100$.  The analysis discovers that only the first
execution of the inner loop depends on the initial value of $y$.  We
again derive tight constant factors.
%

As mentioned, the analysis also handles advanced control flow like
\code{break} and \code{return} statements, and mutual recursion.
\pref{fig:ex_rec} contains two mutually-recursive functions with their
automatically derived tick bounds.  The function \code{count\_down}
decrements its first argument $x$ until it reaches the second argument
$y$. It then recursively calls the function \code{count\_up},
which is dual to \code{count\_down}.  Here, we
count up $y$ by $2$ and recursively call \code{count\_down}.  Our
analysis is the only available system that computes a tight
bound on this example.  \iffull{The analysis amounts to computing the meeting
point of two trains that approach each other with different speeds.}{}
% This is comparable to the
% determination of the point at which to trains meet if they approach
% each other with different speeds.

\iffull{We only show a small selections of the programs that we can
  handle automatically here.}{} \iffull{In \pref{app:cat}}{In the
  extended version~\cite{anon_extended}} is a list of more than $30$
classes of challenging programs that we can automatically analyze.
\pref{sec:exper} contains a more detailed comparison with other
tools\iffull{for automatic bound derivation.}{.}


\begin{figure}[t]
  \centering
    \begin{minipage}[b]{\linewidth}
    \begin{center}
   \begin{lstlisting}
void count_down (int x,int y) {
  if (x>y) { tick(1); count_up(x-1,y); } }

void count_up (int x, int y) {
  if (y+1<x) { tick(1); count_down(x,y+2); } }

   \end{lstlisting}

$0.33 + 0.67 |[y,x]|\;\;$ (\code{count\_down(x,y)})\\
$\;\;\;\;\;\;\,0.67 |[y,x]|\;\;$ (\code{count\_up(x,y)})
\\[.4\baselineskip]
      {\bf t39}
    \end{center}
  \end{minipage}
\vspace{-1.5ex}

  \caption{Two mutually-recursive functions with the computed tick bounds.  The derived constant factors are tight.}
  \label{fig:ex_rec}
\end{figure}


\sectskip
\section{Syntax and Semantics}
\label{sec:sem}
\aftersectskip

We implemented our cost semantics and the quantitative Hoare logic in
Coq for \emph{CompCert Clight}.  Clight is the most abstract
intermediate language used by CompCert.  Mainly, it is a subset of C
in which loops can only be exited with a \code{break} statement and
expressions are free of side effects.

\paragraph{Syntax.}

In this article, we describe our system for a subset of Clight that is
sufficient to discuss the general ideas.  This subset is given by the
following grammar.
%
\begin{align*}
S &:= \code{assert}~E
\mid \code{skip}
\mid \code{break}
\mid \code{return}~x
\mid x \gets E
\mid x \gets f(x^*)
\\
& \mid \code{loop}~S
\mid \code{if}(E)~S~\code{else}~S
\mid S;S
\mid \code{tick}(n)
\end{align*}
%
Expressions $E$ are left abstract in our presentation.  For our
analysis framework, it is only important that they are side-effect
free.
%
The most notable difference to full Clight is that we can only assign
to variables and thus do not consider operations that update the heap.
Moreover, function arguments and return values are assumed to be
variables.  This is only for simplifying the presentation; in the
implementation we can deal with heap updates and general function
calls and returns.  However, we have not implemented our framework for function
pointers, \code{goto} statements, \code{continue} statements, and
\code{switch} statements.

We include the built-in primitive $\code{assert}~e$ that terminates
the program if the argument $e$ evaluates to false and has no effect
otherwise.  This is useful to express assumptions on the inputs of a
program for the automatic analysis.  We also add the built-in function
\code{tick(n)} that can be called with a constant integer $n$ as a
flexible way to model resource consumption or release (if $n$ is
negative).


\iffull{
\begin{figure}[t!]
\begin{mathpar}
%
\Rule{S:Assert}
{ \code{istrue}~\sem{e}_\state }
{ (\state, \code{assert}~e, \cont, \cost)
\smallstep{M}
(\state, \code{skip}, \cont, \cost {-} M_a)
}
%
\and
\Rule{S:BrkSeq}
{}
{ (\state, \code{break}, \Kseq S \cont, \cost)
\smallstep{M}
(\state, \code{break}, \cont, \cost)
}
%
\and
\Rule{S:BrkLoop}
{}
{ (\state, \code{break}, \Kloop S \cont, \cost)
\smallstep{M}
(\state, \code{skip}, \cont, \cost {-} M_b)
}
%
\and
\Rule{S:RetSeq}
{}
{ (\state, \code{return}~x, \Kseq S \cont, \cost)
\smallstep{M}
(\state, \code{return}~x, \cont, \cost)
}
\and
%
\and
\Rule{S:RetLoop}
{}
{ (\state, \code{return}~x, \Kloop S \cont, \cost)
\smallstep{M}
(\state, \code{return}~x, \cont, \cost)
}
%
\and
\Rule{S:RetCall}
{ \state = (\_, \gamma) \\
  \state' = (\theta, \gamma)[r \mapsto \state(x)]
}
{ (\state, \code{return}~x, \Kcall r \theta \cont, \cost)
\smallstep{M}
(\state', \code{skip}, \cont, \cost {-} M_r)
}
%
\and
\Rule{S:Update}
{ \state' = \state[x \mapsto \sem{e}_\state] }
{ (\state, x \gets e, \cont, \cost)
\smallstep{M}
(\state', \code{skip}, \cont, \cost {-} M_u {-} M_e(e))
}
%
\and
\Rule{S:Call}
{ \fenv f = (\vec x, S_f) \\
  \state = (\theta, \gamma) \\
  \state' = (\vec x \mapsto \state(\vec y), \gamma) \\
}
{ (\state, r \gets f(\vec y), \cont, \cost)
\smallstep{M}
(\state', S_f, \Kcall r \theta \cont, \cost {-} M_f)
}
%
\and
\Rule{S:Loop}
{}
{ (\state, \code{loop}~S, \cont, \cost)
\smallstep{M}
(\state, S, \Kloop S \cont, \cost)
}
%
\and
\Rule{S:SkipLoop}
{}
{ (\state, \code{skip}, \Kloop S \cont, \cost)
\smallstep{M}
(\state, \code{loop}~S, \cont, \cost {-} M_l)
}
%
\and
\Rule{S:IfTrue}
{ \code{istrue}~\sem{e}_\sigma
\\ \cost' = \cost {-} M_c^1 {-} M_e(e)
}
{ (\state, \code{if}(e)~S_1~\code{else}~S_2, \cont, \cost)
\smallstep{M}
(\state, S_1, \cont, \cost')
}
%
\and
\Rule{S:IfFalse}
{ \code{isfalse}~\sem{e}_\sigma
\\ \cost' = \cost {-} M_c^2 {-} M_e(e)
}
{ (\state, \code{if}(e)~S_1~\code{else}~S_2, \cont, \cost)
\smallstep{M}
(\state, S_2, \cont, \cost')
}
%
\and
\Rule{S:Seq}
{ }
{ (\state, S_1; S_2, \cont, \cost)
\smallstep{M}
(\state, S_1, \Kseq {S_2} \cont, \cost)
}
%
\and
\Rule{S:SkipSeq}
{}
{ (\state, \code{skip}, \Kseq S \cont, \cost)
\smallstep{M}
(\state, S, \cont, \cost {-} M_s)
}
%
\and
\Rule{S:Tick}
{ }
{ (\state, \code{tick}(n), \cont, \cost)
\smallstep{M}
(\state, \code{skip}, \cont, \cost {-} M_t(n))
}

\end{mathpar}
\caption{\iffull{Rules}{Selected rules} of the operational semantics of statements.}
\label{fig:opsem}
\end{figure}
}{}

\ifshort{
\begin{figure}[t!]
\small
\begin{mathpar}
%
\Rule{S:Assert}
{ \code{istrue}~\sem{e}_\state }
{ (\state, \code{assert}~e, \cont, \cost)
\smallstep{M}
(\state, \code{skip}, \cont, \cost {-} M_a)
}
%
\and
\Rule{S:BrkSeq}
{}
{ (\state, \code{break}, \Kseq S \cont, \cost)
\smallstep{M}
(\state, \code{break}, \cont, \cost)
}
%
\and
\Rule{S:BrkLoop}
{}
{ (\state, \code{break}, \Kloop S \cont, \cost)
\smallstep{M}
(\state, \code{skip}, \cont, \cost {-} M_b)
}
%
\and
\Rule{S:Update}
{ \state' = \state[x \mapsto \sem{e}_\state] }
{ (\state, x \gets e, \cont, \cost)
\smallstep{M}
(\state', \code{skip}, \cont, \cost {-} M_u {-} M_e(e))
}
%
\and
\Rule{S:Loop}
{}
{ (\state, \code{loop}~S, \cont, \cost)
\smallstep{M}
(\state, S, \Kloop S \cont, \cost)
}
%
\and
\Rule{S:SkipLoop}
{}
{ (\state, \code{skip}, \Kloop S \cont, \cost)
\smallstep{M}
(\state, \code{loop}~S, \cont, \cost {-} M_l)
}
%
\and
\Rule{S:Seq}
{ }
{ (\state, S_1; S_2, \cont, \cost)
\smallstep{M}
(\state, S_1, \Kseq {S_2} \cont, \cost)
}
%
\and
\Rule{S:SkipSeq}
{}
{ (\state, \code{skip}, \Kseq S \cont, \cost)
\smallstep{M}
(\state, S, \cont, \cost {-} M_s)
}
\end{mathpar}
\caption{\iffull{Rules}{Selected rules} of the operational semantics of statements.}
\label{fig:opsem}
\end{figure}
}{}




\ifshort{
\begin{figure*}[t]
\small
\begin{mathpar}
%
\Rule{L:Skip}
{ }
{ \Delta; B; R \vdash
\htriple
  { Q }{ \code{skip} }{ Q }
}
%
\and
\Rule{L:Break}
{ }
{ \Delta; B; R \vdash
\htriple
  { M_b + B }{ \code{break} }{ Q }
}
%
\and
\Rule{L:Return}
{ }
{ \Delta; B; R \vdash
\htriple
  { R\,(\state(x)) }{ \code{return}~x }{ Q }
}
%
\and
\Rule[leftskip=.0cm,rightskip=.0cm]{L:Update}
{ }
{ \Delta; B; R \vdash
\htriple
  { \lambda \state.\,M_u + M_e(e) + Q\,\state[x \mapsto \sem{e}_\state] }{ x \gets e }{ Q }
}
%
\and
\Rule[leftskip=.0cm,rightskip=.0cm]{L:Seq}
{ \Delta; B; R \vdash
\htriple{ P }{ S_1 }{ Q' + M_s }
\\
\!\!  \Delta; B; R \vdash
\htriple{ Q' }{ S_2 }{ Q }
}
{ \Delta; B; R \vdash
\htriple{ P }{ S_1;S_2 }{ Q }
}
%
\and
\Rule{L:Call}
{
  \Delta(f) = \forall z\,\vec v\,v.(P_f\,z\,\vec v, Q_f\,z\,v) \\
  P \models P_f \, y \, (\state(\vec x)) \land A \\
  \forall v.\, (Q_f \, y \, v \land A \models \lambda \state.\, Q \, \state[r \mapsto v])
}
{ \Delta; B; R \vdash
\htriple
  { M_f + P }
  { r \gets f(\vec x) }
  { Q - M_r }
}
%
\and
\Rule{L:Assert}
{ }
{ \Delta; B; R \vdash
\htriple
  { \code{istrue}~\sem{e}_\state \implies Q + M_a }
  { \code{assert}~e }
  { Q }
}
%
\and
\Rule{L:Tick}
{ }
{ \Delta; B; R \vdash
\htriple
  { Q + M_t(n) }
  { \code{tick}(n) }
  { Q }
}
%
\and
\Rule{L:Loop}
{ \Delta; Q; R \vdash
\htriple
  { I }{ S }{ I + M_l }
}
{ \Delta; B; R \vdash
\htriple
  { I }{ \code{loop}~S }{ Q }
}
%
\and
\Rule{L:If}
{ \Delta; B; R \vdash
\htriple
  { \code{istrue}~\sem{e}_\state + P - M_c^1}
  { S_1 }{ Q }
\\
  \Delta; B; R \vdash
\htriple
  { \code{isfalse}~\sem{e}_\state + P - M_c^2 }
  { S_2 }{ Q }
}
{ \Delta; B; R \vdash
\htriple
  { P + M_e(e) }{ \code{if}(e)~S_1~\code{else}~S_2 }{ Q }
}
%
\and
\Rule{L:Weaken}
{
  \!\! P \models P' \!\! \\
  \!\! \Delta; B'; R' \vdash
  \htriple{ P' }{ S }{ Q' } \!\! \\
  \!\! Q' \models Q \!\! \\
  \!\! B' \models B \!\! \\
  \!\! \forall v.\, (R'\,v \models R\,v) \!\!
}
{ \Delta; B; R \vdash
\htriple{ P }{ S }{ Q }
}
%
\and
\Rule{L:Frame}
{ \Delta; B; R \vdash
\htriple{ P }{ S }{ Q } \\
  x \in \Qplusz
}
{ \!\! \Delta; B + x; R + x \vdash
\htriple{ P + x }{ S }{ Q + x } \!\!
}
%
\and
\Rule{L:Extend}
{ \Delta \cup \Delta'; B; R \vdash
  \htriple{ P }{ S }{ Q }
\\
  \forall f\,P_f\,Q_f.\,
  \Delta'(f) = \forall z\,\vec v\,v.(P_f\,z\,\vec v, Q_f\,z\,v)
  \rightarrow
  \forall y\,\vec v.\,
  (\Delta \cup \Delta'; \bot; Q_f\,y \vdash
    \htriple{ P_f\,y\,\vec v }{ S_f }{ \bot })
}
{ \Delta; B; R \vdash
\htriple{ P }{ S }{ Q }
}
%
\end{mathpar}

\caption{Rules of the Quantitative Hoare Logic}
\label{fig:logic}
\end{figure*}
}{}


\paragraph{Semantics.}

CompCert Clight's operational semantics is based on small-step
transitions and continuations.  Expressions---which do not have side
effects---are evaluated in a big-step fashion.  Here, we describe a
simplified version of Clight's semantics for the subset we consider.

A program state $\state = (\theta, \gamma)$ is composed of two maps from
variable names to integers. The first map, $\theta : \text{Locals} \to
\Z$, assigns integers to local variables of a function, and the second
map, $\gamma : \text{Globals} \to \Z$, gives values to global variables
of the program.  In this article, we assume that all values are
integers but in the implementation we support all data types of
Clight.
%
The evaluation function \evalE{\cdot}{} maps an expression $e \in
E$ to a value $\evalE{e}{\state} \in \Z$ in the program state
$\state$.

For simplicity, we assume that local variables and global variables
are always different.  We just write $\state(x)$ to obtain the value of
$x$ in program state $\state$. \iffull{Such a lookup is defined as
$$
(\theta, \gamma)(x) =
\left\{
\begin{array}{lr}
\theta(x) & \mbox{ if } x \in \dom \theta \\
\gamma(x) & \mbox{ if } x \in \dom \gamma
\end{array}
\right..
$$}{}
\iffull{For a program state $\state$, we}{Similarly} we write $\state[x \mapsto v]$
for the program state that maps $x$ to $v$ and behaves as $\state$ for
all other variables, regardless whether $x \in \dom \theta$ or $x \in
\dom \gamma$.

The small-step semantics is standard, except that it tracks the
resource consumption of a program.  The semantics is parametric in the
resource of interest for the user of our system.  We achieve this
independence by parameterizing evaluations with a resource
\emph{metric} $M$; a tuple of nine rational numbers \ifshort{$M_a$, $M_b$, $M_r$, $\ldots$}{}, one map $M_e :
\text{Expr} \to \Q$ from expressions to rational numbers, and one map
$M_t : \Z \to \Q$ from integers to rational numbers.
%
\iffull{$$
M = (
M_a,
M_b,
M_r,
M_u,
M_f,
M_l,
M_c^1,
M_c^2,
M_s,
M_e(E),
M_t(n)
)
$$}{

}
Each of these rational numbers indicates the amount of resource
consumed by a corresponding operation.  If the assigned resource cost
is negative then it means that resources are released.  The metric
that we use in the implementation can also depend on other information
that is statically available such as the name of the called function
or the number of arguments.

\pref{fig:opsem} contains \ifshort{selected}{the} reduction rules of the semantics.  The
rules define a rewrite system for program configurations of the form
$(\state, S, \cont, \cost)$, where $\state$ is the program state, $S$
is the statement being executed, $\cont$ is a continuation that
describes what remains to be done after the execution of $S$, and
$\cost \in \Q$ is the non-negative number of resources available for
further execution.  All rules that can decrease $\cost$ have the
implicit side condition that the resource quantity available
\emph{before} the step is non-negative.  This means that we allow
$(\state, S, \cont, \cost)$ with $c<0$ on the right-hand side
transition relation $\smallstep{M}^*$ to indicate that the execution
ran out of resources.  However, every execution that reaches such a
state is stuck.

\iffull{
A continuation $\cont$ represents the context of the execution of a
given statement.  A continuation can be the empty continuation
$\Kstop$ (used to start a program), a sequenced computation $\Kseq S
\cont$, a loop continuation $\Kloop S \cont$, or the continuation
$\Kcall r \theta \cont$ of a function call.
$$
\cont := \Kstop
\mid \Kseq S \cont
\mid \Kloop S \cont
\mid \Kcall r \theta \cont
$$
%
During the execution we assume a fixed function context $\fenv$ that
maps function names to a list of variables and the statement that
defines the function body.  It is used in the rule \textsc{S:Call}.
}{}

The intuitive meaning of an evaluation \iffull{judgement}{} $(\state,
S, \cont, \cost) \smallstep{M}^* (\state', S', \cont', \cost')$ is the
following.  If the statement $S$ is executed in program state
$\state$, with continuation $\cont$, and with $\cost$ resources
available then---after a finite number of steps---the evaluation will
reach the new machine state $(\state', S', \cont',\cost')$ and there
are $\cost'$ resources available.  If $\cost' \ge 0$ then the
execution did not run out of resources and the resource consumption up
to this point is $\cost - \cost'$.  If this difference is negative
then resource became available during the execution.  If however
$\cost' < 0$ then the execution ran out of resources and is stuck.
The cost of the execution is then $\cost \ge 0$.

\iffull{
\pref{lem:sem1} summarizes the main properties of the resource
counter of the semantics.  It can be proved by induction on the number
of steps.
\begin{lemma}
\label{lem:sem1}
\itemskip
  \begin{enumerate}
  \item If $(\state, S, \cont, \cost) \smallstep{M}^k (\state', S',
    \cont', \cost')$ and $n\ge 0$ then $(\state, S, \cont, \cost {+} n)
    \smallstep{M}^k (\state', S', \cont', \cost' {+} n)$.
\itemskipIn
  \item If $(\state, S, \cont, \cost) \smallstep{M}^k (\state', S',
    \cont', \cost')$ and $(\state, S, \cont, d) \smallstep{M}^k (\state', S',
    \cont', d')$ then $\cost - \cost' = d - d'$.
  \end{enumerate}
\itemskip
\end{lemma}
%
If we assume that we have a \emph{positive resource metric}, that is,
a metric that assigns a positive cost to each evaluation step then we
can link bounds on the resource consumption under the metric to
termination.  This is formalized by \pref{lem:sem_term}.
%
\begin{lemma}
 \label{lem:sem_term}
 Let $M$ be a positive metric and assume that there exists $b \in
 \Qplusz$ such that $(\state, S, \cont, \cost) \smallstep{M}^*
 (\state', S', \cont', \cost') \implies \cost {-} \cost' {\leq} b$.
 Then there exists $B \in \N$ such that $(\state, S, \cont, d)
 \smallstep{M}^n (\state'', S', \cont', d')$ for some $n$, $d$, $d'$
 and $\state''$ implies $n \le B$.
\end{lemma}
}
{
The extended version of this article~\cite{anon_extended} contains all rules and
lemmas that state the main properties of the cost semantics.
}


\iffull{
\begin{figure*}[t]
\begin{mathpar}
%
\Rule{L:Skip}
{ }
{ \Delta; B; R \vdash
\htriple
  { Q }{ \code{skip} }{ Q }
}
%
\and
\Rule{L:Break}
{ }
{ \Delta; B; R \vdash
\htriple
  { M_b + B }{ \code{break} }{ Q }
}
%
\and
\Rule{L:Return}
{ }
{ \Delta; B; R \vdash
\htriple
  { R\,(\state(x)) }{ \code{return}~x }{ Q }
}
%
\and
\Rule[leftskip=.0cm,rightskip=.0cm]{L:Update}
{ }
{ \Delta; B; R \vdash
\htriple
  { \lambda \state.\,M_u + M_e(e) + Q\,\state[x \mapsto \sem{e}_\state] }{ x \gets e }{ Q }
}
%
\and
\Rule[leftskip=.0cm,rightskip=.0cm]{L:Seq}
{ \Delta; B; R \vdash
\htriple{ P }{ S_1 }{ Q' + M_s }
\\
\!\!  \Delta; B; R \vdash
\htriple{ Q' }{ S_2 }{ Q }
}
{ \Delta; B; R \vdash
\htriple{ P }{ S_1;S_2 }{ Q }
}
%
\and
\Rule{L:Call}
{
  \Delta(f) = \forall z\,\vec v\,v.(P_f\,z\,\vec v, Q_f\,z\,v) \\
  P \models P_f \, y \, (\state(\vec x)) \land A \\
  \forall v.\, (Q_f \, y \, v \land A \models \lambda \state.\, Q \, \state[r \mapsto v])
}
{ \Delta; B; R \vdash
\htriple
  { M_f + P }
  { r \gets f(\vec x) }
  { Q - M_r }
}
%
\and
\Rule{L:Assert}
{ }
{ \Delta; B; R \vdash
\htriple
  { \code{istrue}~\sem{e}_\state \implies Q + M_a }
  { \code{assert}~e }
  { Q }
}
%
\and
\Rule{L:Tick}
{ }
{ \Delta; B; R \vdash
\htriple
  { Q + M_t(n) }
  { \code{tick}(n) }
  { Q }
}
%
\and
\Rule{L:Loop}
{ \Delta; Q; R \vdash
\htriple
  { I }{ S }{ I + M_l }
}
{ \Delta; B; R \vdash
\htriple
  { I }{ \code{loop}~S }{ Q }
}
%
\and
\Rule{L:If}
{ \Delta; B; R \vdash
\htriple
  { \code{istrue}~\sem{e}_\state + P - M_c^1}
  { S_1 }{ Q }
\\
  \Delta; B; R \vdash
\htriple
  { \code{isfalse}~\sem{e}_\state + P - M_c^2 }
  { S_2 }{ Q }
}
{ \Delta; B; R \vdash
\htriple
  { P + M_e(e) }{ \code{if}(e)~S_1~\code{else}~S_2 }{ Q }
}
%
\and
\Rule{L:Weaken}
{
  \!\! P \models P' \!\! \\
  \!\! \Delta; B'; R' \vdash
  \htriple{ P' }{ S }{ Q' } \!\! \\
  \!\! Q' \models Q \!\! \\
  \!\! B' \models B \!\! \\
  \!\! \forall v.\, (R'\,v \models R\,v) \!\!
}
{ \Delta; B; R \vdash
\htriple{ P }{ S }{ Q }
}
%
\and
\Rule{L:Frame}
{ \Delta; B; R \vdash
\htriple{ P }{ S }{ Q } \\
  x \in \Qplusz
}
{ \!\! \Delta; B + x; R + x \vdash
\htriple{ P + x }{ S }{ Q + x } \!\!
}
%
\and
\Rule{L:Extend}
{ \Delta \cup \Delta'; B; R \vdash
  \htriple{ P }{ S }{ Q }
\\
  \forall f\,P_f\,Q_f.\,
  \Delta'(f) = \forall z\,\vec v\,v.(P_f\,z\,\vec v, Q_f\,z\,v)
  \rightarrow
  \forall y\,\vec v.\,
  (\Delta \cup \Delta'; \bot; Q_f\,y \vdash
    \htriple{ P_f\,y\,\vec v }{ S_f }{ \bot })
}
{ \Delta; B; R \vdash
\htriple{ P }{ S }{ Q }
}
%
\end{mathpar}

\caption{Rules of the Quantitative Hoare Logic}
\label{fig:logic}
\end{figure*}
}{}

\sectskip
\section{Quantitative Hoare Logic}
\label{sec:logic}
\aftersectskip

In this section we describe a simplified version of the quantitative
Hoare logic that we use in Coq to interactively prove resource bounds.
%
We generalize classic Hoare logic to express not only classical
boolean-valued assertions but also assertions that talk about the
future resource usage.  Instead of the usual assertions $P : \State
\to \mathit{bool}$ of Hoare logic we use assertions
$$
P : \State \to \Qplusz \cup \{ \infty \} \; .
$$
This can be understood as a refinement of boolean assertions where
$\mathit{false}$ is $\infty$ and $\mathit{true}$ is refined by $\Qplusz$.
We write $\Assn$ for $\State \to \Qplusz \cup \{ \infty \}$ and $\bot$ for
$\lambda \state . \, \infty$.  We sometimes call assertions
\emph{potential functions}.  To use Coq's support for propositional
reasoning, assertions have the type $\State \to \Qplusz \to \mathrm{Prop}$
in the implementation.  For a given $\state \in \State$, such an
assertion can be seen as a set $B \subseteq \Qplusz$ of valid bounds.
However, we find the presentation in this article easier to read.

Due to \code{break} and \code{return} statements of Clight, there are
different possible ways to exit a block of code.  We also have to keep
track of the resource specifications of functions.  To account for
this in the logic, our quantitative Hoare triples have the form
$$
\Delta; B; R \vdash
\htriple
  { Q }
  { S }
  { Q' } \; .
$$
The triple $\htriple{ Q }{ S }{ Q' }$ consists of a statement $S$ and
two assertions $Q,Q' : \Assn$.  It corresponds to triples in classic
Hoare logic and the intuitive meaning is as follows.  If $S$ is
executed with starting state $\state$, the empty continuation
$\Kstop$, and at least $P(\state)$ resources available then the
evaluation does not run out of resources and there are at least
$Q(\state')$ resources left if the evaluation terminates in state
$\state'$.  The assertion $B : \Assn$ provides the
postcondition for the case in which the code block $S$ is exited by a
\code{break} statement.  So if the execution is terminated in state
$\state'$ with a \code{break} then $B(\state')$ resources are
available.  Similarly, $R : \Z \to \Assn$ is the postcondition for the
case in which the code block $S$ is exited by a $\code{return}~x$
statement.  The integer argument of $R$ is the return value.
Finally, the function context of judgements that we write $\Delta$ is
a mapping from function names to specifications of the form
$$
  \forall z \, \vec v \, v .(P_f \, z \, \vec v, Q_f \, z \, v).
$$
The assertion $P_f \, z \, \vec v$ is the precondition of the function
$f$ and the assertion $Q_f \, z \, v$ is its postcondition.  They are
both parameterized by an arbitary logical variable $z$ (which can be a
tuple) that relates the function arguments with the return value.  The
precondition also depends on $\vec v$, the values of the arguments
at the function invocation.  Similarly, the postcondition depends on
the return value $v$ of the function.  The use of logical variables to
express relations between different states of an execution is a
standard technique of Hoare logic.
%
To ensure soundness, we require that $P_f$ and $Q_f$ do not depend on
the local variables on the stack, that is, $\forall z \, \vec v
\, \theta \, \theta' \, \gamma . \, P_f \, z \, \vec v \,
(\theta,\gamma) = P_f \, z \, \vec v \, (\theta',\gamma)$.

For two assertions $P,Q : \Assn$, we write $P \models Q$ to if for all
program state $\state$ $P(\state) \ge Q(\state)$.

\paragraph{Rules of the Quantitative Logic.}

\pref{fig:logic} shows the inference rules of the quantitative logic.
The rules are slightly simplified in comparison to the implemented
rules in Coq.  The main difference is that the presented version does
not formalize the heap operations.

\iffull{In the rule {\sc L:Skip}, we do not have to account for any resource
consumption.  As a result, the precondition $Q$ can be any (potential)
function and we only have to make sure that we do not end up with more
potential.  Since the execution of \code{skip} leaves the program
state unchanged, we can simply use the precondition as postcondition.
The potential functions $B$ for the \code{break} and $R$ for the
\code{return} part of the postcondition are not reachable and can
therefore be arbitrary.}{}

In the rule {\sc L:Assert}, we use the notation
$\code{istrue}~\sem{e}_\state \implies Q + M_a$ to express that we
require potential $Q + M_a$ in the precondition if $e$ evaluates to
$\mathit{true}$ in the current program state.  If $e$ evaluates to
$\mathit{false}$ then the potential in the precondition can be
arbitrary since the program will be terminated.

In the rules {\sc L:Break} and {\sc L:Return}, the postcondition can
be arbitrary since it is unreachable.  Instead, we have to justify the
potential functions $B$ and $R$ that hold after a \code{return} and a
\code{break}, respectively.  In {\sc L:Break}, we require to have
potential $M_b+B$ in the precondition: $M_b$ to pay for the execution
cost of \code{break} and $B$ to pay for the potential after the
\code{break}.  In {\sc L:Return} we only require to have potential
$R$ in the precondition to pay for the potential after the
\code{return}.  The reason is that we found it to be more convenient
to account for the execution cost $M_r$ of the return in rule {\sc
  L:Call}\iffull{ for function calls.}{.}

\iffull{The rule {\sc L:Update} is the standard assignment rule of Hoare
logic.  With the substitution $\state[x \mapsto \sem{e}_\state]$ in
the precondition we ensure that $Q$ evaluates to the same number as in
the postcondition.  We also require that have we the constant
potential $M_u + M_e(e)$ available in the precondition to pay for the
cost of the evaluation of $e$ and the update.

The rule {\sc L:Seq} rule is crucial to understand how the
quantitative Hoare logic works.  To account for early exits of
statements, we must ensure in the \code{break} part $B$ of $S_1$'s
judgement that the \code{break} part $B$ of of $S_1;S_2$ holds. The
same is true for the return part $R$ of the judgements for $S_1$ and
$S_2$.  The interaction between the actual pre- and postconditions is
analogous to standard Hoare logic.  In the postcondition of $S_1$ we
account for the cost $M_s$ of the execution of the sequence.

}
{
The rules {\sc L:Update} and {\sc L:Seq} correspond to the respective
rules of standard Hoare logic.
}
In the rule {\sc L:Loop}, the \code{break} part of the
loop body $S$ becomes the postcondition of the loop statement. We use
an arbitrary $B$ as the \code{break} part of the judgement for
$\code{loop}\, S$ since its operational semantics ensures that it can
only terminate with a \code{skip} or a \code{return}.  The
precondition $I$ of the loop is the loop invariant.  \iffull{That is why we
require to have potential $I$ available in the precondition of the
loop body $S$.}{}  In the postcondition of $S$, the potential must be
sufficient to pay for the invariant $I$ and the cost $M_l$ of the loop
iteration.

\iffull{
The {\sc L:If} is similar to the rule for the conditional in classic
Hoare logic.  In the preconditions of the judgments for the two
branches $S_1$ and $S_2$ we lift the boolean assertions
$\code{isfalse}~\sem{e}_\state$ and $\code{isfalse}~\sem{e}_\state$ to
quantitative assertions.  In the precondition of the rule {\sc L:Tick}
we account for the execution cost $M_t(n)$ of the \code{tick}
statement that depends on the integer $n$.
}{}

The rule {\sc L:Call} accounts for the execution cost of both $M_f$
for function calls and $M_r$ for \code{return} statements.
\iffull{This justifies that the {\sc L:Return} rule does not account
  for resource consumption.}{} The pre- and postcondition $P_f$ and
$Q_f$ are taken from the function context $\Delta$.  The assertions in
the context are parametric with respect to both the values of the
function arguments and the return value. This allows us to specify a
bound for a function whose resource consumption depends on its
arguments.  The arguments are instantiated by the call rule
using the result of the evaluation of the argument variables in the
current state.  \iffull{Recall that we require that $P_f$ and $Q_f$ do
  not depend on the local variables on the call stack, that is,
  $\forall z \, \vec v \, \theta \, \theta' \, \gamma . \, P_f \, z \,
  \vec v \, (\theta,\gamma) = P_f \, z \, \vec v \,
  (\theta',\gamma)$.}{} To transfer potential that depends on local
variables of the callee from the precondition $P$ to the postcondition
$Q$, we use an assertion $A : \Assn$ that is independent of global
variables, that is, $\forall \theta \, \gamma \, \gamma' . \, A
(\theta,\gamma) = A(\theta,\gamma')$.  It is still possible to express
relations between global and local variables using logical
variables\iffull{ (see the following paragraph for details).}{.}

Finally, we describe the rules which are not syntax directed.  There
are two weakening rules available in the quantitative Hoare logic.
The framing rule {\sc L:Frame} is designed to weaken a statement by
stating that if $S$ needs $P$ resources to run and leaves $Q$
resources available after its execution, then it can very well run
with $P + c$ resources and return $Q + c$ resources.  The consequence
{\sc L:Conseq} rule is directly imported from classical Hoare logic
except that instead of using the logical implication $\Rightarrow$ we
use the quantitative $\models$ that point-wise applies $\ge$.  \iffull{This
rule indeed weakens the statement since it requires more resource to
run the statement and yields less than what has been proved to be
available after its termination.}{}

\ifx\fullversion\undefined{}\else{
\paragraph{Logical Variables and Shallow Embedding.}

When specifying a program, it is often necessary to relate certain
invariants in the pre- and postcondition.  A standard solution of Hoare
logic is to use \emph{logical variables}~\cite{Kleymann99}.  These
additional variables (also called auxiliary state in the literature)
are constant across the derivation.  For example, if $Z$ is such
a logical variable, we can specify the function $\code{double}()$
which doubles the global variable $x$ as
$$
  \htriple{z = Z}{\code{double}()}{x = 2 \cdot Z}.
$$
%
When formalizing Hoare logics in a proof assistant one can either
fully specify the syntax and semantics of assertions and hence get a
deep embedding, or use the assertions of the host theory to get a
shallow embedding.  Because of its flexibility, we used the latter
approach in our development.  This choice makes it possible to have
logical variables almost for free: we can simply use the variable and
binding mechanisms of Coq, our host theory.  When an logical variable
is needed we introduce it using a universal quantifier of Coq before
using the logic to derive triples.  For example, the Coq theorem for
the above example would look as follows.
%
\begin{lstlisting}
Theorem dbl_triple: forall Z,
  qtriple ($\lambda\sigma$.$\sigma$(x) = Z) ($\code{double}()$) ($\lambda\sigma$.$\sigma$(x) = 2*Z).
\end{lstlisting}
%
However, this trick alone is often not sufficient when working with a
recursive function $f$.  In that case we apply the {\sc L:Extend} rule
of the logic.  First we add a specification $\forall z\,\vec
v\,v.(P_f\,z\,\vec v, Q_f\,z\,v)$ of $f$ to the function context
$\Delta$.  Then we proceed to prove the function body $S_f$ with this
induction hypothesis.  In this process it can be the case that we have
to use the induction hypothesis with a different value of a logical
variable (e.g., because the values of the arguments in the recursive
call differ from the values of the arguments of the callee).  To cope
with this problem, assumptions in the function context $\Delta$ are
universally quantified over logical variables.  The {\sc L:Extend}
rule uses the host proof assistant to require that the triple on $f$'s
body is proved for every possible logical variable $y$.
}\fi

\paragraph{Using the Quantitative Logic.}

In the following we demonstrate the use of the logic with two example
derivations.

In the example in~\pref{fig:xmplmax} we derive a precise runtime
bound on a program that searches a maximal element in an array.  The
cost metric that we use simply counts the assignments performed by the
program.  Hence, the resource cost is closely related to the
number of times the test \code{a[i] > m} is true during the
execution.
%
If we define
$$
 A(i) =  \# \{ k \mid i \le k < N \land \forall\, 0 \le j < k.\, a[j] < a[k] \} \, .
$$
where we write $\# S$ for the cardinal of the set $S$ then $A(1)+1$ is
the number of ``maximum candidates'' in the array \code{a} seen
by the algorithm. $A(1)$ is bounded by $N$, the size of the array.
So any automated tool would at best derive the linear bound $2 \cdot
N$ for that program.  But with the expressivity of our logic it is
possible to use the previous set cardinal directly and precisely tie
the bound to the initial contents of the array.
%
The non-trivial part of this derivation is finding the loop invariant
$(m = \max_{k \in [0, i-1]} a[k]) + A(i) + (N-i)$ for the while loop.
When the condition \code{a[i] > m} is true, we know that we
encountered a ``maximum so far'' because $m$ is a maximal element of
\code{a[0 \dots i]}, thus $A(i) = 1 + A(i+1)$ and we get one
potential unit to pay for the assignment.  In the other case, no
maximum so far is encountered so $A(i) = A(i+1)$.
% As a side remark, if $K \ge 0$, it is possible to
% have $\{ A(0) + N + K \}$ and $\{ K \}$ as pre- and postcondition
% of the same program.  It can be done either by adapting the proof or by
% applying the {\sc L:Frame} rule one top of the triple already derived.
% More generally, without function calls, the {\sc L:Frame}
% rule is admissible in our system.

The example in~\pref{fig:xmplbs} shows a use case for logical
variables as well as a metric for stack consumption. In a stack
metric, we account a constant cost for a function call ($M_f>0$) that
is returned after the call ($M_r = -M_f <0$).  All other resource cost
are $0$.  We are interested in showing that a binary search function
$\code{bsearch}$ has logarithmic stack consumption.  We use a logical
variable $Z$ in the function specification $\htriple{(Z = \log_2(h-l))
  + Z \cdot M_\code{bsearch}}{}{Z \cdot M_\code{bsearch}}$ to express that the stack required by the
function is returned after the call.
%
The critical step in the proof is the application of the
{\sc L:Call} rule to the recursive call.  At this point
the context $\Delta$ contains the specification
$
  \forall y \, (x,l,h) \, \_.
  ( (\lambda y \, (\_, l, h) .\, (y = \log_2(h {-} l)) {+} y {\cdot} M_\code{bsearch})
  , (\lambda y \, \_ .\, y {\cdot} M_\code{bsearch})
  )
$.
%
Using the rule {\sc L:Call} it is possible to instantiate $y$ with $Z
- 1$, and because $M_f = M_\code{bsearch}$ and $M_r =
-M_\code{bsearch}$ in the stack metric.  The rest of the proof does
not involve any resource manipulation and is just bookkeeping of
logical assertions.



\begin{figure}
\begin{lstlisting}
${\color{blue} \{ A(0) + N \} }$
i=1; m=a[0];
${\color{blue}%
  \{ (i {=} 1 \land m=a[0]) + A(1) + (N-1) \} }$
while (i < N) {
  ${\color{blue}%
    \{ (m = \max_{k \in [0, i-1]} a[k]) + A(i) + (N-i) \} }$
  if (a[i] > m)
    m=a[i];
  ${\color{blue}%
    \{ (m = \max_{k \in [0, i]} a[k]) + A(i+1) + (N-i) \} }$
  i=i+1;
  ${\color{blue}%
    \{ (m = \max_{k \in [0, i-1]} a[k]) + A(i) + (N-i) \} }$
} ${\color{blue} \{ 0 \} }$
\end{lstlisting}
\caption{Example derivation where we wrote $A(i)$
  for $\#\{ k \mid i \le k \le N \land \forall\, 0\le j<k.\, a[j] < a[k]\}$,
  the metric used here assigns a cost of 1 to every assignment
  and 0 to all other operations.
  }
\label{fig:xmplmax}
\end{figure}

\begin{figure}
\begin{lstlisting}
${\color{blue}%
  \{ (Z = \log_2(h-l)) + Z \cdot M_\code{bsearch} \} }$
bsearch(x,l,h) {
  if (h-l > 1) {
    ${\color{blue}%
      \{ (Z \ge 1 \land Z = \log_2(h-l)) + Z \cdot M_\code{bsearch} \} }$
    m = h + (h-l)/2;
    ${\color{blue}%
      \{ (m = \frac{h+l}{2} \land Z \ge 1 \land Z = \log_2(h-l)) + Z \cdot M_\code{bsearch} \} }$
    if (a[m]>x) h=m; else l=m;
    ${\color{blue}%
      \{ (Z - 1 = \log_2(h-l)) + (Z-1) \cdot M_\code{bsearch} + M_\code{bsearch} \} }$
    l = bsearch(x,l,h); }
  ${\color{blue}%
    \{ (Z-1) \cdot M_\code{bsearch} - (-M_\code{bsearch}) \} }$
  return l;
} ${\color{blue} \{ Z \cdot M_\code{bsearch} \} }$
\end{lstlisting}
\caption{Example derivation of a stack usage bound for a binary
  search program.  The used resource metric defines the cost $M_\code{bsearch}$ before
  the function call and $-M_\code{bsearch}$ after the call.  $M_\code{bsearch}$ is the stack
  frame size of the function $\code{bsearch}$.  $Z$ is a
  logical variable.
  }
\label{fig:xmplbs}
\end{figure}


\paragraph{Soundness of Quantitative Hoare Triples.}

We already gave an intuition of the meaning of judgements
derived in the logic.  To make it formal, we define the
\emph{resource safety} $\safe n P S \cont$ of an assertion
and a program configuration as $\forall \state \, \cost \, m \, \cost' .$
\begin{align*}
  (m {\le} n \land P(\state) {\le} \cost \land
    (\state, S, \cont, \cost) \smallstep{M}^m (\_, \_, \_, \cost'))
  \implies \cost' {\ge} 0.
\end{align*}
This predicate is step indexed by an integer $n$ that is used for
induction in the soundness proof for the function-call and loop cases.
The constraint $\cost' \geq 0$ imposed by the definition ensures that
the program does not stop because of a resource error (recall that a
negative resource counter on the right-hand side is a resource
failure).  However, it does not rule out memory safety errors or
assertion failures. This is because our logic does not prove any
safety or correctness theorems but only focuses on resource usage.
%
An interesting detail in the definition is the natural number $m$.  We
simply use it to ensure that $\safe {n+1} P S \cont \implies \safe n P S
\cont $.  This would not be the case if we replaced all occurrences of
$m$ by $n$ in the definition\iffull{ of validity.}{.}

The resource safety of a continuation $\cont$ is defined using three
assertions, one for each of the possible outcomes of a program
statement.  \iffull{We define it as follows.
\begin{align*}
\safeK n B R Q \cont & := & & \safe n B {\code{break}} \cont \\
& & \land& \safe n {\lambda \state.\,R\,(\state(x))} {\code{return}~x} \cont \\
& & \land& \safe n Q {\code{skip}} \cont
\end{align*}}
{
We define $\safeK n B R Q \cont \mathop{:=}
\safe n B {\code{break}} \cont \, \land \,
\safe n Q {\code{skip}} \cont \, \land \,
\safe n {\lambda \state.\,R\,(\state(x))} {\code{return}~x} \cont$.

}
%
We can now define the \emph{semantic validity} of a judgement $B; R
\vdash \htriple P S Q$ of the quantitative logic without function
context as $\valid n B R P S Q :=$
\begin{align*}
& \forall m \, \cont \, x \geq 0 .\, (m \le n \land \safeK m {B{+}M_b{+}x} {R{+}x} {Q{+}x} \cont) \\
& \implies \safe m {P+x} S \cont \; .
\end{align*}
Note how the validity of a triple embeds the frame rule of
our logic. This refinement is necessary to have a stronger
induction hypothesis available during the proof.
%
We again need to add the auxiliary $m$ to ensure that $\valid {n{+}1} B R P
S Q$ implies $\valid n B R P S Q$.

Using the semantic validity of triples we define the validity
of a function context $\Delta$, written $\validC n \Delta$, as
\begin{align*}
  &\forall f .\, \Delta(f) =
    \forall z \, \vec v \, v .(P_f \, z \, \vec v, Q_f \, z \, v)
    \implies \\
  &\qquad \forall z \, \vec v .\,
  \valid n \bot {Q_f \, z} {P_f \, z \, \vec v} {S_f} \bot ,
\end{align*}
where $S_f$ is the body of the function $f$. \iffull{A full
judgement that mentions a non-empty function context
$\Delta$ is in fact a guarded statement: it makes
assumptions on some functions' behavior.}{}  The predicate
$\validC n \Delta$ gives the precise meaning of the
assumptions made.  It is also step-indexed to prove the
soundness of the {\sc L:Extend} rule by induction.
%
We are now able to state the soundness of the quantitative logic.
%
\begin{theorem}[Soundness of the logic]
  If $\Delta; B; R \vdash \htriple P S Q$ is derivable then
  $
    \forall n .\, \validC n \Delta
      \implies \valid {n {+} 1} B R P S Q.
  $
\end{theorem}
%
\noindent
The difference $\delta = 1$ between the index in the triple
validity and the one in context validity arises from
the soundness proofs of {\sc L:Call} and {\sc L:Extend}.  For
{\sc L:Call}, the language semantics makes one step and
proceeds with the function body, so we must have
$\delta \le 1$ to use the assumptions in $\Delta$.
For {\sc L:Extend}, we have to show that $\Delta \cup \Delta'$
is a valid context for $n$ steps.  The induction hypothesis
in that case says that if $\Delta \cup \Delta'$ is valid
for $m$ steps, $\Delta'$ is valid for $m+\delta$ steps.
So if we want to solve this goal by induction, it is
necessary that $\delta \ge 1$.  These two constraints force
$\delta$ to be exactly one in the theorem statement.

Assume that $S$ is a complete program and $\Delta$ is empty.  By
expanding the definitions we see that $\Delta$ is valid
for every $n$ and that $\Kstop$ is safe for every $n$. So
we derive
$$
\Delta; B; R \vdash \htriple P S Q \implies   \forall n .\, \safe n P S \Kstop.
$$
This means that from any starting state $\state$, $P(\state)$
provides enough resources for any run of the program $S$.  \iffull{This
setting is actually the main use case of the previous theorem
which is stated as a stronger result to allow a proof by
induction.}{}



\sectskip
\section{Automatic Amortized Analysis}
\label{sec:AAA}
\aftersectskip

In this section we present the automatic amortized analysis that we
use in our implementation to derive resource bounds for C
programs.

\paragraph{Linear Potential Functions.}

To find derivations in the quantitative Hoare logic automatically, we
have to focus on bounds that have a certain shape.  The general form
of \emph{potential functions} (or assertions) that we consider is
$$
  \Phi(\state) = q_0 + \sum_{x, y \in \dom\state \land x \neq y}
    q_{(x, y)} \cdot |\inter {\state(x)} {\state(y)}| \, .
$$
%
Here $\state : (\text{Locals} \to \Z) \times (\text{Globals} \to \Z)$
is again a simplified program state as introduced in \pref{sec:sem},
$|\inter a b| = max(0,b-a)$, and $q_i \in \Qplusz$.  To simplify the
references to the linear coefficients $q_i$, we introduce an \emph{index
  set} $I$.  This set is defined to be $\{0\} \cup \{(x, y) \mid x, y
\in \Var \land x \neq y \}$.  Each index $i$ corresponds to a \emph{base function}
$f_i$ in the potential function: $0$ corresponds to the constant
function $\state \mapsto 1$, and $(x,y)$ corresponds to $\state \mapsto
|\inter {\state(x)} {\state(y)}|$.  Using these notations we can
rewrite the above equality as\iffull{
$$
  \Phi = \sum_{i \in I} q_i f_i.
$$}
{ $\Phi = \sum_{i \in I} q_i f_i$.}  We often write $xy$ to denote the
index $(x,y)$.  \iffull{The family $(f_i)_{i\in I}$ is actually a basis (in
the linear-algebra sense) for potential functions.}{}  This allows us to
uniquely represent any linear potential function
$\Phi$ as a \emph{quantitative annotation} $Q = (q_i)_{i \in I}$, that
is, a family of \ifshort{non-negative}{} rational numbers where only
a finite number of elements are not zero. \iffull{ For soundness, we
  require that potential functions always give non-negative results.
  This is acheived by restricting coefficients in a quantitative
  annotation to be always non-negative.}{}

In the potential functions, we treat constants as global variables
that cannot be assigned to.  For example, if the program contains the
constant $8128$ then we have a variable $c_{8128}$ and
$\state(c_{8128}) = 8128$.  We assume that every program state
includes the constant $c_0$.

\paragraph{Logical State.}

In addition to the quantitative annotations our automatic amortized
analysis needs to maintain a minimal logical state to justify certain
operations on quantitative annotations.  For example when analyzing
the code $x \gets x + y$, it is helpful to know the sign of $y$ to
determine which intervals will increase or decrease.  The knowledge
needed by our rules can be inferred by local reasoning (i.e., in
basic blocks without recursion and loops) within usual theories
(e.g. Presburger arithmetic or bit vectors).

In contrast to the quantitative annotations, logical contexts $\Gamma$
in the presented inference system are left abstract.  This allows for
simpler rules and leaves room for future improvements.
%
Our implementation uses conjunctions of linear inequalities as
logical state.  We never compute fixpoints and take $\top$
as pre- and postconditions for functions. It has proved to be
sufficient for the variety of examples we are interested in or found
in the literature.


\iffull{
\begin{figure*}[t]
\begin{mathpar}
%
\Rule[leftskip=.4cm,rightskip=.3cm]{Q:Skip}
{ }
{ \!\! B; R; (\Gamma, Q) \vdash \code{skip} \dashv (\Gamma, Q) \!!}
%
\and
\Rule{Q:Break}
{ }
{\!\!  (\Gamma, Q_B); R; (\Gamma, Q_B {+} M_b) \vdash \code{break} \dashv (\Gamma', Q') \!\!}
%
\and
\Rule[leftskip=.3cm,rightskip=.4cm]{Q:Tick}
{ }
{ \!\! B; R; (\Gamma, Q{+}M_t(n)) \vdash \code{tick}(n) \dashv (\Gamma, Q) \!\!}
%
\and
\Rule{Q:Return}
{ P = Q_R[\Vret/x]
\\ \Gamma = \Gamma_R[\Vret/x]
\\ \forall i \in \dom{P} . \, p_i = q_i
}
{ B; (\Gamma_R,Q_R); (\Gamma, Q) \vdash \code{return}~x \dashv (\Gamma',Q') }
%
\and \Rule{Q:Update}
{ q'_{xy}, q'_{yx} \in \Qplusz
\\ \forall u. (q_{yu} = q'_{xu} + q'_{yu} \land q_{uy} = q'_{ux} + q'_{uy})
}
{ B; R; (\Gamma[x/y], Q {+} M_u {+} M_e(y)) \vdash x \gets y \dashv (\Gamma, Q') }
%
\\ \Rule[leftskip=.1cm,rightskip=.1cm]{Q:Loop}
{ (\Gamma', Q'); R; (\Gamma,Q) \vdash S \dashv (\Gamma,  Q {+} M_l) }
{ B; R; (\Gamma, Q) \vdash \code{loop}~S \dashv (\Gamma', Q') }
%
\and \Rule[leftskip=.2cm,rightskip=.1cm]{Q:IncP}
{ \Gamma \models y \ge 0 \!
\\ \mathcal U = \left\{ u \mid \Gamma \models x + y \in \inter x u \right\} \!
\\ \textstyle q'_{0y} = q_{0y}
      + \sum_{u \in \mathcal U} q_{xu}
      - \sum_{v \not\in \mathcal U} q_{vx} \!
}
{ \textstyle
  B; R; (\Gamma[x/x {+} y], Q {+} M_u {+} M_e(x {+} y)) \vdash x \gets x + y \dashv (\Gamma, Q')
}
%
\and \Rule{Q:DecP}
{ \Gamma \models y \ge 0
\\ \mathcal U = \left\{ u \mid \Gamma \models x - y \in \inter u x \right\}
\\\\ \textstyle q'_{y0} = q_{y0}
      + \sum_{u \in \mathcal U} q_{ux}
      - \sum_{v \not\in \mathcal U} q_{xv}
}
{ \textstyle
  B; R; (\Gamma[x/x {-} y],Q {+} M_u {+} M_e(x {-} y)) \vdash x \gets x - y \dashv (\Gamma, Q')
}
%
\and \Rule{Q:Inc}
{ M = M_u + M_e(x {\pm} y)
\\\\ \textstyle q'_{0y} = q_{0y} - \sum_{v} q_{vx}
\\ \textstyle q'_{y0} = q_{y0} - \sum_{v} q_{xv}
}
{ \textstyle
  B; R; (\Gamma[x/x {\pm} y], Q {+} M) \vdash x \gets x \pm y \dashv (\Gamma, Q')
}
%
\and \Rule{Q:If}
{ B; R; (\Gamma \land e, Q {-} M_c^1) \vdash S_1 \dashv (\Gamma',Q')
\\\\ B; R; (\Gamma \land \neg e, Q {-} M_c^2) \vdash S_2 \dashv (\Gamma',Q')
}
{ B; R; (\Gamma, Q{+}M_e(e)) \vdash \code{if}(e)~S_1~\code{else}~S_2 \dashv (\Gamma',Q') }
%
\and\Rule{Q:Seq}
{  B; R; (\Gamma, Q) \vdash S_1 \dashv (\Gamma', Q' {+} M_s)
\\ B; R; (\Gamma', Q') \vdash S_2 \dashv (\Gamma'',Q'')
}
{ B; R; (\Gamma, Q) \vdash S_1;S_2 \dashv (\Gamma'',Q'') }
%
\and\Rule{Q:Call}
{ (\Gamma_f, Q_f,  \Gamma_f', Q_f') \in \Delta(f) \!
\\ \text{Loc} = \text{Locals}(Q) \!
\\ \forall i \neq j . \, x_i \neq x_j \!
\\ c \in \Qplusz \!
\\ Q = P + S \!
\\ Q' = P' + S
\\ U = Q_f[\Vargs / \vec x]
\\ U' = Q'_f[\Vret/r]
\\ \forall i \in \dom{U} . \, p_i = u_i
\\ \forall i \in \dom{U'} .\, p'_i = u'_i
\\ \forall i \not\in \dom{U'} .\, p'_i = 0
\\ \forall i \not\in \text{Loc}
  .\, s_i = 0
}
{ B; R; (\Gamma_f[\Vargs / \vec x] \land \Gamma_{\text{Loc}}, Q {+} c {+} M_f)
  \vdash r \gets f(\vec x) \dashv
  (\Gamma_f'[\Vret/r] \land \Gamma_{\text{Loc}}, Q' {+} c {-} M_r)
}
%
\and
\Rule[leftskip=.2cm,rightskip=.2cm]{Q:Assert}
{ }
{ B; R; (\Gamma, Q {+} M_a) \vdash \code{assert}~e \dashv (\Gamma \land e, Q) }
%
\and
\Rule[leftskip=.2cm,rightskip=.2cm]{Q:Extend}
{ \fenv f = (\vec y, S_f)
\\\\ B; (\Gamma_f', Q'_f); (\Gamma_f[\Vargs/\vec y], Q_f[\Vargs/ \vec y] ) \vdash S_f \dashv (\Gamma', Q')
}
{ (\Gamma_f, Q_f, \Gamma_f', Q_f') \in \Delta(f) }
%
\and \Rule{Q:Weak}
{ B; R; (\Gamma_2,Q_2) \vdash S \dashv (\Gamma'_2, Q'_2)
\\ \!\! \Gamma_1 \models \Gamma_2
\\\\ Q_1 \succeq_{\Gamma_1} Q_2
\\ \Gamma'_2 \models \Gamma'_1
\\ Q'_2 \succeq_{\Gamma_2'} Q'_1
}
{ B; R; (\Gamma_1,Q_1) \vdash S \dashv (\Gamma'_1, Q'_1) }
%
\and \Rule{Relax}
{ \mathcal L = \{ xy \mid \exists l_{xy} {\in} \N \, . \, \Gamma \models l_{xy} \le |\inter x y| \}
\\ \mathcal U = \{ xy \mid \exists u_{xy}{\in} \N \, . \, \Gamma \models |\inter x y| \le u_{xy} \}
\\\\ \forall i \in \mathcal U.\, q'_i \ge q_i - r_i
\\ \forall i \in \mathcal L.\, q'_i \ge q_i + p_i
\\ \forall i \not\in \mathcal U {\cup} \mathcal L {\cup} \{0\} .\, q'_i \ge q_i
\\ q'_0 \geq q_0 {\textstyle + \sum_{i \in \mathcal U} u_i r_i - \sum_{i \in \mathcal L} l_i p_i}
}
{ Q' \succeq_\Gamma Q }

\end{mathpar}
\caption{\iffull{Inference rules}{Selected inference rules} of the quantitative analysis.}
\label{fig:auto}
\end{figure*}
}{}



\ifshort{
\begin{figure*}[t]
\small
\begin{mathpar}
\iffull{
%
\Rule[leftskip=.4cm,rightskip=.3cm]{Q:Skip}
{ }
{ \!\! B; R; (\Gamma, Q) \vdash \code{skip} \dashv (\Gamma, Q) \!!}
%
\and
\Rule{Q:Break}
{ }
{\!\!  (\Gamma, Q_B); R; (\Gamma, Q_B {+} M_b) \vdash \code{break} \dashv (\Gamma', Q') \!\!}
%
\and
\Rule[leftskip=.3cm,rightskip=.4cm]{Q:Tick}
{ }
{ \!\! B; R; (\Gamma, Q{+}M_t(n)) \vdash \code{tick}(n) \dashv (\Gamma, Q) \!\!}
%
\and }{}
\Rule[leftskip=.2cm,rightskip=.1cm]{Q:Return}
{ P = Q_R[\Vret/x]
\\ \forall i \in \dom{P} . \, p_i = q_i
}
{ B; (\Gamma_R,Q_R); (\Gamma_R[\Vret/x], Q) \vdash \code{return}~x \dashv (\Gamma',Q') }
%
\and \Rule[leftskip=.1cm,rightskip=.2cm]{Q:Update}
{ q'_{xy}, q'_{yx} \in \Qplusz
\\ \forall u. (q_{yu} = q'_{xu} + q'_{yu} \land q_{uy} = q'_{ux} + q'_{uy})
}
{ B; R; (\Gamma[x/y], Q {+} M_u {+} M_e(y)) \vdash x \gets y \dashv (\Gamma, Q') }
%
\\ \Rule[leftskip=.1cm,rightskip=.1cm]{Q:Loop}
{ (\Gamma', Q'); R; (\Gamma,Q) \vdash S \dashv (\Gamma,  Q {+} M_l) }
{ B; R; (\Gamma, Q) \vdash \code{loop}~S \dashv (\Gamma', Q') }
%
\and \Rule[leftskip=.2cm,rightskip=.1cm]{Q:IncP}
{ \Gamma \models y \ge 0 \!
\\ \mathcal U = \left\{ u \mid \Gamma \models x + y \in \inter x u \right\} \!
\\ \textstyle q'_{0y} = q_{0y}
      + \sum_{u \in \mathcal U} q_{xu}
      - \sum_{v \not\in \mathcal U} q_{vx} \!
}
{ \textstyle
  B; R; (\Gamma[x/x {+} y], Q {+} M_u {+} M_e(x {+} y)) \vdash x \gets x + y \dashv (\Gamma, Q')
}
%
\and \Rule{Q:DecP}
{ \Gamma \models y \ge 0
\\ \mathcal U = \left\{ u \mid \Gamma \models x - y \in \inter u x \right\}
\\\\ \textstyle q'_{y0} = q_{y0}
      + \sum_{u \in \mathcal U} q_{ux}
      - \sum_{v \not\in \mathcal U} q_{xv}
}
{ \textstyle
  B; R; (\Gamma[x/x {-} y],Q {+} M_u {+} M_e(x {-} y)) \vdash x \gets x - y \dashv (\Gamma, Q')
}
%
\and \Rule{Q:Inc}
{ M = M_u + M_e(x {\pm} y)
\\\\ \textstyle q'_{0y} = q_{0y} - \sum_{v} q_{vx}
\\ \textstyle q'_{y0} = q_{y0} - \sum_{v} q_{xv}
}
{ \textstyle
  B; R; (\Gamma[x/x {\pm} y], Q {+} M) \vdash x \gets x \pm y \dashv (\Gamma, Q')
}
%
\and \Rule{Q:If}
{ B; R; (\Gamma \land e, Q {-} M_c^1) \vdash S_1 \dashv (\Gamma',Q')
\\\\ B; R; (\Gamma \land \neg e, Q {-} M_c^2) \vdash S_2 \dashv (\Gamma',Q')
}
{ B; R; (\Gamma, Q{+}M_e(e)) \vdash \code{if}(e)~S_1~\code{else}~S_2 \dashv (\Gamma',Q') }
%
\and\Rule{Q:Seq}
{  B; R; (\Gamma, Q) \vdash S_1 \dashv (\Gamma', Q' {+} M_s)
\\\\ B; R; (\Gamma', Q') \vdash S_2 \dashv (\Gamma'',Q'')
}
{ B; R; (\Gamma, Q) \vdash S_1;S_2 \dashv (\Gamma'',Q'') }
%
\and\Rule{Q:Call}
{ (\Gamma_f, Q_f,  \Gamma_f', Q_f') \in \Delta(f) \!
\\ \text{Loc} = \text{Locals}(Q) \!
\\ \forall i \neq j . \, x_i \neq x_j \!
\\ c \in \Qplusz \!
\\ Q = P + S \!
\\ Q' = P' + S
\\ U = Q_f[\Vargs / \vec x]
\\ U' = Q'_f[\Vret/r]
\\ \forall i \in \dom{U} . \, p_i = u_i
\\ \forall i \in \dom{U'} .\, p'_i = u'_i
\\ \forall i \not\in \dom{U'} .\, p'_i = 0
\\ \forall i \not\in \text{Loc}
  .\, s_i = 0
}
{ B; R; (\Gamma_f[\Vargs / \vec x] \land \Gamma_{\text{Loc}}, Q {+} c {+} M_f)
  \vdash r \gets f(\vec x) \dashv
  (\Gamma_f'[\Vret/r] \land \Gamma_{\text{Loc}}, Q' {+} c {-} M_r)
}
%
\iffull{\and
\Rule[leftskip=.2cm,rightskip=.2cm]{Q:Assert}
{ }
{ B; R; (\Gamma, Q {+} M_a) \vdash \code{assert}~e \dashv (\Gamma \land e, Q) }
}{}
%
\and
\Rule[leftskip=.2cm,rightskip=.2cm]{Q:Extend}
{ \fenv f = (\vec y, S_f)
\\\\ B; (\Gamma_f', Q'_f); (\Gamma_f[\Vargs/\vec y], Q_f[\Vargs/ \vec y] ) \vdash S_f \dashv (\Gamma', Q')
}
{ (\Gamma_f, Q_f, \Gamma_f', Q_f') \in \Delta(f) }
%
\and \Rule{Q:Weak}
{ B; R; (\Gamma_2,Q_2) \vdash S \dashv (\Gamma'_2, Q'_2)
\\ \!\! \Gamma_1 \models \Gamma_2
\\\\ Q_1 \succeq_{\Gamma_1} Q_2
\\ \Gamma'_2 \models \Gamma'_1
\\ Q'_2 \succeq_{\Gamma_2'} Q'_1
}
{ B; R; (\Gamma_1,Q_1) \vdash S \dashv (\Gamma'_1, Q'_1) }
%
\and \Rule{Relax}
{ \mathcal L = \{ xy \mid \exists l_{xy} {\in} \N \, . \, \Gamma \models l_{xy} \le |\inter x y| \}
\\ \mathcal U = \{ xy \mid \exists u_{xy}{\in} \N \, . \, \Gamma \models |\inter x y| \le u_{xy} \}
\\\\ \forall i \in \mathcal U.\, q'_i \ge q_i - r_i
\\ \forall i \in \mathcal L.\, q'_i \ge q_i + p_i
\\ \forall i \not\in \mathcal U {\cup} \mathcal L {\cup} \{0\} .\, q'_i \ge q_i
\\ q'_0 \geq q_0 {\textstyle + \sum_{i \in \mathcal U} u_i r_i - \sum_{i \in \mathcal L} l_i p_i}
}
{ Q' \succeq_\Gamma Q }

\end{mathpar}
\caption{\iffull{Inference rules}{Selected inference rules} of the quantitative analysis.}
\label{fig:auto}
\end{figure*}
}{}

\paragraph{Judgements of the Automatic Analysis.}

The inference system for the automatic amortized analysis is defined
in~\pref{fig:auto}.  The inference rules derive judgements of the form
$$
  (\Gamma_B, Q_B); (\Gamma_R, Q_R);
    (\Gamma, Q) \vdash S \dashv (\Gamma', Q').
$$
These judgements correspond to the logic judgements $\Delta; B; R
\vdash \htriple P S Q$ where each assertion splits in two parts, a
logical part $\Gamma$ and a quantitative part $Q$.  That means that
$(\Gamma_B, Q_B)$ is the postcondition of \code{break} statements,
$(\Gamma_R, Q_R)$ is the postcondition for return statements, and
$(\Gamma, Q) \vdash S \dashv (\Gamma', Q')$ can be understood as a
simple Hoare triple.  We leave out the function context $\Delta$ in
the presentation to avoid cluttering notations.  Instead we assume a
fixed function context $\Delta$ that is implicit on all judgements.
Function contexts are treated exactly as in the logic.  % We call
% precondition the pair $(\Gamma, Q)$ and postcondition the pair
% $(\Gamma', Q')$.
The correspondence between judgements of the
automation and the ones of the quantitative logic is made formal by
our soundness proof of the automatic amortized analysis at the end of
this section.

As a convention, if $Q$ and $Q'$ are quantitative annotations then we
assume that $Q = (q_i)_{i\in I}$ is a family with elements $q_i$, $Q'
= (q'_i)_{i \in I}$, etc.  The notation $Q \pm n$ used in many rules
defines a new context $Q'$ such that $q'_0 = q_0 \pm n$ and $\forall i
\neq 0 .\, q'_i = q_i$.  We have the implicit side condition that all
rational coefficients are non-negative.  Finally, if a rule mentions
$Q$ and $Q'$ and leaves the latter undefined at some index $i$ we will
implicitly assume that $q'_i = q_i$.

We describe the automatic amortized analysis for a subset of
expressions of Clight.  Assignments must have the form $x \gets
y$ or $x \gets x \pm y$.  In the implementation, a Clight
program is converted into this form prior to analysis without changing
the resource cost.  This achieved by using a series of \emph{cost-free
  assignments} that do not result in additional cost in the
analysis.  Non-linear operations such as $x \gets z*y$ are simply
handled by assigning zero potential to coefficients like $q_{xa}$ and
$q_{ax}$ that contain $x$ after the assignment.

\paragraph{Inference Rules.}

Most of the rules correspond exactly to the respective rules of the
quantitative Hoare logic.
\iffull{
  For example, the rule {\sc Q:Skip} simply reuses its postcondition
  as precondition.  This is sound since evaluating a $\code{skip}$
  statement never costs any resources and does not change the program
  state.  Similarly, the rule {\sc Q:Tick} enforces that the
  quantitative annotation of the precondition is at least $M_t(n)$ to
  pay for the resource consumption.  Since the program state is also
  unchanged in this case, the logical part $\Gamma$ of the
  precondition can soundly be reused in the postcondition.  In the
  rule {\sc Q:Break}, where the linear control flow is broken, the
  postcondition $(\Gamma', Q')$ can be arbitrary.  The precondition is
  taken from the break part of the judgement similarly as it is done
  in the quantitative logic.
}
%
{ Examples include {\sc Q:Skip, Q:Tick, Q:Break, Q:Seq, Q:Assert}, and
  {\sc Q:Loop}.  We only show {\sc Q:Seq} and {\sc Q:Loop}.  The other
  rules can be found in the extended version.
}
%
The rule {\sc Q:Return} exhibits a slight
difference to the quantitative logic.  While we have a function $\Z
\to \Assn$ to represent return values in the logic, we assume that a
special variable $\ret$ is part of the index set of $Q_R$.  The
substitution $Q[\ret/x]$ denotes an potential annotation $Q'$ with
$q'_{xy} = q_{\ret y}$ and $q'_{yx} = q_{y \ret}$ for all $y$ and
$q_i'=q_i$ otherwise.

Most interesting are the rules {\sc Q:IncP}, {\sc Q:DecP}, and {\sc
  Q:Inc} for increments and decrements, which describe how the
potential is distributed after a size change of a variable.  The rule
{\sc Q:IncP} is for increments $x \gets x + y$ and {\sc Q:DecP} is
for decrements $x \gets x - y$, they both apply only when we can deduce
from the logical context $\Gamma$ that $y \ge 0$.
Of course, we have symmetrical rules {\sc
  Q:IncN} and {\sc Q:DecN} that can be applied if $y$ is not positive.
The rules are equivalent in the case where $y=0$.  Finally, the rule
{\sc Q:Inc} can be applied if we cannot deduce any information about
$y$.  In the implementation, the rules for increments and decrements
are combined into one syntax directed rule where the specialized rules
take priority over {\sc Q:Inc}.

To explain how rules for increment and decrement work, it is
sufficient to understand the rule {\sc Q:IncP}.  The others follow the
same idea and are symmetrical.  In {\sc Q:IncP}, the program updates a
variable $x$ with $x+y$ where $y \ge 0$.  Since $x$ is changed, the
quantitative annotation must be updated to reflect the change of the
program state.  We write $x'$ for the value of $x$ after the assignment.
Since $x$ is the only variable changed, only intervals of the form
$\inter u x$ and $\inter x u$ will be resized.  Note that for any $u$,
$\inter x u$ will get smaller with the update, and if $x' \in \inter x
u$ we have $|\inter x u| = |\inter x {x'}| + |\inter {x'} u|$.  But
$|\inter x {x'}| = |\inter 0 y|$ which means that the potential
$q'_{0y}$ in the postcondition can be increased by $q_{xu}$ under the
guard that $x' \in \inter x u$.  Dually, the interval $\inter v x$ can
get bigger with the update.  We know that $|\inter v {x'}| \le y +
|\inter v x|$.  So we decrease the potential of $\inter 0 y$ by
$q_{vx}$ to pay for this change.  The rule ensures this
only for $v \not\in \mathcal U$ because we know that $x \le v$
otherwise, and thus $|\inter v x| = 0$.

Another interesting rule is {\sc Q:Call}.  It needs to account for the
changes to the stack caused by the function call, the arguments/return
value passing, and the preservation of local variables.  We can sum up
the main ideas of the rule as follows.
\itemskip
\begin{itemize}
\item The potential in the pre- and postcondition of the function
  specification is equalized to its matching potential in the callee's
  pre- and postcondition.
\itemskipIn
\item
  $\!$The potential of intervals $|\inter x y|$ is preserved
  if $x$ and $y$ are local\iffull{ variables.}{.}
\item
  The unknown potentials after the call (e.g.
  $|\inter x g|$, with $x$ local and $g$ global)
  are set to zero in the postcondition.
\end{itemize}
\itemskip
If $x$ and $y$ are local variables and $f(x,y)$ is
called, {\sc Q:Call} splits the potential of
$|\inter x y|$ in two parts, one part to perform the
computation in the function $f$ and one part to keep for
later use after the function call.  This splitting is
realized by the equations $Q = P {+} S$ and $Q' =
P' {+} S'$.  Arguments in the function precondition
$(\Gamma_f, Q_f)$ are named using a fixed vector $\Vargs$
of names different from all program variables.  This
prevents name conflicts from happening and ensures that
the substitution $[\Vargs/\vec x]$ is meaningful.
Symmetrically, we use the unique name $\Vret$ to represent
the return value in the function's postcondition
$(\Gamma'_f, Q'_f)$.

The rule {\sc Q:Weak} is not syntax directed.  In the implementation
we apply {\sc Q:Weak} before loops and between the two statements of a
sequential composition.  \iffull{We could integrate weakening into every
syntax directed rule but this simple heuristic helps to make the
analysis efficient.}{}  The high-level idea of {\sc Q:Weak} is the same
as in the rule {\sc L:Weak} of the quantitative logic: If we have a
sound judgement, then it is sound to add more potential to the
precondition and remove potential from the postcondition.  The concept
of \emph{more potential} is formalized by the relation $Q'
\succeq_\Gamma Q$ that is defined in the rule {\sc Relax}.  Here, $Q$
and $Q'$ are potential annotations and $\Gamma$ is a logical context.
The rule {\sc Relax} deals also with the important task of transferring
constant potential (represented by $q_0$) to interval sizes and vice
versa.  If we can deduce from the logical context that the interval
size $|\inter x y| \geq l$ is larger then a constant $l$ then we can
transfer potential $q_{xy} {\cdot} |\inter x y|$ form the interval to
constant potential $l{\cdot}q_{xy}$ and guarantee that we do not gain
potential.  Conversely, if $|\inter x y| \leq u$ for a constant $u$
then we can transfer constant potential $u{\cdot}q_{xy}$ to the
interval potential $q_{xy} {\cdot} |\inter x y|$ without gaining
potential.

\paragraph{Automatic Inference via LP Solving.}

We separate the search of a derivation into two steps.  As a first
step we go through the whole program and apply inductively the
inference rules of the automatic amortized analysis.  During this
process our tool uses symbolic names for the rational coefficients
$(q_i)$ in the rules.  Every time a linear constraint must be
satisfied by these coefficients, it is recorded in a global list using
the symbolic names.  We then feed the collected constraints to an
off the shelf LP solver~\footnote{We currently use Coin-Or's CLP.}.
If the solver successfully finds a solution, we know that a derivation
exists for the considered program.  We can then extract the initial
$Q$ from the solver and get a resource bound for the program.  To get
a full derivation we simply extract the complete solution from the
solver, and apply it to the symbolic names $(q_i)$ of the coefficients
in the derivation.  If the LP solver fails to find a solution, an
error is reported to the user.


\begin{figure*}
\centering
\begin{tabular}{r|c|c|c|c|c}
%
& t08 (modified t08a) & t19 & t30 & t15 & t13\\
%
\hline
&
\begin{lstlisting}[basicstyle=\tt\scriptsize]
while (y > x) {
  x++; tick(1);
}
while (x > 2) {
  x -= 3; tick(1);
}
\end{lstlisting}
&
\begin{lstlisting}[basicstyle=\tt\scriptsize]
while (i > 100) {
  i--; tick(1) }
i += k+50;
while (i >= 0) {
  i--; tick(1);
}
\end{lstlisting}
&
\begin{lstlisting}[basicstyle=\tt\scriptsize]
while (x > 0) {
  x--;
  t=x, x=y, y=t;
  tick(1);
}
\end{lstlisting}
&
\begin{lstlisting}[basicstyle=\tt\scriptsize]
assert(y >= 0);
while (x > y) {
  x -= y+1;
  for (z=y; z > 0; z--)
    tick(1);
  tick(1);
}
\end{lstlisting}
&
\begin{lstlisting}[basicstyle=\tt\scriptsize]
  while (x > 0) {
    x--;
    if (*) y++;
    else
      while (y > 0) {
        y--; tick(1); }
    tick(1); }
\end{lstlisting}
%
\\
\hline
%
%
%
AAA
  & {\small $1.33|[x,y]| {+} 0.33 |[0,x]|$ }
  & {\small $50 {+} |\inter {-1} i| {+} |\inter 0 k|$ }
  & {\small $|\inter 0 x| {+} |\inter 0 y|$}
  & {\small $|\inter 0 x|$}
  & {\small $2 |\inter 0 x| {+} |\inter 0 y|$}
\\
%
Rank &
{\small $2 {+} y {-} x (?)$ } &
{\small $54+k+i$ } &
{\small ---} &
{\small $2+2x-y$ } &
{\small $0.5{\cdot}y^2 {+} yx +0.5{\cdot}x^2\ldots$} % + 1{-}0.5{\cdot}y){-}0.5{\cdot}x $}
\\
%
LOOPUS &
{\small
  \begin{tabular}{c}
    $\max(0, x{-}2)$\\ $+ 2\max(0, y {-} x)$
  \end{tabular}
} &
{\small
  \begin{tabular}{c}
    $\max(0, i{-}100)$ \\
    $+ \max(0, k {+} i {+} 51)$
  \end{tabular}
 } &
{\small ---}% $1 + \max(x, y)$ (???)
&
{\small ---}
&
{\small $2\max(x, 0) {+} \max(y, 0)$}
\end{tabular}
\vspace{.1cm}
\caption{ Comparison of resource bounds derived by different tools on
  several examples with linear bounds.  AAA stands for our automatic
  amortized analysis for Clight.  The output of Rank has been manually
  simplified to fit the table.  }
\label{fig:compar}
\end{figure*}

\paragraph{Soundness Proof.}

The soundness of the analysis builds on the quantitative logic.  We
translate judgements of the automatic amortized analysis as defined in
\pref{fig:auto} into quantitative Hoare triples.  We define a
translation function $\tr$, such that if a judgement $J$ in the
automatic analysis is derivable, $\tr(J)$ is derivable in the
quantitative logic.  By using $\tr$ to translate derivations of the
automatic analysis to derivations in the quantitative logic we can
automatically obtain a certified resource bound for the analyzed
program.

The translation of an assertion $(\Gamma, Q)$ in the automatic
analysis is defined by
$$
  \tr(\Gamma, Q) := \lambda\state .\, (\Gamma(\state)) + \Phi_Q(\state),
$$
where we write $\Phi_Q$ for the unique linear potential function
defined by the quantitative annotation $Q$.  We also need to
translate the assumptions in function contexts of the automatic
analysis.  We define $\tr(\Gamma_f, Q_f, \Gamma'_f, Q'_f) := \forall z \, \vec v \, v . $
%
$$
(
  \lambda \_ \, \vec v \, \state .\, \tr(\Gamma_f, Q_f)(\state[\Vargs {\mapsto} \vec v]),
  \lambda \_ \, v \, \state .\, \tr(\Gamma'_f, Q'_f)(\state[\Vret {\mapsto} v])
  )
$$
%
These definitions let us translate the judgement $J = B; R; P \vdash
S \dashv P'$ in the context $\Delta$ by
$$
  \tr(J) :=
  (\lambda f .\, \tr(\Delta(f))); \tr(B); \tr(R) \vdash
    \htriple{\tr(P)}{S}{\tr(P')}.
$$
The soundness of the automatic analysis can now be stated
formally with the following theorem.
%
\begin{theorem}[Soundness of the automatic analysis]
  If $J$ is a judgement derived with the rules of
  \pref{fig:auto}, then $\tr(J)$ is a quantitative
  Hoare triple derivable with the rules of
  \pref{fig:logic}.
\end{theorem}
%
\noindent
The proof of this theorem is constructive and basically maps each rule
of the automatic analysis directly to its counterpart in the
quantitative logic.  The most tricky parts are the translations of the
rules for increments and decrements and the rule {\sc Q:Weak} for
weakening because we have to show that the preconditions of the rules
{\sc L:Weak} or {\sc L:Update}, respectively, are met.  \iffull{ For instance
we prove the following lemma to show the soundness of the translation
of {\sc Q:Weak}.

\begin{lemma}[Relax]
  If $Q' \succeq_\Gamma Q$ and $\state$ is a program state such that
  $\state \models \Gamma$ then we have $\Phi_{Q'}(\state) \ge \Phi_Q(\state)$.
\end{lemma}
\begin{proof}
We observe the following inequations.
\begin{align*}
\Phi_{Q'}(\state) &= \sum_i q'_i f_i(\state) \\
&\ge q_0 + \sum_{i\in\mathcal U} u_i p_i - \sum_{i\in\mathcal L} l_i p_i
  + \sum_{i\in\mathcal U} (q_i - p_i) f_i(\state)  \\ &
  \;\;\;\;\;\;\;\; + \sum_{i\in\mathcal L} (q_i + p_i) f_i(\state)
  + \sum_{i\not\in \mathcal U \cup \mathcal L} q_i \\
 &\ge \Phi_Q(\state)
  + \sum_{i\in\mathcal U} (u_i - f_i(\state)) p_i
  + \sum_{i\in\mathcal L} (f_i(\state) - l_i) p_i \\
 &\ge \Phi_Q(\state).
\end{align*}
Since $\state \models \Gamma$, using the transitivity of $\models$ and
the definition of $\mathcal U$ and $\mathcal L$ we get
$\forall i\in\mathcal L.\, l_i \le f_i(\state)$ and
$\forall i\in\mathcal U.\, f_i(\state)\le u_i$.
These two inequalities justify the last step.
\end{proof}
%
\noindent
We prove similar lemmas for the rules that deal with assignment.  With
the computational content of the proof, we have an effective algorithm
to construct certificates for upper bounds derived automatically.
}{}


% \sectskip
% \section{Implementation}

% \subsection{Logical Annotations Generation}

% \paragraph{Program conditions.}
% The first pass in our OCaml implementation generates the
% logical annotations $\Gamma$ around all program statements.
% These annotations are used later on by the quantitative
% analysis to generate constraints.
% %
% The first design choice we had to make is to give a fixed form
% to program conditions.  Because of their generality and because
% they are closed under negation, we decided to consider
% inequalities over linear combinations as program conditions.
% \begin{mathpar}
% C := L < L \mid L > L \mid L \le L \mid L \ge L
% \and
% L := x \mid k \mid k * L \mid L + L \mid L - L
% \end{mathpar}
% A Condition is represented internally as a pair of one constant
% $k \in \mathbb N$ and one map $m : V \rightarrow \mathbb N$.
% Its intended semantics is
% $$
% \llbracket m \rrbracket(H) = k + \sum_v m_v \cdot H(v) \le 0.
% $$
% Note that conditions of this shape are closed by negation, as showed
% by the following equivalence
% $$
% \neg (k + \sum_v m_v \cdot H(v) \le 0)
% \Leftrightarrow (1-k) + \sum_v -m_v \cdot H(v) \le 0
% $$
% We can also note that all the ``questions'' asked to the logic
% assertions during a proof derivation can be described using
% linear combinations.  Conveniently enough, this language of
% assertions is exactly Presburger's arithmetic.  This means that
% we have a sound and complete decision procedure to answer
% all the questions asked during the analysis.

% \paragraph{Logical contexts.}
% The second problem related to the logic is how to figure out
% the logical contexts $\Gamma$ for each program statement.
% This is a question that is answered by abstract interpretation
% but, to show that our work really does not need the full power
% of abstract interpretation, we designed and implemented a very
% simple top-down approach: starting from a statement $S$ and
% a pre-condition $\Gamma$ (given as a list of inequalities), we
% generate $\mathcal P(S, \Gamma)$ a postcondition for $S$ using the
% following equations:

% \begin{align*}
% \mathcal P(\code{skip}, \Gamma) &:= \Gamma \\
% \mathcal P(x \gets y, \Gamma) &:= \{ e \le 0 \mid (e \le 0) \in \Gamma \land \{x\} \# e \} \\
% \mathcal P(x \gets x + y, \Gamma) &:= \Gamma[x / x - y] \\
% \mathcal P(x \gets x - y, \Gamma) &:= \Gamma[x / x + y] \\
% \mathcal P(\code{assert}(C), \Gamma) &:= \Gamma,C \\
% \mathcal P(S_1; S_2, \Gamma) &:= \mathcal P(S_2, \mathcal P(S_1, \Gamma)) \\
% \mathcal P(\code{if} (C)~S_1~\code{else}~S_2) &:=
%   \mathcal P(S_1, \Gamma,C) \lor \mathcal P(S_2, \Gamma,\neg C) \\
% \mathcal P(\code{while}(C)~S, \Gamma) &:=
%   \Gamma \lor (\neg C, \mathcal P(S, (C, \{ e \le 0 \mid (e \le 0) \in \Gamma \land {\rm set}(S) \# e \})))
% \end{align*}
% We used the notation ${\rm set}(S)$ to designate the set of variables
% assigned to in the statement $S$ and $s\#e$ to mean that variables
% in $s$ do not appear in the linear combination $e$.  Because contexts
% are simply conjunction of conditions we must define $\Gamma \lor \Delta$
% as a function on contexts:
% $$
% \Gamma \lor \Delta := \{ e \le 0 \mid (e \le 0) \in \Gamma \cup \Delta \land \Gamma \models (e \le 0) \land \Delta \models (e \le 0) \}.
% $$
% It is important to note that the $\mathcal P$ function defined above
% is not performing any fixpoint computation.  Despite this flagrant
% lack of sophistication our complete analysis is still able to figure out some
% tricky invariants as exemplified earlier.
% Finally, to generate annotations for the whole program $S$,
% we run $\mathcal P(S,\bullet)$ (where $\bullet$ is the empty context)
% and memorize all the intermediate results.  In practice, these results are stored in
% a table that can be referenced at any time during the analysis to get the
% pre and post-condition of any statement in the program.

% \paragraph{Decision procedure.}
% To decide logical entailment $\Gamma \models \Delta$ we use
% Presburger's decision procedure.  But what this procedure gives us
% is in fact only an answer the the question: Is there any heap $H$
% such that $H \models \Gamma$.  While this sentence is existential
% the question $\Gamma \models \Delta$ is universal.  To use the
% decision procedure, we use the well known trick of double negation.
% \begin{align*}
% \Gamma \models C
% &\Leftrightarrow \neg\neg (\Gamma \models C) \\
% &\Leftrightarrow \neg\neg (\forall H.\, H\models\Gamma \implies H\models C) \\
% &\Leftrightarrow \neg \exists H, H\models\Gamma \land \neg H\models C \\
% &\Leftrightarrow \neg \exists H, H\models\Gamma \land H\models \neg C \\
% &\Leftrightarrow \neg \exists H, H\models (\Gamma, \neg C)
% \end{align*}
% Thus, to know if $\Gamma \models C$ we ask the decision
% procedure if $\Gamma, \neg C$ is satisfiable and negate the result.
% When we want to know about $\Gamma \models \Delta$ we simply ask
% one question for each of the conjuncts of $\Delta$.

% \subsection{Linear Constraints Generation}

% \paragraph{Generating a linear program.}
% Because figuring out the coefficents that are in $Q$ right away is a hard
% problem, we separate the search of a derivation in two steps.  As a first
% step we go through the whole program and apply inductively the rules
% given in the proof system section.  During this process our tool uses
% symbolic names for the coefficients $(q_i)$.  Every time a side condition
% must be satisfied by these coefficients, it is recorded using the symbolic
% names in a global list.  When
% the list of all constraints that must be satisfied by the coefficients is
% collected, we feed it to an off the shelf LP-solver.  If the solver successfully
% finds a solution to the given problem, we know that a derivation
% exists for the considered program.  We can even ask the solver
% for the initial $Q$ and get a resource bound for the program.
% If the LP-solver fails to find a solution, an error is reported to the
% user.

% \paragraph{Making the proof system algorithmic.}
% One problem with the proof system as described is that it is not
% fully syntax directed.  The rule {\sc Q:Weak} can be applied at
% all times.  However, if we apply this weakening rule at every step
% of the derivation, the number of generated constraints and
% the number of queries to the logic system explode.  This blowup
% causes efficiency problems.  To avoid these performance problems
% our implementation only applies the weakening rules at some
% strategic points in the derivation.  More specifically, at sequence
% points, at the beginning of while loops and around conditional
% statements.  So far, this selection of weakening spots has been
% driven by practice but it might be possible to prove that these
% program points are the only ones where weakening is needed.


% \sectskip
% \section{Polynomial Potential}

% \paragraph{Index Sets}

% Let $V$ be a set of variables.  An \emph{index} $I \in \ind(V)$ is a
% family that maps two-element sets of variables to natural numbers,
% that is,
% $$
% I = (i_{\{x,y\}})_{\{x,y\} \subseteq V} \; .
% $$
% %
% We identify a family $I$ with the set $\{ (\{x,y\},i_{\{x,y\}})
% \mid \{x,y\} \subseteq V\}$.

% Let $\ind(V)$ denote the set of all such indices.  We write $\ind$
% instead of $\ind(V)$ if the set of variables $V$ is fixed or obvious
% from the context.
% %
% We assume that allways $0 \in V$ and sometimes write $i_x$ instead of $i_{\{x,0\}}$.

% The \emph{degree} $\deg(I)$ of an index $I = (i_{\{x,y\}})_{\{x,y\}
%   \subseteq V}$ is defined as
% $$
% \deg(I) = \sum_{\{x,y\} \in V} i_{\{x,y\}} \;.
% $$
% We define $\ind_k(V) = \{ I \mid I \in \ind(V) \text{ and } \deg(i) \leq k
% \}$ to be the set of indices of degree at most $k$.

% \paragraph{Resource Polynomials}

% Let $V$ be a set of variables.  An index $I \in \ind(V)$ denotes a
% \emph{base polynomial} $P_I : \states \to \N$ for $V$ that maps a
% program state $H$ to product of binomial coefficients (a natural
% number).  We define
% $$
% P_I(H) = \prod_{{\{x,y\}} \subseteq V} \binom {|H(x){-}H(y)|} {i_{\{x,y\}}} \; .
% $$
% %
% A \emph{resource polynomial} $R$ for the variable set $V$ is a
% non-negative linear combination of the base polynomials for $V$.

% \paragraph{Potential Annotations}

% A \emph{potential annotation} for the variable set $V$ is a family
% $$Q = (q_I)_{I \in \ind(V)}$$
% of non-negative rational numbers.  Such an annotation denotes the
% resource polynomial $R_Q$ that is defined by
% $$
% R_Q(H) = \sum_{I \in \ind(V)} q_I \cdot P_I(H) \; .
% $$
% %
% We say that the potential annotation $Q$ is of degree $k$ if $q_I = 0$
% for $I \in \ind(V)$ with $\deg(I) > k$.

% \paragraph{Additive Shifts}

% Let $Q$ be a potential annotation for a variable set $V$ and let
% $\{x,y\} \subseteq V$ be a two-element variable set.  The
% \emph{additive shift} with respect to $\{x,y\}$ is a potential
% annotation $\shift_{\{x,y\}}(Q) = (q'_I)_{I \in \ind(V)} $ for $V$
% that is defined through
% $$
% q'_I = q_I + q_{I^{\{x,y\}{+}1}} \; .
% $$
% For an index $I = (i_{\{x,y\}})_{\{x,y\} \subseteq V}$ we use the
% notation $I^{\{x,y\}{+}k}$ to denote the index
% $(i'_{\{x,y\}})_{\{x,y\} \subseteq V}$ such that
% $$
% i'_{\{t,u\}} = \left\{
%   \begin{array}{ll}
%     i_{\{t,u\}} + k  & \text{if } \{t,u\} = \{x,y\} \\
%     i_{\{t,u\}} & \text{otherwise}
%   \end{array}
% \right.
% \;.
% $$
% %
% The additive shift for natural numbers reflects the identity
% \begin{equation}
% \label{eq:shift}
% \sum_{0 {\leq} i \leq {k}} q_i \binom{n+1}{i} = \sum_{0 {\leq} i \leq {k}} (q_i{+}q_{i+1}) \binom{n}{i}
% \end{equation}
% where $q_{k+1} = 0$.  It is used in the effect system if the
% difference $n+1$ between two variables $x,y$ decreases by one.

% \begin{lemma} Let $V$ be a set of variables with $x,y \in V$ and let
%   $H$ be a program state. Let $|H'(t) {-} H'(u)| = |H(t) {-} H(u)|$
%   for $\{t,u\} \neq \{x,y\}$ and let $|H'(x) {-} H'(y)| = |H(x) {-}
%   H(y)| - 1$.
%   %
%   If $Q' = \shift_{\{x,y\}}(Q)$ then $R_Q(H) = R_{Q'}(H')$.
% \end{lemma}

% We now study the effect of multiple simultaneous shifts.  Let $Q$ be a
% resource annotation for a variable set $V$ and let $U_1,\ldots,U_n
% \subseteq V$ with $|U_i| = 2$ for all $i$ and $U_i \neq U_j$ for $i
% \neq j$ be pairwise distinct two-element variable sets.  The
% simulations additive shift $\shift_{U_1,\ldots,U_n}(Q)$ of $Q$ with
% respect to $U_1,\ldots,U_n$ is defined by
% $$
% \shift_{U_1,\ldots,U_n}(Q) = \shift_{U_1}( \cdots \shift_{U_n}(Q) \cdots ) \; .
% $$
% %
% \begin{proposition}
%   Let $V$ be a set of variables and let $U_1,\ldots,U_n$ be pairwise
%   distinct two-element variable sets.  Let $|H'(x) {-} H'(y)| = |H(x)
%   {-} H(y)|$ for $\{x,y\} \not\in \{U_1,\ldots,U_n\}$ and let $|H'(x)
%   {-} H'(y)| = |H(x) {-} H(y)| - 1$ for $\{x,y\} \in
%   \{U_1,\ldots,U_n\}$.
%   %
%   If $Q' = \shift_{U_1,\ldots,U_n}(Q)$ then $R_Q(H) = R_{Q'}(H')$.
% \end{proposition}
% %
% As shown by the following lemma, the order in which the shifts for the
% individual $U_i$ are applied is insignificant.
% %
% \begin{lemma}
%   Let $\sigma : \{1,\ldots,n\} \to \{1,\ldots,n\}$ be a
%   permutation. Then $\shift_{U_1,\ldots,U_n}(Q) =
%   \shift_{U_{\sigma(1)},\ldots,U_{\sigma(n)}}(Q)$.
% \end{lemma}
% %
% For reasons of efficiency in the constraint generation, we give a more
% direct formula for the simultaneous shift.  Let $I \in \ind(V)$ and
% let $U_1,\ldots,U_n$ be pairwise distinct two-element variable sets.
% We define the index $I^{U_1,\ldots,U_n + k}$ as the family $(i'_{\{x,y\}})_{\{x,y\} \subseteq V}$ such that
% $$
% i'_{\{t,u\}} = \left\{
%   \begin{array}{ll}
%     i_{\{t,u\}} + k  & \text{if } \{t,u\} \in \{U_1,\ldots,U_n\} \\
%     i_{\{t,u\}} & \text{otherwise}
%   \end{array}
% \right.  \;.
% $$

% %
% \begin{lemma}
%   Let $V$ be a variable set and let $U_1,\ldots,U_n$ be pairwise
%   distinct two-element variable sets.
%   %
%   Let $Q = (q_I)_{I \in \ind(V)}$ be a resource annotation for
%   $V$ and let $ Q' = (q'_I)_{I \in \ind(V)}$ where
%   $$
%   q'_I = \sum_{\{j_1,\ldots,j_m\} \subseteq \{1,\ldots,n\} } q_{I^{U_{j_1},\ldots,U_{j_m}+1}} \; .
%   $$
%   Then $Q' = \shift_{U_1,\ldots,U_n}(Q)$.
% \end{lemma}


%%
% Note: Later we need to shift in many directions at once like
%   Q' = shift_{x,y} (shift_{x,u} (Q))
% To do: Give a combined formula for that (concise constraint system).
%%



\sectskip
\section{Experimental Evaluation}
\label{sec:exper}
\aftersectskip

We have experimentally evaluated the practicality of our automatic
amortized analysis with more than 30 challenging loop and recursion
patterns from open-source code and the
literature~\cite{GulwaniMC09,GulwaniJK09,GulwaniZ10}.  A full list of
examples \iffull{together with the derived bounds is given in
  \pref{app:cat}.}{is given in the extended
  version~\cite{anon_extended}.}

\pref{fig:compar} shows five representative loop patterns from the
evaluation.  Example \emph{t08} is a slightly modified version of
\emph{t08a}, which is described in~\pref{sec:inform}.  Example
\emph{t19} demonstrates the compositionaliy of the analysis.  The
program consists of two loops that decrement a variable $i$.  In the
first loop, $i$ is decremented down to 100 and in the second loop $i$
is decremented further down to $-1$.  However, between the loops we
assign \code{i=i+k+50}.  So in total the program performs $52 + |[-1,i]| +
|[0,k]|$ increments.  Our analysis finds this tight bound because our
amortized analysis naturally takes into account the relation between
the two loops.

At first sight, example \emph{t30} appears to be a simple loop that
decrements the variable $x$ down to zero.  However, a closer look
reveals that the loop actually decrements both input variables $x$ and
$y$ down to zero before terminating.  In the loop body, first $x$ is
decremented by one.  Then the values of the variables $x$ and $y$ are
switched using the local variable $t$ as a buffer.  Our analysis
infers the tight bound $|[0,x]|+|[0,y]|$.  Sometimes we need some
assumptions on the inputs in order to derive a bound.  Example
\emph{t15} is such a case.  We assume here that the input variable $y$
is non-negative and write \code{assert(y>=0)}.  If we enter the loop
then we know that $x>0$ and we can obtain constant potential from the
assignment \code{x=x-1}.  After the assignment we know that $x\geq y$
and $y\geq 0$.  As a consequence, we can share the potential
$|[0,x]|$ before the assignment \code{x=x-y} between $|[0,x]|$ and
$|[0,y]|$ after the assignment.  In this way, we derive a tight linear
bound.

Example \emph{t13} shows how amortization can be used to find linear
bounds for nested loops.  The outer loop is iterated $|[0,x]|$ times.
In the conditional, we either (the branching condition is arbitrary)
increment the variable $y$ or we execute an inner loop in which $y$ is
counted back to $0$.  The analysis computes a tight bound.
\iffull{Again, the constants $0$ and $1$ in the
  inner loop can as well be replace by something more interesting, say
  $9$ and $10$ like in example \emph{t08}.  Then we still obtain a
  tight linear bound.}{}

\begin{figure}[t]
  \centering
    \begin{minipage}[b]{.9\linewidth}
    \begin{center}
   \begin{lstlisting}
int srch(
  int t[], int n,  /* haystack */
  int p[], int m,  /* needle */
  int b[]
) { int i=0, j=0, k=-1;
    while (i < n) {
      while (j >= 0 && t[i]!=p[j]) {
        k = b[j];
        assert(k > 0 && k <= j + 1);
        j -= k; tick(1)
      }
      i++, j++;
      if (j == m) break;
      tick(1);
    }
    return i;
  }
   \end{lstlisting}
\vspace{-1.2cm}
\hspace{3.8cm}
\footnotesize
\begin{tabular}{|rl|}
\hline
AAA & $1 {+} 2|\inter 0 n|$ \\
Rank & $O(n^2)$ \\
LOOPUS & $2 {+} 3\max(n, 0)$\\
\hline
\end{tabular}
    \end{center}
\vspace{-0.1cm}
  \end{minipage}
  \caption{The Knuth-Morris-Pratt algorithm for string search.}
  \label{fig:KMP}
\end{figure}


Finally, \pref{fig:KMP} contains the search function of the
Knuth-Morris-Pratt algorithm for string search.  Our automatic
amortized analysis finds the tight linear bound $1+2|[0,n]|$.  We need
to assert that the elements $b[j]$ of the failure table $b$ are in the
interval $[1,j+1]$.  This is guaranteed by
construction of the table in the initialization procedure of the
algorithm, which we can also analyze automatically.  We need the
assertion since we do not infer any logical assertions on the contents
of the heap.  Rank derives a complex quadratic bound and LOOPUS
derives a linear bound.


To compare our tool with existing work, we focused on loop bounds and
use a simple metric that counts the number of back edges (i.e., number
of loop iterations) that are followed in the execution of the program
because most other tools only bound this specific cost. In
\pref{fig:compar}, we show the bounds we derived (AAA) together with
the bounds derived by LOOPUS~\cite{SinnZV14} and
Rank~\cite{AliasDFG10}.  We also contacted the authors of SPEED but
have not been able to obtain this tool.
KoAT~\cite{BrockschmidtEFFG14} and PUBS~\cite{AlbertAGPZ12} currently
cannot operate on C code and the examples would need to be manually
translated into a term-rewriting system to be analyzed by these tools.  For
Rank it is not totally clear how the computed bound relates to the C
program since the computed bound is for transitions in an automaton
that is derived from the C code.  For instance, the bound $2+y-x$ that
is derived for \emph{t08} only applies to the first loop in the
program.

\begin{table}[t]
\centering
\begin{tabular}{r|c|c|c|c|c}
& KoAT & Rank & LOOPUS & SPEED & AAA \\
\hline
\#bounds & 9 & 24 & 20 & 14 & 32
\\
\#lin.\ bounds & 9 & 21 & 20 & 14 & 32
\\
\#best bounds & 0 & 0 & 11 & 14 & 29
\\
\#tested & 14 & 33 & 33 & 14 & 33
\end{tabular}
\vspace{.1cm}
\caption{Comparison of our automatic amortized analysis with other
  automatic tools.  We have not been able to obtain a version of
  SPEED~\cite{GulwaniMC09} and just use the bounds that have been
  reported by the authors. Similarly, KoAT~\cite{BrockschmidtEFFG14}
  does currently not work on C programs and we use the bounds that
  have been reported in the author's experimental evaluation.}
\label{tab:compar}
\end{table}


\pref{tab:compar} summarizes the results of our experiments.  It shows
for each tool the number of derived bounds (\#bounds), the number of
asymptotically tight bounds (\#lin. bounds), the number of bounds with
the best constant factors in comparison with the other tools (\#best
bounds), and the number of examples that we were able to test with the
tool (\#tested).  Since we were not able to run the experiments for KoAT
and SPEED, we simply used the bounds that have been reported by the
authors of the respective tools.  The results show that our automatic
amortized analysis outperforms the existing tools on our example
programs.  However, this experimental evaluation has to be taken with
a grain of salt.  Our goal in this work is not to set a new standard
for automatic bound analysis but only to show that our approach has
advantages on examples with linear bounds.  Overall the existing tools
are more powerful since they can derive polynomial bounds and support
more features of C.  We were particularly impressed by LOOPUS which is
very robust, works on large C files, and derives very precise bounds.
We did not include the running times of the tools in the table since
all tested tools work very efficiently and need less then a second
on every tested example.

Since we have a formal cost semantics, we can run our examples with
this semantics for different inputs and measure the cost to compare it
to our derived bound.  \pref{fig:3d} shows such a comparison for
example \emph{t08}.  One can see that the derived constant factors are the
best possible if the input variable $x$ is non-negative.  If $x$ is
negative then the bound is only slightly off.

\sectskip
\section{Related Work}
\label{sec:related}
\aftersectskip

Most closely related to this article is \iffull{our}{a} previous work
on end-to-end stack-bound verification of Clight
programs~\cite{veristack14}.  In that work, \iffull{we}{the authors}
have implemented and verified a quantitative logic to reason about
stack-space usage.  In this article, we present a more general
quantitative Hoare logic that is parametric in the resource of
interest.  The main innovation is the novel automatic amortized analysis for
Clight programs that computes logical derivations for non-trivial
bounds for programs with loops and recursion.

Our work has been inspired by type-based amortized resource analysis
for functional programs~\cite{Jost03,HoffmannH10,HoffmannAH12}.  There
are three major improvements over previous work in the current paper,
which presents the first automatic amortized resource analysis for C.
First, we solved the long-standing open problem of extending automatic
amortized resource analysis to compute bounds for programs that loop
on (possibly negative) integers.  Second, we extended the analysis
system to deal with non-linear control flow that is introduced by
\code{break} and \code{return} statements.  Third, for the first time,
we have combined an automatic amortized analysis with a system for
interactively deriving bounds.

In the development of our quantitative Hoare logic we have drawn
inspiration from mechanically verified Hoare logics.
Nipkow's~\cite{Nipkow02} description of his implementations of Hoare
logics in Isabelle/HOL has been helpful to understand the interaction
of auxiliary variables with the consequence rule.  Appel's separation
logic for CompCert Clight~\cite{AppelLogic} has been a blueprint for
the general structure of the quantitative logic.  \iffull{Since we do not deal
with memory safety, our logic is much simpler and it would be possible
to integrate it with Appel's logic.}{}  The continuation passing style
that we use in the quantitative logic is not only used by
Appel~\cite{AppelLogic} but also in Hoare logics for low-level
code~\cite{NiS06,JensenBK13}.

There exist quantitative logics that are integrated into separation
logic~\cite{Atkey10,HoffmannMZ13} and they are closely related to our
quantitative logic.  However, the purpose of these logics is slightly
different since they focus on the verification of bounds that depend
on the shape of heap data structures and they are not implemented for
C.  Also closely related to our logic is a VDM-style logic for
reasoning about resource usage of an abstract fragment of JVM byte
code by Aspinall et al.~\cite{AspinallBHLM07}.  Their logic is not
Hoare-style, does not target C code, and is not designed for
interactive bound development but to produce certificates for bounds
derived for high-level functional programs.

\begin{figure}[t]
\center
\includegraphics[width=.9\linewidth]{fig/bound3d}
\caption{The automatically derived bound $1.33|[x,y]| + 0.33 |[0,x]|$
  (blue lines) and the measured runtime cost (red crosses) for example
  \emph{t08}. For $x\ge 0$ the derived bound is tight.}
\label{fig:3d}
\end{figure}

There exist many tools that can automatically derive loop and
recursion bounds for imperative programs such as
SPEED~\cite{GulwaniMC09,GulwaniZ10}, KoAT~\cite{BrockschmidtEFFG14},
PUBS~\cite{AlbertAGPZ12}, Rank~\cite{AliasDFG10},
ABC~\cite{BlancHHK10} and LOOPUS~\cite{Zuleger11,SinnZV14}.  These
tools are based on abstract interpretation--based invariant generation
and/or term rewriting techniques, and they derive impressive results
on realistic software.  The importance of amortization to derive tight
bounds is well known in the resource analysis
community~\cite{AlonsoG12,Moser14,SinnZV14}.  Currently, the only
other available tools that can be directly applied to C code are Rank
and LOOPUS.  Our analysis framework does not aim to set a new standard
for automatic bound analysis.  Our contribution is rather a principled
approach that produces certificates that are proved sound with respect
to a formal cost semantics.  Moreover, we have a system that enables
both automatic and interactive bound derivation, a formal soundness
proof in Coq, and a method that can handle resources that can be
released (e.g., memory).  However, as we have shown
in~\pref{sec:exper}, our automatic amortized analysis matches the
state of the art in automatic bound analysis for linear bounds and
sometimes even derives better constant factors than the other tools.
Since Rank and Loopus do not handle recursion, our automatic amortized
analysis is also the only available tool that can derive bounds for
recursive C programs.

There are techniques~\cite{Braberman08} that can compute the memory
requirements of object oriented programs with region-based garbage
collection.  These systems infer invariants and use external tools
that count the number of integer points in the corresponding polytopes
to obtain bonds.  The described technique can handle loops but not
recursive or composed functions.

We are only aware of two verified quantitative analysis systems.
Albert et al.~\cite{AlbertBGHR12} rely on the KeY tool to
automatically verify previously inferred loop invariants, size
relations, and ranking functions for Java Card programs.  However,
they do not have a formal cost semantics and do not prove the bounds
correct with respect to a cost model.  Blazy et al.~\cite{Blazy13}
have verified a loop bound analysis for CompCert's RTL intermediate
language.  However, there is no way to interactively derive bounds or
to deal with resources like memory usage.  Furthermore, Blazy et al.'s
automatic bound analysis does not compute symbolic bounds.


\newcommand{\loopbody}[0]{\ensuremath{\code{if~(x - 10 \geq 0)~\{x=x{-}10; tick(5);\}~else~break;}}}

\begin{figure*}[t]
  \centering
  \footnotesize
  \begin{mathpar}
  \inferrule*[right={\scriptsize (Q:Loop)},leftskip=0cm,rightskip=0cm,width=20cm]
  { \inferrule*[right={\scriptsize (Q:If)},leftskip=0cm,rightskip=0cm,width=20cm]
    {
      \inferrule*[right={\scriptsize (Q:Weak)},leftskip=2cm,rightskip=4cm, vdots=1.5cm]
      {
        \inferrule*[right={\scriptsize (Q:Seq)},leftskip=0cm,rightskip=0cm]
        {
\inferrule*[right={\scriptsize (Q:Weak)},leftskip=0cm,rightskip=0cm]
{
  \inferrule*[right={\scriptsize (Q:DecP)},leftskip=0cm,rightskip=0cm]
  { }
  {(x {<} 10, B^\text{de}); ((x {\geq} 10,Q^\text{de})  \vdash \code{x = x - 10}
   \dashv (\cdot, P^\text{de})
  }
}
{(x {<} 10, B^\text{we}); ((x {\geq} 10,Q^\text{we})  \vdash \code{x = x - 10}
\dashv (\cdot, P^\text{we})
}
\and
\inferrule*[right={\scriptsize (Q:Tick)},leftskip=0cm,rightskip=0cm]
{ }
{(x {<} 10, B^\text{ti}); (\cdot,Q^\text{ti})  \vdash \code{tick(5)}
 \dashv (\cdot, P^\text{ti})
}
} {
          (x {<} 10, B^\text{sq}); ((x {\geq} 10,Q^\text{sq}) \vdash \code{x=x{-}10; tick(5);} \dashv (\cdot,P^\text{sq})
        }
      }
      { (x {<} 10, B^\text{if}); ((x {\geq} 10,Q^\text{if}) \vdash \code{x=x{-}10; tick(5);} \dashv (\cdot,P^\text{if}) }
      \and
      \inferrule*[right={\scriptsize (Q:Weak)},leftskip=4.5cm,rightskip=4cm]
      { \inferrule*[right={\scriptsize (Q:Break)},leftskip=0cm,rightskip=0cm]
        { }
        {(x {<} 10, B^\text{br}); (x {<} 10,Q^\text{br}) \vdash \code{break} \dashv (\bot,P^\text{br})}
      }
      {
        (x {<} 10, B^\text{el}); (x {<} 10,Q^\text{el}) \vdash \code{break} \dashv (\cdot,P^\text{el})
      }
    }
    { \hspace{3cm} (x {<} 10, B^\text{lo}); (\cdot,Q^\text{lo}) \vdash \loopbody \dashv (\cdot,P^\text{lo}) \hspace{3cm}
    }
  }
  {
   (\cdot, B); (\cdot, Q) \vdash \code{loop}~{\loopbody} \dashv (x {<} 10, P)
  }
\end{mathpar}
\flushleft Constraints:
$$
\begin{array}{llll}
P = B^\text{lo} & Q  = Q^\text{lo} & Q^\text{lo} = P^\text{lo} \\
B^\text{el} = B^\text{if} = B^\text{lo} & Q^\text{el} = Q^\text{if} = Q^\text{lo} & P^\text{el} = P^\text{if} = P^\text{lo} \\
B^\text{el} = B^\text{br}  & Q^\text{el} \succeq_{(x{<}10)} Q^\text{br}  & P^\text{br} \succeq_{(\cdot)} P^\text{el} \\
B^\text{br} = Q^\text{br}\\
B^\text{if} = B^\text{sq}  & Q^\text{if} \succeq_{(x{<}10)} Q^\text{sq}  & P^\text{sq} \succeq_{(\cdot)} P^\text{if} \\
B^\text{sq} = B^\text{we} = B^\text{ti} & Q^\text{sq} = Q^\text{we} & P^\text{we} = Q^\text{ti} & P^\text{ti} = P^\text{sq} \\
Q^\text{ti} = P^\text{ti} + 5 \\
B^\text{we} = B^\text{de}  & Q^\text{we} \succeq_{(x{<}10)} Q^\text{de}  & P^\text{de} \succeq_{(\cdot)} P^\text{we} \\
p^\text{de}_{0,10} = q^\text{de}_{0,10} + q^\text{de}_{0,x}  & p^\text{de}_{0} = q^\text{de}_{0}
& \forall (\alpha, \beta) \neq (0,10)  . \, p^\text{de}_{\alpha,\beta} = q^\text{de}_{\alpha,\beta}
\end{array}
$$
\vspace{1.5ex}


\begin{tabular}{l@{\hspace{5em}}l}
 Linear Objective Function: &  Constant Objective Function: \\
\hspace{3em} $
q_{x,0} + 10000{\cdot}q_{0,x} + 10{\cdot}q_{x,10} + 9990q_{10,x}
$
&
\hspace{3em}
$
q_0 + 11{\cdot}q_{0,10}
$
\end{tabular}

\caption{An example derivation as produced by the automatic tool.}
\label{fig:derivation}
\end{figure*}
\sectskip
\section{Conclusion}
\label{sec:concl}
\aftersectskip

We have introduced an novel quantitative analysis system for CompCert
Clight programs.  To the best of our knowledge, this article presents
the first resource analysis framework for C that makes it possible to
combine non-trivial automatically derived bounds with interactively
derived bounds in a proof system that produces verifiable certificates
for the bounds.  The main technical innovations are a quantitative
Hoare logic for reasoning about user-defined resource cost and an
automatic amortized analysis that derives bounds for programs whose
resource consumption depends on (possibly negative) integers and
non-sequential control flow as introduced by \code{break} and
\code{return} statements.

We will continue to improve our framework to reason more precisely
about resource usage of system software such as the hypervisor kernel
CertiKOS~\cite{GuVFSC11}, which \iffull{we are currently developing
  and verifying}{is currently developed and verified}.  For one thing,
we will improve the automation to derive \emph{polynomial bounds} and
to handle non-linear size changes, as already developed for functional
programs~\cite{HoffmannS13}.  For another thing, we will build on
previous work~\cite{HoffmannMZ13} to generalize the quantitative
Hoare logic to handle \emph{concurrent programs}\iffull{ with locks and
lock-free data structures.}{.}

% \acks

% This research is based on work supported in part by NSF grants 1319671
% and 1065451, and DARPA grants FA8750-10-2-0254 and FA8750-12-2-0293.
% Any opinions, findings, and conclusions contained in this document are
% those of the authors and do not reflect the views of these agencies.



\ifx\fullversion\undefined{}\else{
%\clearpage
\appendix

\newpage

\section{Example Derivation in the Automatic System}
\label{app:derivation}


\pref{fig:derivation} contains an example derivation as it is produced
by our automatic analyzer.  The upper case letters (with optional
superscript) such as $Q^\text{de}$ stand for a family of rational
variables that are later part of the constraint system that is passed
to the LP solver.  For example $Q^\text{de}$ stands for the potential
function $q^\text{de}_0 + q^\text{de}_{x,0} |[x,0]| +
q^\text{de}_{0,x} |[0,x]| + q^\text{de}_{x,10} |[x,10]| +
q^\text{de}_{10,x} |[10,x]| + q^\text{de}_{0,10} |[0,10]|$, where the
variables such as $q^\text{de}_{x,10}$ are yet unknown and later
instantiated by the LP solver.  The derivation with the instantiated
variables is then a sound derivation in the verified program logic.

The rules are syntax directed and applied inductively.  That means
that we follow the inductive definition of the syntax.  For example,
the outermost expression is a loop, so we use the rule \text{Q:Loop}
at the root of the derivation tree.  At this point, we don't know yet
whether a loop invariant exists.  But we produce the constraints
$Q^\text{lo} = P^\text{lo}$ which is short for the following
constraint set.
$$
\begin{array}{lll}
q^\text{lo}_0 = p^\text{lo}_0 &
q^\text{lo}_{x,0}= p^\text{lo}_{x,0} &
q^\text{lo}_{0,x}= p^\text{lo}_{0,x} \\
q^\text{lo}_{x,10}= p^\text{lo}_{x,10} &
q^\text{lo}_{10,x}= p^\text{lo}_{10,x} &
q^\text{lo}_{0,10}= p^\text{lo}_{0,10}
\end{array}
$$
%
This constraint set expresses the fact that the potential before and
after the loop body is equal and thus constitutes an invariant.

In general, the weakening rule is applied after every syntax directed
rule.  Applying weakening everywhere does not lead to performance
problems. However, it can be left out in practice at some places to
increase the efficiency of the tool.

The weakening operation $\succeq_{\Gamma}$ is defined in the rule
\textsc{Relax}.  For example, we have
$$
\begin{array}{lll}
P^\text{de} \succeq_{(\cdot)} P^\text{we}
& \equiv &  p^\text{we}_{0,10} \leq p^\text{de}_{0,10} + u_{0,10} - v_{0.10}  \\
& \land &   p^\text{we}_{0} \leq p^\text{de}_{0} - 10 {\cdot} u_{0,10} + 10 {\cdot} v_{0.10}  \\
& \land &  \forall (\alpha, \beta) \neq (0,10)  . \, p^\text{we}_{\alpha,\beta} \leq p^\text{de}_{\alpha,\beta} \;\;\; .
\end{array}
$$

After the constraint generation, the LP solver is provided with a
so-called objective function that it will minimize.  We wish to
minimize the initial potential, which is the resource bound on the
whole program.  Here it is given by $Q$.  Moreover, we would like to
express that minimization of linear potential such as
$$
q_{10,x} |[10,x]|
$$
takes priority over minimization of constant potential such as
$$
q_{0,10} |[0,10]| \; .
$$

To this end, modern LP solvers support efficient resolving of
constraint sets with additional constraints: First we take our initial
constraint set as given in \pref{fig:derivation}.  We then as the
solver to find a solution that satisfies the constraint and minimizes
the linear expression
$$
q_{x,0} + 10000{\cdot}q_{0,x} + 10{\cdot}q_{x,10} + 9990q_{10,x} \; .
$$
Note the linear penalties (this is a random choice) that we give to
some of the factors.  We use it to express that the interval $[10,x]$
is smaller than the interval $[0,x]$ and, similarly, that the interval
$[x,0]$ is smaller than the interval $[x,10]$.

The LP solver now returns a solution of the constraint set and an
objective value, that is, a mapping from variables to floating-point
numbers and the value of the objective function with this
instantiation.  The solver also memorizes the optimization path that
led to the optimal solution.  In this case, the objective value would
be
$
5000
$
since the LP solver assigns $q_{0,x} = 0.5$ and $q_{*} = 0$ otherwise.
We now add the constraint
$$
q_{x,0} + 10000{\cdot}q_{0,x} + 10{\cdot}q_{x,10} + 9990q_{10,x} = 5000
$$
to our constraint set and ask the solver to optimize the objective
function
$$
q_0 + 11{\cdot}q_{0,10} \; .
$$
This happens in almost no time in practice.

The final solution is $q_{0,x} = 0.5$ and $q_{*} = 0$ otherwise.  Thus
the derived bound is $0.5 |[0,x]|$.

Another interesting note is that the constraints that we generate have
a particularly simple form that is known as \emph{network problem}.
Such problems can be solved in linear time in practice.

\newpage

\section{Catalog of Automatically Analyzed Programs}
\label{app:cat}

In this appendix we provide a non-exhaustive catalog of classes of
programs that can be automatically analyzed by our system.  For
simplicity, and to compare our analysis with existing tools, we use a
cost metric that counts the number of loop iterations and function
calls.  This is the only cost metric supported by tools like
KoAT~\cite{BrockschmidtEFFG14} and LOOPUS~\cite{SinnZV14}.  Sometimes
we also use the \emph{tick metric} if we want to discuss features such as
resource restitution.  The tick metric assigns cost $n$ to the
statement \code{tick(n)} and cost $0$ to all other statements. Of
course, the examples can also be analyzed with any other cost metric
in our system.

We assume that free variables in the code snippets are the inputs of
the program. Some of the examples contain constants on which the
computed bound depends.  These constants are randomly chosen to
present an example but the analysis works for other constants as well.
Note however that it is sometimes crucial that constants are positive
(or negative) or that other relations hold.

\pref{tab:eval} contains the details of our comparative evaluation.


\begin{figure*}[t!]
\setlength{\progwidth}{.24\linewidth}
  \centering

  \begin{minipage}[b]{\progwidth}
    \begin{center}
   \begin{lstlisting}
  while (y-x>0) {
    x = x+1;
  }
  while (x>2) {
    x=x-3;
  }
   \end{lstlisting}

$1.33|[x,y]| + 0.33|[0,x]|$
\\[.7\baselineskip]
      {\bf t08}
    \end{center}
  \end{minipage}
%
%
%
  \begin{minipage}[b]{\progwidth}
    \begin{center}
   \begin{lstlisting}
  while (x-y>0) {
    if (*)
      y=y+1;
    else
      x=x-1;
  }
   \end{lstlisting}

$|[y,x]|$
\\[.7\baselineskip]
      {\bf t10}
    \end{center}
  \end{minipage}
%
%
%
  \begin{minipage}[b]{\progwidth}
    \begin{center}
   \begin{lstlisting}
  while (x>0) {
    x=x-1;
    if (*)
      y=y+1;
    else {
      while (y>0)
        y=y-1;
    }
  }
   \end{lstlisting}

$2|[0,x]| + |[0,y]|$
\\[.7\baselineskip]
      {\bf t13}
    \end{center}
  \end{minipage}
%
%
%
  \begin{minipage}[b]{\progwidth}
    \begin{center}
   \begin{lstlisting}
while (n<0) {
  n=n+1;
  y=y+1000;
  while (y>=100 && *){
    y=y-100;
  }
}
   \end{lstlisting}

$11|[n,0]| + 0.01|[0,y]|$
\\[.7\baselineskip]
      {\bf t27}
    \end{center}
  \end{minipage}
   \caption{Amortization and Compositionality (a).}
  \label{fig:cat1a}
\end{figure*}


\begin{figure*}[t!]
 \setlength{\progwidth}{.24\linewidth}
  \centering

  \begin{minipage}[b]{\progwidth}
    \begin{center}
   \begin{lstlisting}
  while (x>0) {
    x=x-1;
    y=y+2;
  }
  while (y>0) {
    y=y-1;
  }
  while (y>0) {
    y=y+1;
  }
   \end{lstlisting}

$1 + 3|[0,x]| + |[0,y]|$
\\[.7\baselineskip]
      {\bf t07}
    \end{center}
  \end{minipage}%
%
%
%
  \begin{minipage}[b]{\progwidth}
    \begin{center}
   \begin{lstlisting}
  while (x>y) {
    x=x-1;
    x=x+1000;
    y=y+1000;
  }
  while (y>0) {
    y=y-1;
  }
  while (x<0) {
    x=x+1;
  }
   \end{lstlisting}

$1002|[y,x]|+|[x,0]|+|[0,y]|$
\\[.7\baselineskip]
      {\bf t28}
    \end{center}
  \end{minipage}%
%
%
  \begin{minipage}[b]{\progwidth}
    \begin{center}
   \begin{lstlisting}
  assert (y>=0);
  while (x-y>0) {
    x=x-1;
    x=x-y;
    z=y;
    while (z>0) {
      z=z-1;
    }
  }
   \end{lstlisting}

$|[0,x]|$
\\[.7\baselineskip]
      {\bf t15}
    \end{center}
  \end{minipage}
%
%
  \begin{minipage}[b]{\progwidth}
    \begin{center}
   \begin{lstlisting}
  assert (y>=0);
  while (x-y>0) {
    x=x-1;
    x=x-y;
    z=y;
    z=z+y;
    z=z+100;
    while (z>0) {
      z=z-1;
    }
  }
   \end{lstlisting}

$101|[y,x]|$
\\[.7\baselineskip]
      {\bf t16}
    \end{center}
  \end{minipage}


   \caption{Amortization and Compositionality (b).}
  \label{fig:cat1b}
\end{figure*}


\begin{figure*}[t!]
 \setlength{\progwidth}{.24\linewidth}
  \centering

  \begin{minipage}[b]{\progwidth}
    \begin{center}
   \begin{lstlisting}
  while (i>100) {
    i--;
  }
  i=i+k+50;
  while (i>=0) {
    i--;
  }
   \end{lstlisting}

$50 + |[-1,i]| + |[0,k]|$
\\[.7\baselineskip]
      {\bf t19}
    \end{center}
  \end{minipage}%
%
%
%
  \begin{minipage}[b]{\progwidth}
    \begin{center}
   \begin{lstlisting}
  while (x<y) {
    x=x+1;
  }
  while (y<x) {
    y=y+1;
  }
   \end{lstlisting}

$|[x,y]|+|[y,x]|$
\\[.7\baselineskip]
      {\bf t20}
    \end{center}
  \end{minipage}%
%
%
  \begin{minipage}[b]{\progwidth}
    \begin{center}
   \begin{lstlisting}
  while (x>0) {
    x=x-1;
    t=x;
    x=y;
    y=t;
  }
   \end{lstlisting}

$|[0,x]|+|[0,y]|$
\\[.7\baselineskip]
      {\bf t30}
    \end{center}
  \end{minipage}
%
%
  \begin{minipage}[b]{\progwidth}
    \begin{center}
   \begin{lstlisting}
  flag=1;
  while (flag>0) {
    if (n>0 && *) {
      n=n-1;
      flag=1;
    } else
      flag=0;
  }
   \end{lstlisting}

$1 + |[0, n]|$
\\[.7\baselineskip]
      {\bf t47}
    \end{center}
  \end{minipage}

   \caption{Amortization and Compositionality (c).}
  \label{fig:cat1c}
\end{figure*}

\paragraph{Amortization and Compositionality}

Figures~\ref{fig:cat1a}, \ref{fig:cat1b}, and~\ref{fig:cat1c} show
code snippets that need amortization and compositionality to obtain a
whole program bound.

Example \emph{t07} demonstrates two different features of the
analysis.  For one thing it shows that we can precisely track size
changes inside loops.  In the first loop, we increment $y$ by $2$ in
each of the $|[0,x]|$ iterations.  An in the second loop, we decrement
$y$.  For another thing it shows that we automatically recognize dead
code if we find conflicting assertions on a branching path: After the
second loop we know $y \leq 0$ and as a result can assign arbitrary
potential inside the third loop where we know that $y>0$.  As a
result, we obtain a tight bound.

Example \emph{t08} shows the ability of the analysis to handle
negative and non-negative numbers.  Note that there are no
restrictions on the signs of $y$ and $x$.  We also see again that we
accurately track the size change of $x$ in the first loop.
Furthermore, \emph{t08} shows that we do not handle the constants $1$
or $0$ in any special way.  In all examples you could replace $0$ and
$1$ with other constants like we did in the second loop and still
derive a tight bound.  The only information, that the analyzer needs
is $x \geq c$ before assigning $x = x - c$.

In example \emph{t10} we also do not restrict the inputs $x$ and $y$.
They can be negative, positive, or zero.  The star {\tt *} in the
conditional, stands for an arbitrary assertion.  In each branch of the
conditional we can obtain the constant potential $1$ since the interval
size $|[y,x]|$ is decreasing.

Example \emph{t13} shows how amortization can be used to handle tricky
nested loops.  The outer loop is iterated $|[0,x]|$ times.  In the
conditional, we either (the branching condition is again arbitrary)
increment the variable $y$ or we execute an inner loop in which $y$ is
counted back to $0$.  The analysis computes a tight linear bound for
this program.  Again, the constants $0$ and $1$ in the inner loop can
as well be replace by something more interesting, say $9$ and $10$
like in example \emph{t08}.  Then we still obtain a tight linear
bound.

Example \emph{t27} is similar to example \emph{t13}.  Instead of
decrementing the variable $x$ in the outer loop we this time increment
the variable $n$ till $n = 0$.  In each of the $|[n,0]|$ iterations,
we increment the variable $y$ by $1000$.  We then execute an inner
loop that increments $y$ by $100$ until $y=0$.  The analysis can
derive that only the first execution of the inner loop depends on the
initial value of $y$.  We again derive a tight bound.

Example \emph{t28} is particularly interesting.  In the first loop we
decrement the size $|[y,x]|$.  However, we also shift the interval
$[y,x]$ to the interval $[y+1000,x+1000]$.  The analysis can derive
that this does not change the size of the interval and computes the
tight loop bound $|[y,x]|$.  The additional two loops are in the
program to show that the size tracking in the first loop works
accurately.  The second loop is executed $|[0,y]| + 1000|[y,x]|$ times
in the worst case.  The third loop is executed $|[x,0]| + |[y,x]|$ in
the worst case (if $x$ and $y$ are negative).

Sometimes we need some assumptions on the inputs in order to derive a
bound.  Example \emph{t15} is such a case.  We assume here that the
input variable $y$ is non-negative and write \code{assert(y>=0)}.  The
semantic of \code{assert} is that it has no effect if the assertion is
true and that the program is terminated without further cost
otherwise.  If we enter the loop then we know that $x>0$ and we can
obtain constant potential from the assignment \code{x=x-1}.  After the
assignment we know that $x\geq y$ and $y\geq 0$.  As a consequence, we
can share the potential $|[0,x]|$ before the assignment \code{x=x-y}
between $|[0,x]|$ and $|[0,y]|$ after the assignment.  In this way, we
derive a tight linear bound.

Example \emph{t16} is an extension of example \emph{t15}. We again
assume that $y$ is non-negative and use the same mechanism to iterate
the outer loop as in \emph{t15}.  In the inner loop, we also count the
variable $z$ down to zero and perform $|[0,z]|$ iterations.  However,
instead of assigning \code{z=y}, we assign \code{z=2y+100}.  The analysis
computes the linear bound $101|[0,x]|$.  The
assignment of potential to the size interval $|[0,x]|$ in instead of
$|[y,x]|$ is a random choice of the LP solver.

Example \emph{t19} consists of two loops that decrement a variable
$i$.  In the first loop, $i$ is decremented down to 100 and in the
second loop $i$ is increment further down to $-1$.  However, between
the loops we assign $i=i+k+50$.  So in total the program performs $50
+ |[-1,i]| + |[0,k]|$ iterations.  Our analysis finds this tight bound
because our amortized analysis naturally takes into account the
relation between the two loops.  Techniques that do not use
amortization derive a more conservative bound such as $50 + |[-1,i]| +
|[0,k]| + |[100,i]|$.

Example \emph{t20} shows how we can handle programs in which bounds
contain absolute values like $|x-y|$.  The first loop body is only
executed if $x<y$ and the second loop body is only executed if $y<x$.
The analyzer finds a tight bound.

At first sight, example \emph{t30} appears to be a simple loop that
decrements the variable $x$ down to zero.  However, a closer look
reveals that the loop actually decrements both input variables $x$ and
$y$ down to zero before terminating.  In the loop body, first $x$ is
decremented by one.  Then the values of the variables $x$ and $y$ are
switched using the local variable $t$ as a buffer.  Our analysis
infers the tight bound $|[0,x]|+|[0,y]|$.

Example \emph{t47} demonstrates how we can use integers as Booleans to
amortize the cost of loops that depend on boolean flags.  The outer
loop is executed as long as the variable flag is ``true'', that is,
\code{flag>0}.  Inside the loop, there is a conditional that either
(if $n>0$) decrements $n$ and assigns \code{flag=1}, or (if $n\geq0$)
leaves n unchanged and assigns \code{flag=0}.  The analyzer computes
the tight bound $1 + |[0, n]|$.  The potential in the loop invariant
is $|[0,\mathit{flag}]| + |[0, n]|$.  In the \emph{then} branch of
the conditional, we use the potential $|[0, n]|$ and the fact that
$n>0$.  In the \emph{else} branch, we use the potential
$|[0,\mathit{flag}]|$ and the fact that $\mathit{flag}=1$.




\begin{figure*}[t!]
 \setlength{\progwidth}{.24\linewidth}
  \centering
  \begin{minipage}[b]{\progwidth}
    \begin{center}
   \begin{lstlisting}
  while (n>x) {
    if (m>y)
      y = y+1;
    else
      x = x+1;
  }
   \end{lstlisting}

$|[x, n]| + |[y, m]|$
\\[.7\baselineskip]
      {\bf fig2\_1}
    \end{center}
  \end{minipage}
%
%
  \begin{minipage}[b]{\progwidth}
    \begin{center}
   \begin{lstlisting}
  while (x<n) {
    if (z>x)
      x=x+1;
    else
      z=z+1;
  }
   \end{lstlisting}

$|[x, n]| + |[z, n]|$
\\[.7\baselineskip]
      {\bf fig2\_2}
    \end{center}
  \end{minipage}
%
%
  \begin{minipage}[b]{\progwidth}
    \begin{center}
   \begin{lstlisting}
  while (x<n) {
    while (y<m) {
      if (*) break;
      y=y+1;
    }
    x=x+1;
  }
   \end{lstlisting}

$|[x, n]| + |[y, m]|$
\\[.7\baselineskip]
      {\bf nested\_multiple}
    \end{center}
  \end{minipage}
%
%
  \begin{minipage}[b]{\progwidth}
    \begin{center}
   \begin{lstlisting}
  x=0;
  while (x<n) {
    x=x+1;
    while (x<n) {
      if (*) break;
      x=x+1;
    }
  }
   \end{lstlisting}

$|[0, n]|$
\\[.7\baselineskip]
      {\bf nested\_single}
    \end{center}
  \end{minipage}

   \caption{Examples from Gulwani et al's SPEED~\cite{GulwaniMC09} (a).}
  \label{fig:cat2a}
\end{figure*}

\begin{figure*}[t!]
 \setlength{\progwidth}{.24\linewidth}
  \centering
  \begin{minipage}[b]{\progwidth}
    \begin{center}
   \begin{lstlisting}
  x=0;
  while (x<n) {
    if (*) break;
    x=x+1;
  }
  while (x<n)
    x=x+1;
   \end{lstlisting}

$|[0,n]|$
\\[.7\baselineskip]
      {\bf sequential\_single}
    \end{center}
  \end{minipage}
%
%
  \begin{minipage}[b]{\progwidth}
    \begin{center}
   \begin{lstlisting}
  x=0; y=0;
  while (x<n) {
    if (y<m)
      y=y+1;
    else
      x=x+1;
  }
   \end{lstlisting}
$|[0, m]| + |[0, n]|$
\\[.7\baselineskip]
      {\bf simple\_multiple}
    \end{center}
  \end{minipage}
%
%
  \begin{minipage}[b]{\progwidth}
    \begin{center}
   \begin{lstlisting}
  x=0;
  while (x<n) {
    if (*)
      x=x+1;
    else
      x=x+1;
  }
   \end{lstlisting}

$|[0,n]|$
\\[.7\baselineskip]
      {\bf simple\_single}
    \end{center}
  \end{minipage}
%
%
  \begin{minipage}[b]{\progwidth}
    \begin{center}
   \begin{lstlisting}
  x=0; y=0;
  while (*) {
    if (x<N) {
      x=x+1; y=y+1;
    } else if (y<M ) {
      x=x+1; y=y+1;
    } else
      break;
  }
   \end{lstlisting}

$|[0, M]| + |[0, N]|$
\\[.7\baselineskip]
      {\bf simple\_single\_2}
    \end{center}
  \end{minipage}

   \caption{Examples from Gulwani et al's SPEED~\cite{GulwaniMC09} (b).}
  \label{fig:cat2b}
\end{figure*}





\begin{figure*}[t!]
 \setlength{\progwidth}{.24\linewidth}
  \centering
%
%
  \begin{minipage}[b]{\progwidth}
    \begin{center}
   \begin{lstlisting}
  assert n>0;
  assert m>0;
  va = n; vb = 0;
  while (va>0 && *) {
    if (vb<m) {
      vb=vb+1;
      va=va-1;
    } else {
      vb=vb-1;
      vb=0;
    }
  }
   \end{lstlisting}
$1 + 2|[0, n]|$
\\[.7\baselineskip]
      {\bf fig4\_2}
    \end{center}
  \end{minipage}
%
%
%
%
  \begin{minipage}[b]{\progwidth}
    \begin{center}
   \begin{lstlisting}
  assert (0<m);
  i = n;
  while (i>0 && *) {
    if (i<m)
      i=i-1;
    else
      i=i-m;
  }
   \end{lstlisting}
$|[0, n]|$
\\[.7\baselineskip]
      {\bf fig4\_4}
    \end{center}
  \end{minipage}
%
%
  \begin{minipage}[b]{\progwidth}
    \begin{center}
   \begin{lstlisting}
  assert (0 < m < n);
  i=m;
  while (0<i<n) {
    if (dir==fwd) i++;
    else i--;
  }
   \end{lstlisting}
$---$
\\[.7\baselineskip]
      {\bf fig4\_5}
    \end{center}
  \end{minipage}

   \caption{Examples from~\cite{GulwaniJK09}.}
  \label{fig:cat2c}
\end{figure*}


\begin{figure*}[t!]
 \setlength{\progwidth}{.24\linewidth}
  \centering
%
%
  \begin{minipage}[b]{\progwidth}
    \begin{center}
   \begin{lstlisting}
  i=0;
  while (i<n) {
    j=i+1;
    while (j<n) {
      if (*) {
        tick(1);
        j=j-1; n=n-1;
      }
      j=j+1;
    }
    i=i+1;
  }
   \end{lstlisting}
$|[0, n]| \text{ ticks}$
\\[.7\baselineskip]
      {\bf ex1}
    \end{center}
  \end{minipage}
%
%
  \begin{minipage}[b]{\progwidth}
    \begin{center}
   \begin{lstlisting}
  while (n>0 && m>0) {
    n--; m--;
    while (nondet()) {
      n--; m++;
    };
    tick(1);
  }
   \end{lstlisting}
$|[0,n]|$ ticks
\\[.7\baselineskip]
      {\bf ex2}
    \end{center}
  \end{minipage}
%
%
  \begin{minipage}[b]{\progwidth}
    \begin{center}
   \begin{lstlisting}
  while (n>0) {
    t = x;
    n=n-1;
    while (n>0) {
      if (*) break;
      n=n-1;
    }
  }
   \end{lstlisting}
$|[0, n]|$
\\[.7\baselineskip]
      {\bf ex3}
    \end{center}
  \end{minipage}
%
%
  \begin{minipage}[b]{\progwidth}
    \begin{center}
   \begin{lstlisting}
  flag=1;
  while (flag>0) {
    flag=0;
    while (n>0 && *) {
      n=n-1;
      flag=1;
    }
  }
   \end{lstlisting}
$1 + 2|[0, n]|$
\\[.7\baselineskip]
      {\bf ex4}
    \end{center}
  \end{minipage}


   \caption{Examples from~\cite{GulwaniZ10}.}
  \label{fig:cat3a}
\end{figure*}


\paragraph{From  the Literature}

Our analyzer can derive almost all linear bounds for programs that
have been described as challenges in the literature on bound
generation.  We found only one program with a linear bound for
which our analyzer could not find a tight bound:  Example
(\emph{fig4\_5}) from~~\cite{GulwaniJK09} requires \emph{path-sensitive
  reasoning} to derive a bound.

Examples \emph{fig2\_1} and \emph{fig2\_2} are taken from Gulwani et
al~\cite{GulwaniMC09}.  They are both handled by the SPEED tool but
require inference of a \emph{disjunctive invariant}.  In the abstract
interpretation community, these invariants are known to be notoriously
difficult to handle.
%
In example \emph{fig2\_1} we have one loop that first increments
variable $y$ up to $m$ and then increments variable $x$ up to $n$.  We
derive the tight bound $|[x, n]| + |[y, m]|$.
%
Example \emph{fig2\_2} is more tricky and trying to understand how it
works may be challenging.  However, with the amortized analysis in
mind, using the potential transfer reasoning, it is almost trivial to
prove a bound.  While the SPEED tool has to find a fairly
involved invariant for the loop, our tool is simply reasoning locally
and works without any clever tricks. We obtain the tight bound $|[x,
n]| + |[z, n]|$.

Example \emph{nested\_multiple} is similar to example \emph{fig2\_1}.
Instead of incrementing variable $y$ in the outer loop, $y$ is here
potentially incremented multiple times in each iteration of the outer
loop.  The idea of example \emph{nested\_single} is similar.  However,
instead of incrementing variable $y$ in the inner loop, we increment
$x$, the counter variable of the outer loop. Our analyzer derives a
tight bound for both programs.  Note that a star \emph{*} in a
branching condition denotes an arbitrary boolean condition that might
of course change while iterating (non-deterministic choice).

Example \emph{sequential\_single} is like example
\emph{nested\_single}.  The only difference is that the inner loop of
\emph{nested\_single} is now evaluated after the outer loop.  Example
\emph{simple\_multiple} is a variant of example \emph{fig2\_1} and
\emph{simple\_single} is a simple variant of \emph{nested\_single}.
We derive tight bounds for all aforementioned programs.

Example \emph{simple\_single\_2} uses conditionals and a \code{break}
statement to control loop iterations.  If $x<N$ then variables $x$ and
$y$ are incremented.  Otherwise, if $y<M$ then the same increment is
executed.  If $y\geq M$ and $x\geq N$ then the loop is terminated with
a break.  Our tool computes the bound $|[0, M]| + |[0, N]|$.
This bound is tight in the sense that there are inputs (such as $M =
-100$ and $N = 100$) for which the bound precisely describes the
execution cost.  However, SPEED can compute the more precise bound
$\max(N,M)$.  We currently cannot express this bound in our system.

Example \emph{fig4\_2} from~\cite{GulwaniJK09} is quite involved.
Amortized reasoning helps to understand how we derive the bound $1 +
2|[0, n]|$.  We start with potential $1 + 2|[0, n]|$ and use the fact
that $vb=0$ to establish the potential $1 + 2|[0, n]| + |[0,vb]|$ that
serves as a loop invariant.  In the \emph{if branch} of the
conditional, we use the constant potential $1$ of the invariant to pay
for the potential of $|[0,vb]|$.  Since we also know that $|[0, n]|>0$
we obtain constant potential $2$ that we use to pay for the loop
iteration ($1$ unit) and to establish the loop invariant again ($1$
unit).  In the \emph{else} branch, we use the potential $|[0,vb]|$
and the fact $vb>0$ to obtain $1$ potential units to pay for the
loop iteration.

In example \emph{fig4\_4} it is essential that $m$ is positive.  That
ensures that we can obtain constant potential for the interval size
$|[0,i]|$ in the \emph{else} branch of the conditional since $|[0,i]|$
decreases.  Example \emph{fig4\_5} is an examples that we can not
handle automatically.  The execution is bounded because the boolean
value of the test \code{dir==fwd} does not change during the iteration
of the loop.  As a result, the variable $i$ is either counted down to
$0$ or up to $n$.  Our tool cannot handle example \emph{fig4\_5}
because we don't do path sensitive reasoning.  Note however that it
would be more efficient to move the test \code{dir==fwd} outside of
the loop (this would be also done by an optimizing compiler).  The
resulting program can then be analyzed by our tool.

Example \emph{ex1} from~\cite{GulwaniZ10} specifically focuses on the
code in the \emph{if} statement.  So we use the \emph{tick metric} and
insert \code{tick(1)} inside the if statement to derive a bound on
the number of times the code in the if statement is executed.  Note
that we cannot derive a bound for the whole program since the outer
loop is executed a quadratic number of times.  Nevertheless it is
straightforward to derive a bound on the number of ticks using the
amortized approach: In the if statement we know that $n>0$ and assign
\code{n=n-1}.  So we can use the potential of the interval size $|[0,n]|$
to pay for the tick.

Similar to example \emph{ex1}, example \emph{ex2}
form~\cite{GulwaniZ10} focuses on the number of iterations of the
\emph{outer loop}.  While the whole program is not terminating, the
number of iterations of the outer loop is bounded by $|[0,n]|$.
Finally, example \emph{ex2} is similar to example
\emph{nested\_single}, and \emph{ex3} is a variant of example
\emph{t47}.

\begin{figure*}[t!]
 \setlength{\progwidth}{.28\linewidth}
  \centering
%
%
  \begin{minipage}[b]{\progwidth}
    \begin{center}
   \begin{lstlisting}
void count_down (int x) {
  int a = x;
  if (a>0) {
    a = a-1;
    count_down(a);
  }
}
int copy (int x, int y) {
  if (x>0) {
    x = x-1;
    y = y+1;
    y=copy(x,y);
  };
  return y;
}
void main (int x,int y) {
  y = copy (x,y);
  count_down(y);
}
   \end{lstlisting}

$3 + 2|[0, x]| + |[0, y]|$
\\[.7\baselineskip]
      {\bf t37}
    \end{center}
  \end{minipage}
%
%
  \begin{minipage}[b]{\progwidth}
    \begin{center}
   \begin{lstlisting}
void count_down (int x,int y)
{ int a = x;
  if (a>y) {
    a = a-1;
    count_up(a,y);
  }
}

void count_up (int x, int y)
{ int a = y;
  if (a+1<x) {
    a = a+2;
    count_down(x,a);
  }
}

void main (int y, int z) {
  count_down(y,z);
}
   \end{lstlisting}

$1.33 + 0.67 |[z,y]|$
\\[.7\baselineskip]
      {\bf t39}
    \end{center}
  \end{minipage}
%
%
  \begin{minipage}[b]{\progwidth}
    \begin{center}
   \begin{lstlisting}
void produce () {
  while (x>0) {
    tick(-1); x=x-1; y=y+1;
  }
}
void consume () {
  while (y>0) {
    y=y-1; x=x+1; tick(1);
  }
}
void main (int y, int z) {
  consume(); produce(); consume();
}
   \end{lstlisting}

$|[0, y]|$ ticks
\\[.7\baselineskip]
      {\bf t46}
    \end{center}
  \end{minipage}

   \caption{Programs with (recursive) functions}
  \label{fig:cat3}
\end{figure*}

\begin{figure*}[t!]
 \setlength{\progwidth}{.32\linewidth}
  \centering
%
%
  \begin{minipage}[b]{\progwidth}
    \begin{center}
   \begin{lstlisting}
int srch(
  int t[], int n,  /* haystack */
  int p[], int m,  /* needle */
  int b[]
) {
  int i=0, j=0, k=-1;

  while (i < n) {
    while (j >= 0 && t[i]!=p[j]) {
      k = b[j];
      assert(k > 0);
      assert(k <= j + 1);
      j -= k;
    }
    i++, j++;
    if (j == m)
      break;
  }
  return i;
}
   \end{lstlisting}

$1 + 2|\inter 0 n|$
\\[.7\baselineskip]
      {\bf Knuth-Morris-Pratt}
    \end{center}
  \end{minipage}
%
%
  \begin{minipage}[b]{\progwidth}
    \begin{center}
   \begin{lstlisting}
int gcd(int x, int y) {
  if (x <= 0) return y;
  if (y <= 0) return x;

  for (;;) {
    if (x>y) x -= y;
    else if (y>x) y -= x;
    else return x;
  }
}
   \end{lstlisting}

$|\inter 0 x| + |\inter 0 y|$
\\[.7\baselineskip]
      {\bf Greatest Common Divisor}
    \end{center}
  \end{minipage}
%
%
  \begin{minipage}[b]{\progwidth}
    \begin{center}
   \begin{lstlisting}
void qsort(int a[], int lo, int hi) {
  int m1, m2, n;

  if (hi - lo < 1) return;

  n = nondet(); /* partition the array */
  assert( n > 0 );
  assert( lo + n <= hi );

  m1 = n + lo;
  m2 = m1 - 1;

  qsort(a, m1, hi);
  qsort(a, lo, m2);
}

void main(int a[], int len) {
  qsort(a, 0, len);
}
   \end{lstlisting}

$1 + 2 |\inter 0 {\code{len}}|$
\\[.7\baselineskip]
      {\bf Quick Sort}
    \end{center}
  \end{minipage}

   \caption{Well-Known Algorithms}
  \label{fig:cat3}
\end{figure*}





\paragraph{Recursive Functions}

Our approach can naturally deal with mutually-recursive functions.
The recursion patterns can be exactly the same that are used in
iterations of loops.  In the following, we present three simple
examples that illustrate the analysis of functions.

Example \emph{t37} illustrates that the analyzer is able to perform
inter-procedural size tracking.  The function \code{copy} adds the
argument $x$ to the argument $y$ if $x$ is positive.  However, this
addition is done in steps of $1$ in each recursive call.  The function
\code{count\_down} recursively decrements its argument down to $0$.
The derived bound $3 + 2|[0, x]| +2|[0, y]|$ is for the function
\code{main} in which we first add $x$ to $y$ using the function
\code{copy} and then count down the variable $y$ using the function
\code{count\_down}.  The derived bound is tight.

Example \emph{t39} uses mutual recursion.  The function
\code{count\_down} is similar to the function with the same name in
example \emph{t37}.  However, we do not count down to $0$ but to a
variable $y$ that is passed as an argument and we call the function
\code{count\_up} afterwards.  The function \code{count\_up} is dual to
\code{count\_down}.  Here, we count up $y$ by $2$ and recursively call
\code{count\_down}.  For the function \code{main}, which calls
\code{count\_down(y,z)}, the analyzer computes the tight bound $1.33 +
0.67 |[z,y]|$.

Example \emph{t46} shows a program that uses and returns resources.
Again, we use the \emph{tick metric} and the function \code{tick} to
describe the resource usage.  The function \code{produce} produces
$|[0,x]|$ resources, that is, in each of the $|[0,x]|$ iterations, it
receives one resource unit.  Similarly, the function \code{consume}
consumes $|[0,y]|$ resources.  The analyzer computes the tight bound
$|[0,y]|$ for the function \code{main}.  This is only possible since
amortized analysis naturally tracks the size changes to the variables
$x$ and $y$, and the interaction between \code{consume} and
\code{produce}.


\begin{table*}
\small

\newcommand{\dir}[2]{\texttt{#1} \\ \texttt{#2}}
\newcommand{\file}[2][c]{%
  \begin{tabular}[#1]{@{}r@{}}#2\end{tabular}}
\renewcommand{\max}[0]{{\rm mx}}
\renewcommand{\min}[0]{{\rm mn}}

\newcolumntype{C}[1]{>{\centering\let\newline\\\arraybackslash\hspace{0pt}}m{#1}}

\centering
\begin{tabular}{r|C{1.1cm}c|C{1.8cm}c|C{1.9cm}c|C{1.8cm}|C{2.65cm}}


File &
\multicolumn{2}{|c|}{KoAT} &
\multicolumn{2}{|c|}{Rank} &
\multicolumn{2}{|c|}{LOOPUS} &
SPEED &
AAA
\\
\hline

%--------------------------------------------
\hline \texttt{gcd.c} &

? &
&

% $((((1+1)+1)+((-5+(2{\cdot}y))+(2{\cdot}x)))+(-3+(2{\cdot}y)))+1$ &
$(((2{+}1)\dots$ &
$O(n)$ &

--- &
&

? &

$|\inter 0 x| {+} |\inter 0 y|$
\\

%--------------------------------------------
\hline \texttt{kmp.c} &

? &
&

% $((((1+1)+(n+(-1{\cdot}i)))+((n+(n{\cdot}j))+((-1+(-1{\cdot}j)){\cdot}i)))+((n+(n{\cdot}j))+((-1+(-1{\cdot}j)){\cdot}i)))+1$ &
$(((2{+}(n{+} \dots$ &
$O(n^2)$ &

% $\max(n, 0) + \max(0, (1 + \max(n, 0))) + \max(0, (1 + \max(n, 0)))$ &
$\max(n, 0) \dots$ &
$O(n)$ &

? &

$1 {+} 2 |\inter 0 n|$
\\

%--------------------------------------------
\hline \texttt{qsort.c} &

? &
&

--- &
&

--- &
&

? &

$1 {+} 2 |\inter 0 {\code{len}}|$
\\

%--------------------------------------------
% \hline \file{\dir{speed\_pldi09}{fig1.c}} &
%
% ? &
% &
%
% --- &
% &
%
% $2 \max(n, 0)$ &
% $O(n)$ &
%
% God damn it &
%
% $1 + 2 |\inter 0 n|$
% \\

%--------------------------------------------
\hline \file{\dir{speed\_pldi09}{fig4\_2.c}} &

--- &
&

% $((((1+1)+n)+(n+(-1{\cdot}m)))+1)+1$ &
$(((2{+}n)\dots$ &
$O(n)$ &

--- &
&

$\frac n m + n$ &

$1 {+} 2 |\inter 0 n|$
\\

%--------------------------------------------
\hline \file{\dir{speed\_pldi09}{fig4\_4.c}} &

--- &
&

% $((((1+1)+(-1+m))+((1+n)+(-1{\cdot}m)))+1)+1$ &
$(((2{+}(-1\dots$ &
$O(n)$ &

--- &
&

$\frac n m + m$ &

$|\inter 0 n|$
\\

%--------------------------------------------
\hline \file{\dir{speed\_pldi09}{fig4\_5.c}} &

$28d + 7g + 27$ &
$O(n)$ &

% $((((1+1)+(-1+n))+m)+1)+1$ &
$(((2{+}(-1\dots$ &
$O(n)$ &

--- &
&

$\max(n, n-m)$ &

---
\\

%--------------------------------------------
\hline \file{\dir{speed\_pldi10}{ex1.c}} &

--- &
&

--- &
&

--- &
&

$n$ &

$|\inter 0 n|$
\\

%--------------------------------------------
\hline \file{\dir{speed\_pldi10}{ex3.c}} &

--- &
&

% $((((1+1)+(-1+n))+(-1+n))+1)+1$ &
$(((2{+}(-1\dots$ &
$O(n)$ &

$2{\cdot}\max(n, 0)$ &
$O(n)$ &

$n$ &

$|\inter 0 n|$
\\

%--------------------------------------------
\hline \file{\dir{speed\_pldi10}{ex4.c}} &

$110 a + 33$ &
$O(n)$ &

--- &
&

--- &
&

$n + 1$ &

$1 {+} 2 |\inter 0 n|$
\\

%--------------------------------------------
\hline \file{\dir{speed\_popl10}{fig2\_1.c}} &

$9a + 9b + \dots$ &
$O(n)$ &

% $(((1+1)+((-1{\cdot}y)+m))+((-1{\cdot}x)+n))+1$ &
$((2{+}((-y\dots$ &
$O(n)$ &

$\max(0, n {-} x) + \max(0, m {-} y)$ &
$O(n)$ &

$\max(0, n{-}x) + \max(0, m{-}y)$ &

$|\inter x n| {+} |\inter y m|$
\\

%--------------------------------------------
\hline \file{\dir{speed\_popl10}{fig2\_2.c}} &

$6a + 9b + 3c + 5$ &
$O(n)$ &

% $(((1+1)+((-1{\cdot}x)+n))+(((-1+(-1{\cdot}z))+(-1{\cdot}x))+(2{\cdot}n)))+1$ &
$((2{-}x\dots$ &
$O(n)$ &

% $\max(0, (\max(0, (x {+} 1 {+} -z)) + \max(0, (n - x)))) + \max(0, (n - x))$ &
$\max(0, (x + 1 {-}z)\dots$ &
$O(n)$ &

$\max(0, n{-}x) + \max(0, n{-}z)$ &

$|\inter x n| {+} |\inter z n|$
\\

%--------------------------------------------
\hline \file{\dir{speed\_popl10}{nstd\_multiple.c}} &

--- &
&

% $(((1+1)+((-1{\cdot}x)+n))+(((y{\cdot}x)+((-1{\cdot}y){\cdot}n))+(((-1{\cdot}x)+n){\cdot}m)))+1$ &
$((2{-}x{+}n\dots$ &
$O(n^2)$ &

$\max(0, m {-} y) + \max(0, n {-} x)$ &
$O(n)$ &

$\max(0, m{-}y) + \max(0, n{-}x)$ &

$|\inter x n| {+} |\inter y m|$
\\

%--------------------------------------------
\hline \file{\dir{speed\_popl10}{nstd\_single.c}} &

$48 b + 16$ &
$O(n)$ &

% $((((1+1)+((-1+(-1{\cdot}x))+n))+((-1+(-1{\cdot}x))+n))+1)+1$ &
$(((1{-}x{+}n\dots$ &
$O(n)$ &

$\max(0, \! n {-} 1)\dots$ &
$O(n)$ &

$n$ &

$|\inter 0 n|$
\\

%--------------------------------------------
\hline \file{\dir{speed\_popl10}{sqntl\_single.c}} &

$21 b + 6$ &
$O(n)$ &

% $(((1+1)+((-1{\cdot}x)+n))+((-1{\cdot}x)+n))+1$ &
$((2{}-x{+}n\dots$ &
$O(n)$ &

$2{\cdot}\max(n, 0)$ &
$O(n)$ &

$n$ &

$|\inter 0 n|$
\\

%--------------------------------------------
\hline \file{\dir{speed\_popl10}{smpl\_multiple.c}} &

$9 c + 10 d + 7$ &
$O(n)$ &

% $(((1+1)+((-1{\cdot}y)+m))+((-1{\cdot}x)+n))+1$ &
$((2{-}y{+}m\dots$ &
$O(n)$ &

$\max(n, 0) + \max(m, 0)$ &
$O(n)$ &

$n + m$ &

$|\inter 0 m| {+} |\inter 0 n|$
\\

%--------------------------------------------
\hline \file{\dir{speed\_popl10}{smpl\_single2.c}} &

$20 d + 12 c + 17$ &
$O(n)$ &

--- &
&

$\max(n, 0) + \max(m, 0)$ &
$O(n)$ &

$n + m$ &

$|\inter 0 n| {+} |\inter 0 m|$
\\

%--------------------------------------------
\hline \file{\dir{speed\_popl10}{smpl\_single.c}} &

$4 b + 6$ &
$O(n)$ &

% $(((1+1)+((-1{\cdot}x)+n))+((-1{\cdot}x)+n))+1$ &
$((2{-}x{+}n\dots$ &
$O(n)$ &

$\max(n, 0)$ &
$O(n)$ &

$n$ &

$|\inter 0 n|$
\\

%--------------------------------------------
\hline \file{\texttt{t07.c}} &

? &
&

$2+x$ &
$O(n)$ &

% $\max(x, 0) + \max(0, (y +  2 {\cdot} \max(x, 0))) + \max(0, ( 2 {\cdot} \max(x, 0) + \max(y, 0)))$ &
$\max(x, 0)\dots$ &
$O(n)$ &

? &

$1 {+} 3 |\inter 0 x| {+} |\inter 0 y|$
\\

%--------------------------------------------
\hline \texttt{t08.c} &

? &
&

% $(((1+1)+-1)+(z+(-1{\cdot}y)))+1$ &
$((2{+}z{-}y\dots$ &
$O(n)$ &

% $\max(0, (\max(0, (y + -2)) + \max(0, (z - y)))) + \max(0, (z - y))$ &
$\max(0, \! y{-}2)\dots$ &
$O(n)$ &

? &

$1.33 |\inter y z| {+} 0.33 |\inter 0 y|$
\\

%--------------------------------------------
\hline \texttt{t10.c} &

? &
&

% $(((1+1)+((-1{\cdot}y)+x))+((-1{\cdot}y)+x))+1$ &
$((2{-}y{+}x\dots$ &
$O(n)$ &

$\max(0, x {-} y)$ &
$O(n)$ &

? &

$|\inter y x|$
\\

%--------------------------------------------
\hline \texttt{t11.c} &

? &
&

% $(((1+1)+((-1{\cdot}y)+m))+((-1{\cdot}x)+n))+1$ &
$((2{-}y{+}m\dots$ &
$O(n)$ &

$\max(0, n {-} x)) + \max(0, m {-} y)$ &
$O(n)$ &

? &

$|\inter x n| {+} |\inter y m|$
\\

%--------------------------------------------
\hline \texttt{t13.c} &

? &
&

% $((((1+1)+-1)+(((((-1/2){\cdot}y)+((1/2){\cdot}(y{\cdot}y)))+(((-1/2)+y){\cdot}x))+((1/2){\cdot}(x{\cdot}x))))+-1)+1$ &
$(((1{+}y^2/2\dots$ &
$O(n^2)$ &

$2{\cdot}\max(x, 0) + \max(y, 0)$ &
$O(n)$ &

? &

$2 |\inter 0 x| {+} |\inter 0 y|$
\\

%--------------------------------------------
\hline \texttt{t15.c} &

? &
&

% $(((1+1)+(-1+x))+((-1{\cdot}y)+x))+1$ &
$((1{+}x\dots$ &
$O(n)$ &

--- &
&

? &

$|\inter 0 x|$
\\

%--------------------------------------------
\hline \texttt{t16.c} &

? &
&

% $(((1+1)+((-1+(-99{\cdot}y))+(101{\cdot}x)))+((-1{\cdot}y)+x))+1$ &
$((-99{\cdot}y\dots$ &
$O(n)$ &

--- &
&

? &

$101 |\inter 0 x|$
\\

%--------------------------------------------
\hline \texttt{t19.c} &

? &
&

% $(((1+1)+(151+k))+(-100+i))+1$ &
$((153{+}k\dots$ &
$O(n)$ &

$\max(0, \! i{-}10^2) \!+ \max(0, \! k {+} i {+} 51)$ &
$O(n)$ &

? &

$50 {+} |\inter {-1} i| {+} |\inter 0 k|$
\\

%--------------------------------------------
\hline \texttt{t20.c} &

? &
&

% $((1+1)+((-1{\cdot}y)+x))+1+((1+1)+(y+(-1{\cdot}x)))+1$ &
$(2{-}y{+}x\dots$ &
$O(n)$ &

% $\max(0, (\max(0, (y - x)) + \max(0, (x - y)))) + \max(0, (y - x))$ &
$2 {\cdot} \max(0, \! y {-} x) \! + \max(0, \! x {-} y)$ &
$O(n)$ &

? &

$|\inter x y| {+} |\inter y x|$
\\

%--------------------------------------------
\hline \texttt{t27.c} &

? &
&

--- &
&

% $\max(0, (1000 +  1000 {\cdot} \max(0, -n) + \max(0, (y + -99)))) + \max(0, -n)$ &
$10^3 \max(0, \discretionary{}{}{} -n) \! \dots$ &
$O(n)$ &

? &

$0.01|\inter n y| {+} 11 |\inter n 0|$
\\

%--------------------------------------------
\hline \texttt{t28.c} &

? &
&

% $(((1+1)+-1)+((-1{\cdot}y)+x))+1$ &
$((1{-}y{+}x\dots$ &
$O(n)$ &

% $\max(0, ( 1000 {\cdot} \max(0, (x - y)) + \max(y, 0))) + \max(0, (x - y)) + \max(0, -x)$ &
$10^3 \,\max(0, x - y)\dots$ &
$O(n)$ &

? &

$|\inter x 0| {+} |\inter 0 y| \discretionary{}{}{} {+} 1002 |\inter y x|$
\\

%--------------------------------------------
\hline \texttt{t30.c} &

? &
&

--- &
&

--- &
&

? &

$|\inter 0 x| {+} |\inter 0 y|$
\\

%--------------------------------------------
\hline \texttt{t37.c} &

? &
&

--- &
&

--- &
&

? &

$3 {+} 2 |\inter 0 x| {+} |\inter 0 y|$
\\

%--------------------------------------------
\hline \texttt{t39.c} &

? &
&

--- &
&

--- &
&

? &

$1.33 {+} 0.67 |\inter z y|$
\\

%--------------------------------------------
\hline \texttt{t46.c} &

? &
&

--- &
&

--- &
&

? &

$|\inter 0 y|$
\\

%--------------------------------------------
\hline \texttt{t47.c} &

? &
&

$4+n$ &
$O(n)$ &

$1 + \max(n, 0)$ &
$O(n)$ &


? &

$1 {+} |\inter 0 n|$
\\

%--------------------------------------------
\end{tabular}
\caption{
  Experimental evaluation comparing the bounds generated KoAT, Rank,
  LOOPUS, SPEED, and our automatic amortized analysis on several
  challenging linear examples. Results for KoAT and SPEED were extracted
  from previous publications because KoAT cannot take C programs as
  input in its current version and SPEED is not available.
  Entries marked with ? indicate that we cannot test the respective example with the tool. Entries
  marked with --- indicate that the tool failed to produce a result.
}
\label{tab:eval}
\end{table*}

\paragraph{Well-Known Algorithms}

\pref{fig:cat3} shows well-known algorithms that can be automatically
analyzed in our framework.  Example \emph{Knuth-Morris-Pratt} shows
the search function of the Knuth-Morris-Pratt algorithm for string
search.  To derive a bound for this algorithm we have add an assertion
that indicates the bounds of the values that are stored in the array
\code{b}.  Using this information we derive a tight linear bound.

Example \emph{Greatest Common Divisor} is the usual implementation of
the GCD algorithm.  We automatically derive a linear bound.  Note that
the two tests at the beginning that check if the inputs are positive
are essential.  Finally, example \emph{Quick Sort} shows a skeleton of
the quick sort sorting algorithm.  Since we can only derive linear
bounds we left out the inner loop that swaps the array elements and
determines the position \code{n+lo} of the pivot.  We only assert that
\code{lo < n+lo <= hi}.  We then derive a tight linear bound.

}\fi

\bibliographystyle{abbrvnat}
\iffull{
\bibliography{lit}
}
{
\sectskip
\scriptsize
\bibliography{lit}
}





\end{document}

%%% Local Variables:
%%% mode: latex
%%% mode: flyspell
%%% TeX-master: t
%%% End:
