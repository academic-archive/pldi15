\section{Polynomial Potential}

\paragraph{Index Sets}

Let $V$ be a set of variables.  An \emph{index} $I \in \ind(V)$ is a
family that maps two-element sets of variables to natural numbers,
that is,
$$
I = (i_{\{x,y\}})_{\{x,y\} \subseteq V} \; .
$$
%
We identify a family $I$ with the set $\{ (\{x,y\},i_{\{x,y\}})
\mid \{x,y\} \subseteq V\}$.

Let $\ind(V)$ denote the set of all such indices.  We write $\ind$
instead of $\ind(V)$ if the set of variables $V$ is fixed or obvious
from the context.
%
We assume that allways $0 \in V$ and sometimes write $i_x$ instead of $i_{\{x,0\}}$.

The \emph{degree} $\deg(I)$ of an index $I = (i_{\{x,y\}})_{\{x,y\}
  \subseteq V}$ is defined as
$$
\deg(I) = \sum_{\{x,y\} \in V} i_{\{x,y\}} \;.
$$
We define $\ind_k(V) = \{ I \mid I \in \ind(V) \text{ and } \deg(i) \leq k
\}$ to be the set of indices of degree at most $k$.

\paragraph{Resource Polynomials}

Let $V$ be a set of variables.  An index $I \in \ind(V)$ denotes a
\emph{base polynomial} $P_I : \states \to \N$ for $V$ that maps a
program state $H$ to product of binomial coefficients (a natural
number).  We define
$$
P_I(H) = \prod_{{\{x,y\}} \subseteq V} \binom {|H(x){-}H(y)|} {i_{\{x,y\}}} \; .
$$
%
A \emph{resource polynomial} $R$ for the variable set $V$ is a
non-negative linear combination of the base polynomials for $V$.

\paragraph{Potential Annotations}

A \emph{potential annotation} for the variable set $V$ is a family
$$Q = (q_I)_{I \in \ind(V)}$$
of non-negative rational numbers.  Such an annotation denotes the
resource polynomial $R_Q$ that is defined by
$$
R_Q(H) = \sum_{I \in \ind(V)} q_I \cdot P_I(H) \; .
$$
%
We say that the potential annotation $Q$ is of degree $k$ if $q_I = 0$
for $I \in \ind(V)$ with $\deg(I) > k$.

\paragraph{Additive Shifts}

Let $Q$ be a potential annotation for a variable set $V$ and let
$\{x,y\} \subseteq V$ be a two-element variable set.  The
\emph{additive shift} with respect to $\{x,y\}$ is a potential
annotation $\shift_{\{x,y\}}(Q) = (q'_I)_{I \in \ind(V)} $ for $V$
that is defined through
$$
q'_I = q_I + q_{I^{\{x,y\}{+}1}} \; .
$$
For an index $I = (i_{\{x,y\}})_{\{x,y\} \subseteq V}$ we use the
notation $I^{\{x,y\}{+}k}$ to denote the index
$(i'_{\{x,y\}})_{\{x,y\} \subseteq V}$ such that
$$
i'_{\{t,u\}} = \left\{
  \begin{array}{ll}
    i_{\{t,u\}} + k  & \text{if } \{t,u\} = \{x,y\} \\
    i_{\{t,u\}} & \text{otherwise}
  \end{array}
\right.
\;.
$$
%
The additive shift for natural numbers reflects the identity
\begin{equation}
\label{eq:shift}
\sum_{0 {\leq} i \leq {k}} q_i \binom{n+1}{i} = \sum_{0 {\leq} i \leq {k}} (q_i{+}q_{i+1}) \binom{n}{i}
\end{equation}
where $q_{k+1} = 0$.  It is used in the effect system if the
difference $n+1$ between two variables $x,y$ decreases by one.

\begin{lemma} Let $V$ be a set of variables with $x,y \in V$ and let
  $H$ be a program state. Let $|H'(t) {-} H'(u)| = |H(t) {-} H(u)|$
  for $\{t,u\} \neq \{x,y\}$ and let $|H'(x) {-} H'(y)| = |H(x) {-}
  H(y)| - 1$.
  %
  If $Q' = \shift_{\{x,y\}}(Q)$ then $R_Q(H) = R_{Q'}(H')$.
\end{lemma}

We now study the effect of multiple simultaneous shifts.  Let $Q$ be a
resource annotation for a variable set $V$ and let $U_1,\ldots,U_n
\subseteq V$ with $|U_i| = 2$ for all $i$ and $U_i \neq U_j$ for $i
\neq j$ be pairwise distinct two-element variable sets.  The
simulations additive shift $\shift_{U_1,\ldots,U_n}(Q)$ of $Q$ with
respect to $U_1,\ldots,U_n$ is defined by
$$
\shift_{U_1,\ldots,U_n}(Q) = \shift_{U_1}( \cdots \shift_{U_n}(Q) \cdots ) \; .
$$
%
\begin{proposition}
  Let $V$ be a set of variables and let $U_1,\ldots,U_n$ be pairwise
  distinct two-element variable sets.  Let $|H'(x) {-} H'(y)| = |H(x)
  {-} H(y)|$ for $\{x,y\} \not\in \{U_1,\ldots,U_n\}$ and let $|H'(x)
  {-} H'(y)| = |H(x) {-} H(y)| - 1$ for $\{x,y\} \in
  \{U_1,\ldots,U_n\}$.
  %
  If $Q' = \shift_{\{U_1,\ldots,U_n\}}(Q)$ then $R_Q(H) = R_{Q'}(H')$.
\end{proposition}
%
As shown by the following lemma, the order in which the shifts for the
individual $U_i$ are applied is insignificant.
%
\begin{lemma}
  Let $\sigma : \{1,\ldots,n\} \to \{1,\ldots,n\}$ be a
  permutation. Then $\shift_{U_1,\ldots,U_n}(Q) =
  \shift_{U_{\sigma(1)},\ldots,U_{\sigma(n)}}(Q)$.
\end{lemma}
%
For reasons of efficiency in the constraint generation, we give a more
direct formula for the simultaneous shift.  Let $I \in \ind(V)$ and
let $U_1,\ldots,U_n$ be pairwise distinct two-element variable sets.
We define the index $I^{U_1,\ldots,U_n + k}$ as the family $(i'_{\{x,y\}})_{\{x,y\} \subseteq V}$ such that
$$
i'_{\{t,u\}} = \left\{
  \begin{array}{ll}
    i_{\{t,u\}} + k  & \text{if } \{t,u\} \in \{U_1,\ldots,U_n\} \\
    i_{\{t,u\}} & \text{otherwise}
  \end{array}
\right.  \;.
$$

%
\begin{lemma}
  Let $V$ be a variable set and let $U_1,\ldots,U_n$ be pairwise
  distinct two-element variable sets.
  %
  Let $Q = (q_I)_{I \in \ind(V)}$ be a resource annotation for
  $V$ and let $ Q' = (q'_I)_{I \in \ind(V)}$ where
  $$
  q'_I = \sum_{\{j_1,\ldots,j_m\} \subseteq \{1,\ldots,n\} } q_{I^{U_{j_1},\ldots,U_{j_m}+1}} \; .
  $$
  Then $Q' = \shift_{U_1,\ldots,U_n}(Q)$.
\end{lemma}




%%
% Note: Later we need to shift in many directions at once like
%   Q' = shift_{x,y} (shift_{x,u} (Q))
% To do: Give a combined formula for that (concise constraint system).
%%
