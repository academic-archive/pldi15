\documentclass[nocopyrightspace,preprint,pldi]{sigplanconf-pldi15}

\usepackage[utf8]{inputenc} %for utf8 input
\usepackage{amssymb} %for shift symbol
\usepackage{amsmath}
\usepackage{listings} %for code
\usepackage{mathpartir} %for typing rules
\usepackage{microtype} %better micro typing
\usepackage{stmaryrd} %for llbracket
\usepackage{mathabx} % for boxes
\usepackage{graphicx} %to include png images
\usepackage{xcolor} %for colors
\usepackage{url}
\usepackage{enumitem}
\usepackage{array} %for stupid tables

%----------------------------------------------------

\usepackage{prettyref}
\newcommand{\pref}[1]{\prettyref{#1}}
\newcommand{\Pref}[1]{\prettyref{#1} \vpageref[]{#1}}
\newcommand{\ppref}[1]{\vpageref[]{#1}}
\newrefformat{fig}{Figure~\ref{#1}}
\newrefformat{app}{Appendix~\ref{#1}}
\newrefformat{tab}{Table~\ref{#1}}
\newrefformat{cha}{Challenge~\ref{#1}}
\newrefformat{compiler}{Point~\ref{#1} of \pref{thm:compiler}}

%----------------------------------------------------

\usepackage{amsthm}
\newtheorem{lemma}{Lemma}
\newtheorem{corollary}{Corollary}
\newtheorem{theorem}{Theorem}
\newtheorem{proposition}{Proposition}

%----------------------------------------------------

\lstset{
   numbers=none, %use numbers=left
   numberstyle=\tiny,
   stepnumber=1,
   numbersep=5pt,
   basicstyle=\tt\small,
   escapechar=\#,
   mathescape,
%   language=CaML,
   columns=flexible,
%   basewidth=0.455em,
   xleftmargin=\leftmargini,
%   xrightmargin=\leftmargini,
%   morekeywords={mod,div,matchD},
   deletekeywords={as},
   literate={<-}{$\;\leftarrow\;\;\,$}{3} {->}{$\;\rightarrow\;\;\,$}{3},
   firstnumber=auto % use name=NAME to identify split listings
%
% Placement: belowskip,aboveskip,lineskip,boxpos=l|c|r
%
% wonna figure-style listings? Use: caption={Useless code,label=useless
%
% keywordstyle=\color{black}\bfseries\underbar
% morekeywords={one,two,three,four}
}
% \lstinline!! 
% use basicstyle=\small ?

%How many languages are there in CompCert?
\newcommand{\howmanylanguages}[0]{11}

%How many passes are there in CompCert?
\newcommand{\howmanypasses}[0]{20}

\newcommand{\lines}[1]{\textsl{{\scriptsize #1}}}

\newcommand{\toolname}[0]{\ensuremath{C^4\!B} }

\newcommand{\valid}[6]{\ensuremath{\mathit{valid}(#1,#2,#3,#4,#5,#6)}}
\newcommand{\validC}[2]{\ensuremath{\mathit{validC}(#1,#2)}}
\newcommand{\safe}[4]{\ensuremath{\mathit{safe}(#1,#2,#3,#4)}}
\newcommand{\safeK}[5]{\ensuremath{\mathit{safeK}(#1,#2,#3,#4,#5)}}

\newcommand{\refines}[0]{\ensuremath{\prec}}

\newcommand{\qrefines}[0]{\ensuremath{\mathop{\prec_Q}}}

\newcommand{\pruned}[1]{\ensuremath{\overline{ #1} }}

\newcommand{\sem}[1]{\ensuremath{\llbracket #1 \rrbracket}}

\newcommand{\progs}[0]{\ensuremath{\mathcal{P}}}

\newcommand{\id}[0]{\ensuremath{\mathit{x}}}

\newcommand{\xevent}[0]{\ensuremath{\nu}}
\newcommand{\cevent}[0]{\ensuremath{\mu}}
\newcommand{\event}[0]{\ensuremath{\iota}}

\newcommand{\evalue}[0]{\ensuremath{v}}

\newcommand{\intval}[1]{\ensuremath{\mathsf{int}(#1)}}

\newcommand{\floatval}[1]{\ensuremath{\mathsf{float}(#1)}}

\newcommand{\call}[1]{\ensuremath{\mathsf{call}(#1)}}
\newcommand{\return}[1]{\ensuremath{\mathsf{ret}(#1)}}
\newcommand{\malloc}[1]{\ensuremath{\mathsf{malloc}(#1)}}
\newcommand{\free}[1]{\ensuremath{\mathsf{free}(#1)}}

\newcommand{\trace}[0]{\ensuremath{t}}

\newcommand{\Trace}[0]{\ensuremath{T}}

\newcommand{\conv}[2]{\ensuremath{\mathsf{conv}(#1,#2)}}

\newcommand{\divt}[1]{\ensuremath{\mathsf{div}(#1)}}

\newcommand{\fail}[1]{\ensuremath{\mathsf{fail}(#1)}}

\newcommand{\behav}[0]{\ensuremath{B}}

\newcommand{\getchar}[0]{\ensuremath{\mathsf{getchar}}}

\newcommand{\Z}[0]{\ensuremath{\mathbb{Z}}}
\newcommand{\N}[0]{\ensuremath{\mathbb{N}}}
\newcommand{\Qplusz}[0]{\ensuremath{\mathbb Q^+_0}}
\newcommand{\Q}[0]{\ensuremath{\mathbb{Q}}}

\newcommand{\weight}[2]{\ensuremath{W_{#2}(#1)}}
\newcommand{\tval}[2]{\ensuremath{V_{#2}(#1)}}

\newcommand{\prefs}[1]{\ensuremath{\mathit{prefs}(#1)}}

\newcommand{\type}[1]{\ensuremath{\mathsf{#1}}}

\newcommand{\code}[1]{\ensuremath{\mathsf{#1}}}

\newcommand{\env}[0]{\ensuremath{\mathit{\theta}}}

\newcommand{\mem}[0]{\ensuremath{\mathit{H}}}

\newcommand{\Mem}[0]{\ensuremath{\mathit{Mem}}}

\newcommand{\Var}[0]{\ensuremath{\mathit{Var}}}

\newcommand{\Val}[0]{\ensuremath{\mathit{Val}}}

\newcommand{\Aux}[0]{\ensuremath{\mathit{Aux}}}

\newcommand{\Input}[0]{\ensuremath{\mathit{StdIn}}}

\newcommand{\Assn}[0]{\ensuremath{\mathit{Assn}}}
\newcommand{\State}[0]{\ensuremath{\mathit{State}}}
\newcommand{\state}[0]{\ensuremath{\sigma}}

\newcommand{\Prop}[0]{\ensuremath{\mathit{Prop}}}

\newcommand{\ret}[0]{\ensuremath{\mathit{ret}}}

\newcommand{\htriple}[3]{\ensuremath{\{#1\}\, #2\, \{#3\}}}

\newcommand{\smallstep}[1]{\ensuremath{\to_{#1}}}

\newcommand{\conform}[2]{\ensuremath{\mathit{agree}(#1,#2)}}

\newcommand{\dom}[1]{\ensuremath{ \text{dom}(#1)}}
\newcommand{\img}[1]{\ensuremath{ \text{img}(#1)}}

\newcommand{\pimp}[2]{\ensuremath{#1 \Rightarrow #2}}

\newcommand{\Assert}[0]{\ensuremath{\mathit{Assert}}}

\newcommand{\VAssert}[0]{\ensuremath{\mathit{VAssert}}}

\newcommand{\eenv}[0]{\ensuremath{\Sigma}}

\newcommand{\genv}[0]{\ensuremath{\Delta}}

\newcommand{\fenv}[0]{\ensuremath{\Sigma}}

\newcommand{\Fundef}[0]{\ensuremath{\mathit{Fundef}}}

\newcommand{\funs}[0]{\ensuremath{\mathit{FID}}}

\newcommand{\vars}[0]{\ensuremath{\mathit{VID}}}

\newcommand{\statements}[0]{\ensuremath{\mathcal{S}}}

\newcommand{\traces}[0]{\ensuremath{\mathcal{T}}}

\newcommand{\behavs}[0]{\ensuremath{\mathcal{B}}}

\newcommand{\events}[0]{\ensuremath{\mathcal{E}}}

\newcommand{\initState}[2]{\ensuremath{\mathit{initSt}(#1,#2)}}

\newcommand{\retState}[1]{\ensuremath{\mathit{retState}(#1)}}

\newcommand{\Potential}[0]{\ensuremath{\mathit{Pot}}}

\newcommand{\evalE}[2]{\ensuremath{\llbracket #1 \rrbracket_{#2}}}

\newcommand{\Kseq}[2]{\ensuremath{\mathsf{Kseq}\, #1 \, #2}}

\newcommand{\Kloop}[2]{\ensuremath{\mathsf{Kloop}\, #1 \, #2}}

\newcommand{\Kstop}[0]{\ensuremath{\mathsf{Kstop}}}

\newcommand{\Kcall}[3]{\ensuremath{\mathsf{Kcall}\, #1 \, #2 \, #3}}

\newcommand{\cont}[0]{\ensuremath{K}}

\newcommand{\cost}[0]{\ensuremath{c}}

\newcommand{\used}[0]{\ensuremath{\sqbullet}}

\newcommand{\init}[0]{\ensuremath{\square}}

\newcommand{\FV}[1]{\ensuremath{\text{FV}(#1)}}

\newcommand{\loc}[0]{\ensuremath{\text{Loc}}}

\newcommand{\inter}[2]{\ensuremath{[#1,#2]}}

\newcommand{\ind}[0]{\ensuremath{\mathcal{I}}}

\newcommand{\states}[0]{\ensuremath{\mathcal{H}}}

\newcommand{\shift}[0]{{\lhd}}

\newcommand{\Vret}[0]{{\ensuremath{\mathit{ret}}}}
\newcommand{\Vargs}[0]{{\ensuremath{\vec {\mathit{args}}}}}

\newcommand{\tr}[0]{\ensuremath{\mathcal T}}

%-------------------------------------------------------
%Type rules (mathpatir)

\newcommand{\Rule}[4][]{\ensuremath{\inferrule*[right={\!(#2)},#1]{#3}{#4}}}
\newcommand{\RuleToplabel}[4][]{\ensuremath{\inferrule[(#2)]{#3}{#4}}}
\newcommand{\RuleNolabel}[3][]{\ensuremath{\inferrule*[#1]{#2}{#3}}}

%-------------------------------------------------------


%%% Local Variables: 
%%% mode: latex
%%% TeX-master: "main"
%%% End: 

\begin{document}

\section{Polynomial Potential}

\paragraph{Index Sets}

Let $V$ be a set of variables.  An \emph{index} $I \in \ind(V)$ is a
family that maps two-element sets of variables to natural numbers,
that is,
$$
I = (i_{\{x,y\}})_{\{x,y\} \subseteq V} \; .
$$
%
We identify a family $I$ with the set $\{ (\{x,y\},i_{\{x,y\}})
\mid \{x,y\} \subseteq V\}$.

Let $\ind(V)$ denote the set of all such indices.  We write $\ind$
instead of $\ind(V)$ if the set of variables $V$ is fixed or obvious
from the context.
%
We assume that allways $0 \in V$ and sometimes write $i_x$ instead of $i_{\{x,0\}}$.

The \emph{degree} $\deg(I)$ of an index $I = (i_{\{x,y\}})_{\{x,y\}
  \subseteq V}$ is defined as
$$
\deg(I) = \sum_{\{x,y\} \in V} i_{\{x,y\}} \;.
$$
We define $\ind_k(V) = \{ I \mid I \in \ind(V) \text{ and } \deg(i) \leq k
\}$ to be the set of indices of degree at most $k$.

\paragraph{Resource Polynomials}

Let $V$ be a set of variables.  An index $I \in \ind(V)$ denotes a
\emph{base polynomial} $P_I : \states \to \N$ for $V$ that maps a
program state $H$ to product of binomial coefficients (a natural
number).  We define
$$
P_I(H) = \prod_{{\{x,y\}} \subseteq V} \binom {|H(x){-}H(y)|} {i_{\{x,y\}}} \; .
$$
%
A \emph{resource polynomial} $R$ for the variable set $V$ is a
non-negative linear combination of the base polynomials for $V$.

\paragraph{Potential Annotations}

A \emph{potential annotation} for the variable set $V$ is a family
$$Q = (q_I)_{I \in \ind(V)}$$
of non-negative rational numbers.  Such an annotation denotes the
resource polynomial $R_Q$ that is defined by
$$
R_Q(H) = \sum_{I \in \ind(V)} q_I \cdot P_I(H) \; .
$$
%
We say that the potential annotation $Q$ is of degree $k$ if $q_I = 0$
for $I \in \ind(V)$ with $\deg(I) > k$.

\paragraph{Additive Shifts}

Let $Q$ be a potential annotation for a variable set $V$ and let
$\{x,y\} \subseteq V$ be a two-element variable set.  The
\emph{additive shift} with respect to $\{x,y\}$ is a potential
annotation $\shift_{\{x,y\}}(Q) = (q'_I)_{I \in \ind(V)} $ for $V$
that is defined through
$$
q'_I = q_I + q_{I^{\{x,y\}{+}1}} \; .
$$
For an index $I = (i_{\{x,y\}})_{\{x,y\} \subseteq V}$ we use the
notation $I^{\{x,y\}{+}k}$ to denote the index
$(i'_{\{x,y\}})_{\{x,y\} \subseteq V}$ such that
$$
i'_{\{t,u\}} = \left\{
  \begin{array}{ll}
    i_{\{t,u\}} + k  & \text{if } \{t,u\} = \{x,y\} \\
    i_{\{t,u\}} & \text{otherwise}
  \end{array}
\right.
\;.
$$
%
The additive shift for natural numbers reflects the identity
\begin{equation}
\label{eq:shift}
\sum_{0 {\leq} i \leq {k}} q_i \binom{n+1}{i} = \sum_{0 {\leq} i \leq {k}} (q_i{+}q_{i+1}) \binom{n}{i}
\end{equation}
where $q_{k+1} = 0$.  It is used in the effect system if the
difference $n+1$ between two variables $x,y$ decreases by one.

\begin{lemma} Let $V$ be a set of variables with $x,y \in V$ and let
  $H$ be a program state. Let $|H'(t) {-} H'(u)| = |H(t) {-} H(u)|$
  for $\{t,u\} \neq \{x,y\}$ and let $|H'(x) {-} H'(y)| = |H(x) {-}
  H(y)| - 1$.
  %
  If $Q' = \shift_{\{x,y\}}(Q)$ then $R_Q(H) = R_{Q'}(H')$.
\end{lemma}

We now study the effect of multiple simultaneous shifts.  Let $Q$ be a
resource annotation for a variable set $V$ and let $U_1,\ldots,U_n
\subseteq V$ with $|U_i| = 2$ for all $i$ and $U_i \neq U_j$ for $i
\neq j$ be pairwise distinct two-element variable sets.  The
simulations additive shift $\shift_{U_1,\ldots,U_n}(Q)$ of $Q$ with
respect to $U_1,\ldots,U_n$ is defined by
$$
\shift_{U_1,\ldots,U_n}(Q) = \shift_{U_1}( \cdots \shift_{U_n}(Q) \cdots ) \; .
$$
%
\begin{proposition}
  Let $V$ be a set of variables and let $U_1,\ldots,U_n$ be pairwise
  distinct two-element variable sets.  Let $|H'(x) {-} H'(y)| = |H(x)
  {-} H(y)|$ for $\{x,y\} \not\in \{U_1,\ldots,U_n\}$ and let $|H'(x)
  {-} H'(y)| = |H(x) {-} H(y)| - 1$ for $\{x,y\} \in
  \{U_1,\ldots,U_n\}$.
  %
  If $Q' = \shift_{\{U_1,\ldots,U_n\}}(Q)$ then $R_Q(H) = R_{Q'}(H')$.
\end{proposition}
%
As shown by the following lemma, the order in which the shifts for the
individual $U_i$ are applied is insignificant.
%
\begin{lemma}
  Let $\sigma : \{1,\ldots,n\} \to \{1,\ldots,n\}$ be a
  permutation. Then $\shift_{U_1,\ldots,U_n}(Q) =
  \shift_{U_{\sigma(1)},\ldots,U_{\sigma(n)}}(Q)$.
\end{lemma}
%
For reasons of efficiency in the constraint generation, we give a more
direct formula for the simultaneous shift.  Let $I \in \ind(V)$ and
let $U_1,\ldots,U_n$ be pairwise distinct two-element variable sets.
We define the index $I^{U_1,\ldots,U_n + k}$ as the family $(i'_{\{x,y\}})_{\{x,y\} \subseteq V}$ such that
$$
i'_{\{t,u\}} = \left\{
  \begin{array}{ll}
    i_{\{t,u\}} + k  & \text{if } \{t,u\} \in \{U_1,\ldots,U_n\} \\
    i_{\{t,u\}} & \text{otherwise}
  \end{array}
\right.  \;.
$$

%
\begin{lemma}
  Let $V$ be a variable set and let $U_1,\ldots,U_n$ be pairwise
  distinct two-element variable sets.
  %
  Let $Q = (q_I)_{I \in \ind(V)}$ be a resource annotation for
  $V$ and let $ Q' = (q'_I)_{I \in \ind(V)}$ where
  $$
  q'_I = \sum_{\{j_1,\ldots,j_m\} \subseteq \{1,\ldots,n\} } q_{I^{U_{j_1},\ldots,U_{j_m}+1}} \; .
  $$
  Then $Q' = \shift_{U_1,\ldots,U_n}(Q)$.
\end{lemma}

\subsection{Binomial basis}

\begin{lemma}
  There exists a family $(c^{ij}_k)_k$, such that
  $$
  \sum_k c^{ij}_k \binom n k = \binom n i \binom n j,
  $$
  and $c^{ij}_k = \binom k j \binom j {i+j - k}$ satisfies
  the previous equation.
\end{lemma}
\begin{proof}
  \begin{align*}
    \binom n i \binom n j
      &= \sum_k \binom {n-j} k \binom j {i-k} \binom n j \\
      &= \sum_k \binom {n-j} {k-j} \binom n j \binom j {i+j-k} \\
      &= \sum_k \binom n k \binom k j \binom j {i+j-k}.
  \end{align*}
  In the first equality, we use Vandermonde's convolution
  formula on $\binom n i$.  In the second equality we change
  the summation variable to $k-j$.  And in the third equation
  we use the identity proved below.
  \begin{align*}
    \binom {n-j} {k-j} \binom n j
      &= \frac {(n-j)!} {(n-k)! (k-j)!} \; \frac {n!} {(n-j)! j!} \\
      &= \frac {n!} {(n-k)! (k-j)! j!} \\
      &= \frac {n!} {(n-k)! k!} \; \frac {k!} {(k-j)! j!} \\
      &= \binom n k \binom k j.
  \end{align*}
\end{proof}



%%
% Note: Later we need to shift in many directions at once like
%   Q' = shift_{x,y} (shift_{x,u} (Q))
% To do: Give a combined formula for that (concise constraint system).
%%

\end{document}
