
\documentclass[letterpaper,11pt]{article}

% \documentclass[preprint]{sigplanconf}
\usepackage[numbers]{natbib}

\usepackage[utf8]{inputenc} %for utf8 input
\usepackage{amssymb} %for shift symbol
\usepackage{amsmath}
\usepackage{listings} %for code
\usepackage{mathpartir} %for typing rules
%\usepackage{graphicx} %to include png images
\usepackage{microtype} %better micro typing
\usepackage[standard,amsmath,thmmarks]{ntheorem}
\usepackage{mathabx} %for blackdiamond

\usepackage{prettyref}
\newcommand{\pref}[1]{\prettyref{#1}}
\newcommand{\Pref}[1]{\prettyref{#1} \vpageref[]{#1}}
\newcommand{\ppref}[1]{\vpageref[]{#1}}


\lstset{
   numbers=none, %use numbers=left
   numberstyle=\tiny,
   stepnumber=1,
   numbersep=5pt,
   basicstyle=\tt\small,
   escapechar=\#,
   mathescape,
%   language=CaML,
   columns=flexible,
%   basewidth=0.455em,
   xleftmargin=\leftmargini,
%   xrightmargin=\leftmargini,
%   morekeywords={mod,div,matchD},
   deletekeywords={as},
   literate={<-}{$\;\leftarrow\;\;\,$}{3} {->}{$\;\rightarrow\;\;\,$}{3},
   firstnumber=auto % use name=NAME to identify split listings
%
% Placement: belowskip,aboveskip,lineskip,boxpos=l|c|r
%
% wonna figure-style listings? Use: caption={Useless code,label=useless
%
% keywordstyle=\color{black}\bfseries\underbar
% morekeywords={one,two,three,four}
}
% \lstinline!! 
% use basicstyle=\small ?

%How many languages are there in CompCert?
\newcommand{\howmanylanguages}[0]{11}

%How many passes are there in CompCert?
\newcommand{\howmanypasses}[0]{20}

\newcommand{\lines}[1]{\textsl{{\scriptsize #1}}}

\newcommand{\toolname}[0]{\ensuremath{C^4\!B} }

\newcommand{\valid}[6]{\ensuremath{\mathit{valid}(#1,#2,#3,#4,#5,#6)}}
\newcommand{\validC}[2]{\ensuremath{\mathit{validC}(#1,#2)}}
\newcommand{\safe}[4]{\ensuremath{\mathit{safe}(#1,#2,#3,#4)}}
\newcommand{\safeK}[5]{\ensuremath{\mathit{safeK}(#1,#2,#3,#4,#5)}}

\newcommand{\refines}[0]{\ensuremath{\prec}}

\newcommand{\qrefines}[0]{\ensuremath{\mathop{\prec_Q}}}

\newcommand{\pruned}[1]{\ensuremath{\overline{ #1} }}

\newcommand{\sem}[1]{\ensuremath{\llbracket #1 \rrbracket}}

\newcommand{\progs}[0]{\ensuremath{\mathcal{P}}}

\newcommand{\id}[0]{\ensuremath{\mathit{x}}}

\newcommand{\xevent}[0]{\ensuremath{\nu}}
\newcommand{\cevent}[0]{\ensuremath{\mu}}
\newcommand{\event}[0]{\ensuremath{\iota}}

\newcommand{\evalue}[0]{\ensuremath{v}}

\newcommand{\intval}[1]{\ensuremath{\mathsf{int}(#1)}}

\newcommand{\floatval}[1]{\ensuremath{\mathsf{float}(#1)}}

\newcommand{\call}[1]{\ensuremath{\mathsf{call}(#1)}}
\newcommand{\return}[1]{\ensuremath{\mathsf{ret}(#1)}}
\newcommand{\malloc}[1]{\ensuremath{\mathsf{malloc}(#1)}}
\newcommand{\free}[1]{\ensuremath{\mathsf{free}(#1)}}

\newcommand{\trace}[0]{\ensuremath{t}}

\newcommand{\Trace}[0]{\ensuremath{T}}

\newcommand{\conv}[2]{\ensuremath{\mathsf{conv}(#1,#2)}}

\newcommand{\divt}[1]{\ensuremath{\mathsf{div}(#1)}}

\newcommand{\fail}[1]{\ensuremath{\mathsf{fail}(#1)}}

\newcommand{\behav}[0]{\ensuremath{B}}

\newcommand{\getchar}[0]{\ensuremath{\mathsf{getchar}}}

\newcommand{\Z}[0]{\ensuremath{\mathbb{Z}}}
\newcommand{\N}[0]{\ensuremath{\mathbb{N}}}
\newcommand{\Qplusz}[0]{\ensuremath{\mathbb Q^+_0}}
\newcommand{\Q}[0]{\ensuremath{\mathbb{Q}}}

\newcommand{\weight}[2]{\ensuremath{W_{#2}(#1)}}
\newcommand{\tval}[2]{\ensuremath{V_{#2}(#1)}}

\newcommand{\prefs}[1]{\ensuremath{\mathit{prefs}(#1)}}

\newcommand{\type}[1]{\ensuremath{\mathsf{#1}}}

\newcommand{\code}[1]{\ensuremath{\mathsf{#1}}}

\newcommand{\env}[0]{\ensuremath{\mathit{\theta}}}

\newcommand{\mem}[0]{\ensuremath{\mathit{H}}}

\newcommand{\Mem}[0]{\ensuremath{\mathit{Mem}}}

\newcommand{\Var}[0]{\ensuremath{\mathit{Var}}}

\newcommand{\Val}[0]{\ensuremath{\mathit{Val}}}

\newcommand{\Aux}[0]{\ensuremath{\mathit{Aux}}}

\newcommand{\Input}[0]{\ensuremath{\mathit{StdIn}}}

\newcommand{\Assn}[0]{\ensuremath{\mathit{Assn}}}
\newcommand{\State}[0]{\ensuremath{\mathit{State}}}
\newcommand{\state}[0]{\ensuremath{\sigma}}

\newcommand{\Prop}[0]{\ensuremath{\mathit{Prop}}}

\newcommand{\ret}[0]{\ensuremath{\mathit{ret}}}

\newcommand{\htriple}[3]{\ensuremath{\{#1\}\, #2\, \{#3\}}}

\newcommand{\smallstep}[1]{\ensuremath{\to_{#1}}}

\newcommand{\conform}[2]{\ensuremath{\mathit{agree}(#1,#2)}}

\newcommand{\dom}[1]{\ensuremath{ \text{dom}(#1)}}
\newcommand{\img}[1]{\ensuremath{ \text{img}(#1)}}

\newcommand{\pimp}[2]{\ensuremath{#1 \Rightarrow #2}}

\newcommand{\Assert}[0]{\ensuremath{\mathit{Assert}}}

\newcommand{\VAssert}[0]{\ensuremath{\mathit{VAssert}}}

\newcommand{\eenv}[0]{\ensuremath{\Sigma}}

\newcommand{\genv}[0]{\ensuremath{\Delta}}

\newcommand{\fenv}[0]{\ensuremath{\Sigma}}

\newcommand{\Fundef}[0]{\ensuremath{\mathit{Fundef}}}

\newcommand{\funs}[0]{\ensuremath{\mathit{FID}}}

\newcommand{\vars}[0]{\ensuremath{\mathit{VID}}}

\newcommand{\statements}[0]{\ensuremath{\mathcal{S}}}

\newcommand{\traces}[0]{\ensuremath{\mathcal{T}}}

\newcommand{\behavs}[0]{\ensuremath{\mathcal{B}}}

\newcommand{\events}[0]{\ensuremath{\mathcal{E}}}

\newcommand{\initState}[2]{\ensuremath{\mathit{initSt}(#1,#2)}}

\newcommand{\retState}[1]{\ensuremath{\mathit{retState}(#1)}}

\newcommand{\Potential}[0]{\ensuremath{\mathit{Pot}}}

\newcommand{\evalE}[2]{\ensuremath{\llbracket #1 \rrbracket_{#2}}}

\newcommand{\Kseq}[2]{\ensuremath{\mathsf{Kseq}\, #1 \, #2}}

\newcommand{\Kloop}[2]{\ensuremath{\mathsf{Kloop}\, #1 \, #2}}

\newcommand{\Kstop}[0]{\ensuremath{\mathsf{Kstop}}}

\newcommand{\Kcall}[3]{\ensuremath{\mathsf{Kcall}\, #1 \, #2 \, #3}}

\newcommand{\cont}[0]{\ensuremath{K}}

\newcommand{\cost}[0]{\ensuremath{c}}

\newcommand{\used}[0]{\ensuremath{\sqbullet}}

\newcommand{\init}[0]{\ensuremath{\square}}

\newcommand{\FV}[1]{\ensuremath{\text{FV}(#1)}}

\newcommand{\loc}[0]{\ensuremath{\text{Loc}}}

\newcommand{\inter}[2]{\ensuremath{[#1,#2]}}

\newcommand{\ind}[0]{\ensuremath{\mathcal{I}}}

\newcommand{\states}[0]{\ensuremath{\mathcal{H}}}

\newcommand{\shift}[0]{{\lhd}}

\newcommand{\Vret}[0]{{\ensuremath{\mathit{ret}}}}
\newcommand{\Vargs}[0]{{\ensuremath{\vec {\mathit{args}}}}}

\newcommand{\tr}[0]{\ensuremath{\mathcal T}}

%-------------------------------------------------------
%Type rules (mathpatir)

\newcommand{\Rule}[4][]{\ensuremath{\inferrule*[right={\!(#2)},#1]{#3}{#4}}}
\newcommand{\RuleToplabel}[4][]{\ensuremath{\inferrule[(#2)]{#3}{#4}}}
\newcommand{\RuleNolabel}[3][]{\ensuremath{\inferrule*[#1]{#2}{#3}}}

%-------------------------------------------------------


%%% Local Variables: 
%%% mode: latex
%%% TeX-master: "main"
%%% End: 


%------------Draft---------------------------------------
\usepackage{eso-pic}
\usepackage{xcolor}
\AddToShipoutPicture{\setlength{\unitlength}{1in}\put(4.25,10.0){\makebox[0pt][c]{\normalfont\textsf{\textbf{\large \color{gray} Draft -- \today}}}}}
%--------------------------------------------------------

\newcommand{\jan}[1]{{\color{blue}{[Jan: #1]}}}

\title{Automated Amortized Resource Bound Analysis \\ for Control Flow Graphs}

\author{Quentin Carbonneaux \and Jan Hoffmann}

\begin{document}

\maketitle


\section{Control Flow Graphs}

\paragraph{Straight Line Statements}

Straight line statements (also just statements) are inductively
defined by the following grammar.
%
\begin{align*}
S &:= \code{tick}(n) \mid \code{assert}~E
\mid \code{skip}
\mid x \gets E
\mid x \gets f(x^*)
\mid S;S
\end{align*}
%
Expressions $E$ are left abstract in our presentation.  For our
analysis framework, it is only important that they are side-effect
free. We write $\statements$ for the set of statements.

\paragraph{Control Flow Graphs}

A control flow graph (CFG) is a set a graph $G=(V,E,\labels)$ with
statement-labeled nodes. Here, $V$ is a set of nodes,
$E \subseteq V \times V$ is a set of edges, and
$\labels : V \to \statements$ is a mapping from nodes to statements.

\section{Operational Semantics of CFGs}


\jan{Here we probably need some non-deterministic choice on the
  edges. We have to watch out that this is consistent
  with the assertions that we insert in the blocks. We might have to
  put assertion on the edges to avoid accounting cost for a jump that
  might not happen.}


\section{Potential Annotations for CFGs}

\subsection{Linear Potential Functions.}

To find resource bounds automatically, we first need to restrict our
search space.  In this work, we focus on the following form of
potential functions, which can express tight bounds for many typical
programs and allows for inference with \emph{linear programming}.
$$
  \Phi(\state) = q_0 + \sum_{x, y \in \dom\state \land x \neq y}
    q_{(x, y)} \cdot |\inter {\state(x)} {\state(y)}| \, .
$$
%
Here $\state : (\text{Locals} \to \Z) \times (\text{Globals} \to \Z)$
is a simplified program state that maps variable names to integers,
$|\inter a b| = \max(0,b-a)$, and $q_i \in \Q$.  To simplify the
references to the linear coefficients $q_i$, we introduce an \emph{index
  set} $I$.  This set is defined to be $\{0\} \cup \{(x, y) \mid x, y
\in \Var \land x \neq y \}$.  Each index $i$ corresponds to a \emph{base function}
$f_i$ in the potential function: $0$ corresponds to the constant
function $\state \mapsto 1$, and $(x,y)$ corresponds to $\state \mapsto
|\inter {\state(x)} {\state(y)}|$.  Using these notations we can
rewrite the above equality as 
$$ \Phi(\state) = \sum_{i \in I} q_i f_i(\state).$$
%
We often write $xy$ to denote the
index $(x,y)$.  This allows us to
uniquely represent any linear potential function
$\Phi$ as a \emph{quantitative annotation} $Q = (q_i)_{i \in I}$, that
is, a family of rational numbers where only
a finite number of elements are not zero.  

In the potential functions, we treat constants as global variables
that cannot be assigned to.  For example, if the program contains the
constant $8128$ then we have a variable $c_{8128}$ and
$\state(c_{8128}) = 8128$.  We assume that every program state
includes the constant $c_0$.

\begin{figure*}
\small
%\vspace{-.4cm}
\def \MathparLineskip {\lineskip=0.35cm}
\begin{mathpar}
\Rule{Q:Skip}
{ }
{ \htriple{ \Gamma; Q }{ \code{skip} }{ \Gamma; Q } }
%
\and \Rule{Q:Assert}
{ }
{ \htriple{ \Gamma; Q {+} M_a }{ \code{assert}~e }{ \Gamma {\land} e; Q } }
%
\and \Rule{Q:Tick}
{ }
{ \htriple{ \Gamma; Q {+} M_t(n) }{ \code{tick}(n) }{ \Gamma; Q } }
%
\and \Rule{Q:Seq}
{  \htriple{ \Gamma; Q }{ S_1 }{ \Gamma'; Q' {+} M_s }
\\ \htriple{ \Gamma'; Q' }{ S_2 }{ \Gamma''; Q'' }
}
{ \htriple{ \Gamma; Q }{ S_1;S_2 }{ \Gamma''; Q'' } }
%
\\ \Rule{Q:IncP}
{ \Gamma \models y \ge 0 \!
\\ \mathcal U = \left\{ u \mid \Gamma \models x + y \in \inter x u \right\} \!
\\\\ \textstyle q'_{0y} = q_{0y}
      + \sum_{u \in \mathcal U} q_{xu}
      - \sum_{v \not\in \mathcal U} q_{vx} \!
}
{ \textstyle
  \htriple{ \Gamma[x/x {+} y]; Q {+} M_u {+} M_e(x {+} y) }{ x \gets x + y }{ \Gamma; Q' }
}
%
\and \Rule{Q:Inc}
{ M = M_u + M_e(x {\pm} y)
\\\\ \textstyle q'_{0y} = q_{0y} - \sum_{v} q_{vx}
\\ \textstyle q'_{y0} = q_{y0} - \sum_{v} q_{xv}
}
{ \textstyle
  \htriple{ \Gamma[x/x {\pm} y]; Q {+} M }{ x \gets x \pm y }{ \Gamma; Q' }
}
%
\and \Rule{Q:Set}
{ q'_{xy}, q'_{yx} \in \Qplusz
\\ \forall u. (q_{yu} = q'_{xu} + q'_{yu} \land q_{uy} = q'_{ux} + q'_{uy})
}
{ \htriple{ \Gamma[x/y]; Q {+} M_u {+} M_e(y) }{ x \gets y }{ \Gamma; Q' } }
%
\and \Rule{Q:DecP}
{ \Gamma \models y \ge 0
\\ \mathcal U = \left\{ u \mid \Gamma \models x - y \in \inter u x \right\}
\\\\ \textstyle q'_{y0} = q_{y0}
      + \sum_{u \in \mathcal U} q_{ux}
      - \sum_{v \not\in \mathcal U} q_{xv}
}
{ \textstyle
  \htriple{ \Gamma[x/x {-} y]; Q {+} M_u {+} M_e(x {-} y) }{ x \gets x - y }{ \Gamma; Q' }
}
% %
% \and\Rule{Q:Call}
% { (\Gamma_f; Q_f,  \Gamma_f'; Q_f') \in \Delta(f) \!
% \\ \text{Loc} = \text{Locals}(Q) \!
% \\ \forall i \neq j . \, x_i \neq x_j \!
% \\ c \in \Qplusz \!
% \\ Q = P + S \!
% \\ Q' = P' + S
% \\ U = Q_f[\Vargs / \vec x]
% \\ U' = Q'_f[\Vret/r]
% \\ \forall i \in \dom{U} . \, p_i = u_i
% \\ \forall i \in \dom{U'} .\, p'_i = u'_i
% \\ \forall i \not\in \dom{U'} .\, p'_i = 0
% \\ \forall i \not\in \text{Loc}
%   .\, s_i = 0
% }
% { B, R \vdash \htriple
%   { \Gamma_f[\Vargs / \vec x] {\land} \Gamma_{\text{Loc}}; Q {+} c {+} M_f }
%   { r \gets f(\vec x) }
%   { \Gamma_f'[\Vret/r] {\land} \Gamma_{\text{Loc}}; Q' {+} c {-} M_r }
% }
%
% \and \Rule{Q:Extend}
% { \fenv f = (\vec y, S_f)
% \\\\ B, (\Gamma_f'; Q'_f) \vdash \htriple{ \Gamma_f[\Vargs/\vec y]; Q_f[\Vargs/ \vec y] }{ S_f }{ \Gamma'; Q' }
% }
% { (\Gamma_f; Q_f, \Gamma_f'; Q_f') \in \Delta(f) }
%
\and \Rule{Q:Weak}
{ \htriple{ \Gamma_2; Q_2 }{ S }{ \Gamma'_2; Q'_2 }
\\ \!\! \Gamma_1 \models \Gamma_2
\\\\ Q_1 \succeq_{\Gamma_1} Q_2
\\ \Gamma'_2 \models \Gamma'_1
\\ Q'_2 \succeq_{\Gamma_2'} Q'_1
}
{ \htriple{ \Gamma_1; Q_1 }{ S }{ \Gamma'_1; Q'_1 } }
%
\\ \Rule{Relax}
{ \mathcal L = \{ xy \mid \exists l_{xy} {\in} \N \, . \, \Gamma \models l_{xy} \le |\inter x y| \}
\\ \mathcal U = \{ xy \mid \exists u_{xy}{\in} \N \, . \, \Gamma \models |\inter x y| \le u_{xy} \}
\\\\ \forall i \in \mathcal U.\, q'_i \ge q_i - r_i
\\ \forall i \in \mathcal L.\, q'_i \ge q_i + p_i
\\ \forall i \not\in \mathcal U {\cup} \mathcal L {\cup} \{0\} .\, q'_i \ge q_i
\\ q'_0 \geq q_0 {\textstyle + \sum_{i \in \mathcal U} u_i r_i - \sum_{i \in \mathcal L} l_i p_i}
}
{ Q' \succeq_\Gamma Q }
%\vspace{-.3cm}
\end{mathpar}
\caption{Inference rules of the straight-line statements.}
\label{fig:auto}
\end{figure*}

\subsection{Analyzing Straight Line Statements}

In the rules, assignments are restricted to the form $x \gets
y$ or $x \gets x \pm y$.  In the implementation, a Clight
program is converted into this form prior to analysis without changing
the resource cost.
Non-linear operations such as $x \gets z*y$ or $x \gets
a[y]$ are handled by assigning $0$ to coefficients
like $q_{xa}$ and $q_{ax}$ that contain $x$ after the assignment.
This sound treatment ensures that no further loop bounds
depend on the result of the non-linear operation.

\paragraph{Judgements.}

The derivation system for the automatic amortized analysis is defined
in~\pref{fig:auto}.  The derivation rules yield judgements of the form
$$
    \htriple{ \Gamma; Q }{ S }{ \Gamma'; Q' }.
$$
%
The judgement
can be seen as a quantitative Hoare triple.  All assertions are split into
two parts, the logical part and the quantitative part.  The
quantitative part $Q$ represents a potential function as a collection
of rational numbers $q_i$ indexed by the index set $I$.  The
logical part $\Gamma$ is left abstract but is enforced by our
derivation system to respect classic Hoare logic constraints.
%
The meaning of this basic judgment is as follows: If $S$ is executed
with starting state $\state$, the assertions in $\Gamma$
hold, $Q' {\ge} 0$, and at least $Q(\state)$ resources are available then the
evaluation does not run out of resources and, if the execution
terminates in state $\state'$, there are at least $Q'(\state')$
resources left and $\Gamma'$ holds for $\state'$.


As a convention, if $Q$ and $Q'$ are quantitative annotations
we assume that $Q = (q_i)_{i\in I}$ and $Q' = (q'_i)_{i \in I}$.
The notation $Q \pm n$ used in many rules defines a new context $Q'$
such that $q'_0 = q_0 \pm n$ and $\forall i \neq 0 .\, q'_i = q_i$.

In all the rules, we have the implicit side condition that all rational
coefficients are non-negative.
%
Finally, if a rule mentions $Q$ and $Q'$
and leaves the latter undefined at some index $i$ we assume that $q'_i = q_i$.

\subsection{Analyzing Control Flow Graphs}

Let $G = (V,E,\labels)$ be a CFG. A potential annotation $\mathcal{P}$
for $G$ is mapping from notes $v \in V$ to pairs
$\left<(\Gamma,Q);(\Gamma',Q')\right>$ of quantitative assertions so
that the following holds.
%
\begin{enumerate}
\item  For every edge $(v_1,v_2) \in E$: 
  If $\mathcal{P}(v_1) = \left<(\Gamma_1,Q_1);(\Gamma'_1,Q'_1)\right>$ and 
   $\mathcal{P}(v_2) = \left<(\Gamma_2,Q_2);(\Gamma'_2,Q'_2)\right>$ then
   $\Gamma'_1 \models \Gamma_2$ and $Q_1' \geq Q_2$.
 \item For every node $v \in V$: If $\labels(v) = S$ and
   $\mathcal{P}(v) = \left<(\Gamma,Q);(\Gamma',Q')\right>$ then
  $\htriple{(\Gamma,Q)}{S}{(\Gamma',Q')}$.
\end{enumerate}

\bibliographystyle{plain}
\bibliography{lit}


\end{document}



%%% Local Variables: 
%%% mode: latex
%%% mode: flyspell
%%% TeX-master: t
%%% End: 
